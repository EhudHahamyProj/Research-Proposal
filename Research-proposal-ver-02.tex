\documentclass[12pt,twoside]{article} 
\usepackage[reqno,tbtags]{amsmath}
\usepackage[active]{srcltx} % not for published version
\usepackage{amsthm}
\usepackage{amsfonts}
\usepackage{amssymb}
\usepackage{graphicx}
\usepackage{subfigure}
\usepackage{verbatim}
% put any defs here
\newcommand{\mat}[2][cccccccccccccccccccccccccccccccccccccccccccc]{\left(\begin{array}{#1}#2 \\ \end{array} \right)}

\newtheorem{lem}{Lemma}[subsubsection]
\newtheorem{thm}{Theorem}[subsection]
\newtheorem{cor}{Corollary}[subsubsection]
\newtheorem{define}{Definition}[subsubsection]


\begin{document}

\title{Numerical schemes for the 1-D wave equation}

\newpage

\section{From Maxwell's equations to generalized Helmholtz equations system in heterogeneous media}

\subsection{Maxell's equation in MKS units}
We state Maxwell's equations in differential form.
\begin{subequations}

	\begin{align}
        \nabla \times \underline{H}-\dfrac{\partial \underline{D}}{\partial t} &= \underline{J}  \label{eqns:CurlH_Maxwell}	\\	
        \nabla \times \underline{E} + \dfrac{\partial \underline{B}}{\partial t}&= 0  \label{eqns:CurlE_Maxwell} \\
		\nabla \cdot \underline{D} &=  \rho	 \label{eqns:DivD_Maxwell}\\		
		\nabla \cdot \underline{B} &= 0	
	\end{align}
\end{subequations}
$\underline{H}$ is the magnetic field, $\underline{E}$ is the electic field, $\underline{D}$ is the electric displacement vector, $\underline{B}$ is the magnetic flux density, $\underline{J}$ is the electric current density, and $\rho$ is the electric charge density.\\
In this work we assume the linear constitutive relations 
\begin{subequations}
\begin{align}
        \underline{D} = \epsilon \underline{E} \label{eqns:constitutiveDepsE}\\
        \underline{B} = \mu \underline{H}	\label{eqns:constitutiveBH}\\
        \underline{J} = \sigma \underline{E}
\end{align}
\end{subequations}
The dielectric permittivity $\epsilon$, the magnetic permeability $\mu$ and the conductivity $\sigma$ are assumed a differentiable function of $x$, $y$, $z$ and time independent.



\subsection{Transforming Maxwell's equations into generalized wave equations in heterogeneous media}

In order to transform Maxwell's equation into a system of wave equations, we need the following set of vector calculus identities:
\begin{subequations}
\label{eqns:vectorCalcIDs}
	\begin{align}
			\nabla \times (\psi \underline{a}) = \nabla \psi \times 	
			\underline{a} + \psi \nabla \times \underline{a}  \label{eqns:ProdCurlID}\\
		\nabla \times \nabla \times \underline{a} = \nabla ( \nabla
			 \cdot \underline{a} ) - \nabla^2\underline{a} \label{eqns:CurlCurlID}\\
		\nabla \cdot (\psi \underline{a}) = \psi \nabla \cdot \underline{a}
			 + \underline{a} \cdot \nabla \psi \label{eqns:DivProdID}
			\end{align}
\end{subequations}

We start from equation (\ref{eqns:CurlE_Maxwell})
\begin{align*}
\nabla \times \underline{E} + \dfrac{\partial \underline{B}}{\partial t}&= 0
\end{align*}
Substituting the relation (\ref{eqns:constitutiveBH}) between magnetic fields in conjunction we get
\begin{align*}
\nabla \times \underline{E} + \mu \dfrac{\partial  \underline{H}}{\partial t}&= 0
\end{align*}
For future purposes, let us divide by $\mu$:
\begin{align*}
\dfrac{1}{\mu} \nabla \times \underline{E} + \dfrac{\partial \underline{H}}{\partial t} = 0
\end{align*}

Taking the curl of both sides and changing the order of temporal and spatial differentiation
\begin{align}
\label{eqns:CulCurlMaxwellModified}
\nabla \times \left( \dfrac{1}{\mu} \nabla \times \underline{E} \right)  + \dfrac{\partial (\nabla \times  \underline{H})}{\partial t} = 0
\end{align}
We'll apply thew vector calculus identities (\ref{eqns:vectorCalcIDs}) to decompose the summed terms in (\ref{eqns:CulCurlMaxwellModified}).\\
Using identity (\ref{eqns:ProdCurlID}) we get
\begin{align}
\label{eqns:CurlProdCurlpart}
\nabla \times \left( \dfrac{1}{\mu} \nabla \times \underline{E} \right) = \nabla \left( \dfrac{1}{\mu} \right) \times \left(\nabla \times \underline{E} \right) + \dfrac{1}{\mu} \nabla \times \nabla \times \underline{E}
\end{align}
Now we apply (\ref{eqns:CurlCurlID}) to (\ref{eqns:CurlProdCurlpart}) and get
\begin{align}
\label{eqns:CurlProdCurlExpanded}
\nabla \times \left( \dfrac{1}{\mu} \nabla \times \underline{E} \right) = \nabla \left( \dfrac{1}{\mu} \right) \times \left(\nabla \times \underline{E} \right) + \dfrac{1}{\mu} \left( \nabla (\nabla \cdot \underline{E}) - \nabla^2 \underline{E}\right)
\end{align}
The $\underline{H}$ - term in (\ref{eqns:CulCurlMaxwellModified}) can be replaced using equation (\ref{eqns:CurlH_Maxwell}):
\begin{align}
\label{eqns:Curl_H_replaced_D}
\dfrac{\partial}{\partial t} \left( \nabla \times \underline{H}\right) = \dfrac{\partial^2 \underline{D}}{\partial t^2} + \dfrac{\partial \underline{J}}{\partial t}
\end{align}

and using the constitutive relation (\ref{eqns:constitutiveDepsE}), equation (\ref{eqns:Curl_H_replaced_D}) becomes
\begin{align}
\label{eqns:Curl_H_replaced_E}
\dfrac{\partial}{\partial t} \left( \nabla \times \underline{H}\right) = \epsilon\dfrac{\partial^2 \underline{E}}{\partial t^2} + \dfrac{\partial \underline{J}}{\partial t}
\end{align}

Substituting (\ref{eqns:CurlProdCurlExpanded}), (\ref{eqns:Curl_H_replaced_E}) into (\ref{eqns:CulCurlMaxwellModified}) and multiplying by we get $\mu$ we get an equation in $\underline{E}$:
\begin{align}
\label{eqns:E_Wave_full}
\mu \nabla \left( \dfrac{1}{\mu} \right) \times \left(\nabla \times \underline{E} \right) +  \nabla (\nabla \cdot \underline{E}) - \nabla^2 \underline{E} + \mu \epsilon\dfrac{\partial^2 \underline{E}}{\partial t^2} + \mu \dfrac{\partial \underline{J}}{\partial t} = 0
\end{align}

We'll refer to the system (\ref{eqns:E_Wave_full}) as the generalized wave equations system.


\subsection{Analysis of the source and material-dependent terms in the generalized wave equations system}

Compared to the standard wave equation $\dfrac{1}{v^2}\dfrac{\partial ^2\underline{E}}{\partial t^2} =  \nabla^2 \underline{E}$, equation (\ref{eqns:E_Wave_full}) has extra terms:
\begin{subequations}
\begin{align}
\nabla \left( \log (\mu) \right) \times \left(\nabla \times \underline{E} \right) \label{eqns:E_Wave_extra_term_log_mu}\\
\nabla (\nabla \cdot \underline{E}) \label{eqns:E_Wave_extra_term_Grad_Div_E} \\
\mu \dfrac{\partial \underline{J}}{\partial t} \label{eqns:E_Wave_extra_term_mu_DJDt} 
\end{align}
\end{subequations}
While the term (\ref{eqns:E_Wave_extra_term_mu_DJDt})is a straight- forward source term, the other two have a more complicated structure. In this section we'll describe the contributions of the terms (\ref{eqns:E_Wave_extra_term_log_mu}), (\ref{eqns:E_Wave_extra_term_Grad_Div_E}) to each equation in the generalized wave equations system.

\subsubsection{The term $\nabla \left( \log (\mu) \right) \times \left(\nabla \times \underline{E} \right)$}

In this work we are primarily interested in visible light optics. Materials which are transparent in visible light are essentially ``nonmagnetic'' (see \cite{Hecht2002}, chapter 3). Evidently, the term (\ref{eqns:E_Wave_extra_term_log_mu}) vanishes if $\mu$ is constant, and specifically for nonmagnetic substances satisfying $\mu \equiv 1$. Still, there exist magnetic substances that are transparent in the infrared and microwave regions of the spectrum. For the sake of completeness, we'll decompose the term to understand its contribution to each wave equation in  the system (\ref{eqns:E_Wave_full}).\\

To handle the repeated cross product we'll use Levi-Civita's permutation symbols $\epsilon_{ijk}$ \cite{ArfkenWeber2005} defined by
\begin{align}
\label{eqns:LeviCivitaSymbol}
\epsilon_{123} = \epsilon_{231} = \epsilon_{312} & = 1	\notag\\
\epsilon_{132} = \epsilon_{213} = \epsilon_{321} &= -1 \\
\text{all other } \epsilon_{ijk} &= 0	\notag
\end{align}
The following lemma is presented as an exercise in \cite{ArfkenWeber2005}, p. 150.
\begin{lem}
\label{lem:LiviCivitaEpsDelta}
Let $\epsilon_{ijk}$ be defined by (\ref{eqns:LeviCivitaSymbol}). Then
\begin{align}
\label{eqns:LiviCivitaEpsilonDelta} 
\epsilon_{ijk} \epsilon_{pqk} = \delta_{ip} \delta_{jq} - \delta_{iq} \delta_{jp}
\end{align} 
where $\delta_{ij}$ is Kronecker's delta.
\end{lem}

\begin{proof}
Noticed that each term in the $k$-sum $\epsilon_{ijk} \epsilon_{pqk}$ is non-vanishing only if $i,j,p,q \neq k$, $i \neq j$, $p \neq q$. For example, for the $k=1$ term we need only to examine the possibilities 
\begin{align*}
i=2,\quad j=3, \quad p=2, \quad q=3	\\
i=3,\quad j=2, \quad p=2, \quad q=3	\\
i=2,\quad j=3, \quad p=3, \quad q=2	\\
i=3,\quad j=2, \quad p=3, \quad q=2
\end{align*} 
Substituting and using (\ref{eqns:LeviCivitaSymbol}) we get that in all cases  $\epsilon_{ij1} \epsilon_{pq1} =  \delta_{ip} \delta_{jq} - \delta_{iq} \delta_{jp}$.\\
Similar results can be shown for $k=2,3$. Finally, the summation (\ref{eqns:LiviCivitaEpsilonDelta}) holds because $i,j,p,q$ substitutions that yield a non-vanishing for a specific value of $k$ cause the other $k$-terms to vanish.
\end{proof}

\begin{lem}
Let $\underline{a}$ be a vector of $x,y,z$-differentiable entries. Then 
\begin{align}
\label{eqns:doubleCrossProdDecomposition}
\left( \underline{a} \times (\nabla \times \underline{E}) \right)_i = \underline{a} \cdot \left( \dfrac{\partial \underline{E}}{\partial x_i}- \nabla E_i \right)
\end{align}
\end{lem}

\begin{proof}
Using the Levi-Civita symbol (\ref{eqns:LeviCivitaSymbol}), 
\begin{align*}
\left( \underline{a} \times (\nabla \times \underline{E}) \right)_i  = \epsilon_{ijk}a_j\left( \epsilon_{kpq}\dfrac{\partial E_q}{\partial x_p} \right) = \epsilon_{ijk} \epsilon_{kpq} a_j \dfrac{\partial E_q}{\partial x_p}
\end{align*}
By lemma (\ref{lem:LiviCivitaEpsDelta})
\begin{align*}
\left( \underline{a} \times (\nabla \times \underline{E}) \right)_i  =  \left( \delta_{ip}\delta{jq} - \delta_{iq}\delta_{jp}\right) a_j \dfrac{\partial E_q}{\partial x_p}
\end{align*}
but
\begin{align*}
\delta_{ip}\delta_{jq}a_j \dfrac{\partial E_q}{\partial x_p} &= a_j \dfrac{\partial E_j}{\partial x_i} \\
\delta_{iq}\delta_{jp}a_j \dfrac{\partial E_q}{\partial x_p} &= a_j \dfrac{\partial E_i}{\partial x_j}
\end{align*}
Hence
\begin{align*}
\left( \underline{a} \times (\nabla \times \underline{E}) \right)_i = a_j\left( \dfrac{\partial E_j}{\partial x_i} - \dfrac{\partial E_i}{\partial x_j} \right)
\end{align*}
Explicit summation yields
\begin{align*}
\left( \underline{a} \times (\nabla \times \underline{E}) \right)_i &= 
a_1\left( \dfrac{\partial E_1}{\partial x_i} - \dfrac{\partial E_i}{\partial x_1} \right) +
a_2\left( \dfrac{\partial E_2}{\partial x_i} - \dfrac{\partial E_i}{\partial x_2} \right) +
a_3\left( \dfrac{\partial E_3}{\partial x_i} - \dfrac{\partial E_i}{\partial x_3} \right) \\
&= \underline{a} \cdot \left( \dfrac{\partial \underline{E}}{\partial x_i}- \nabla E_i \right)
\end{align*}
as required.
\end{proof}

\begin{cor}
The term (\ref{eqns:E_Wave_extra_term_log_mu}) $i$'th component has the form 
\begin{align}
\label{eqns:E_Wave_extra_term_log_mu_expanded}
\left( \nabla \left( \log \mu \right) \times \left(\nabla \times \underline{E} \right) \right)_i = 
\nabla (\log \mu) \cdot \left( \dfrac{\partial \underline{E}}{\partial x_i}- \nabla E_i \right)
\end{align}
\end{cor}





\subsubsection{The term $\nabla (\nabla \cdot \underline{E})$}
To decompose the term (\ref{eqns:E_Wave_extra_term_Grad_Div_E}), let us substitute the constitutive relation (\ref{eqns:constitutiveDepsE}) into equation (\ref{eqns:DivD_Maxwell}). We get
\begin{align*}
\nabla \cdot (\epsilon \underline{E}) = \rho
\end{align*}
Using the product rule for differentiation (\ref{eqns:DivProdID}), 
\begin{align*}
\epsilon\nabla \cdot \underline{E} + \underline{E} \cdot \nabla \epsilon = \rho
\end{align*}
Hence
\begin{align}
\label{eqns:DivE}
\nabla \cdot \underline{E} = \dfrac{\rho}{\epsilon} - \underline{E} \cdot \nabla (\log \epsilon)
\end{align}

Taking the gradient of both sides of (\ref{eqns:DivE}) we get
\begin{align*}
\nabla \left( \nabla \cdot \underline{E} \right)_i &=\nabla\left( \dfrac{\rho}{\epsilon}\right)_i-\dfrac{\partial}{\partial x_i}\left( E_j \dfrac{\partial \log(\epsilon)}{\partial x_j}\right) \\&= \nabla\left( \dfrac{\rho}{\epsilon}\right)_i - \left( \dfrac{\partial E_j}{\partial x_i}\dfrac{\partial \log(\epsilon)}{\partial x_j} + E_j \dfrac{\partial^2 \log(\epsilon)}{\partial x_i \partial x_j}\right)
\end{align*}
Hence
\begin{align}
\label{eqns:E_Wave_extra_term_Grad_Div_E_expanded}
\nabla \left( \nabla \cdot \underline{E} \right)_i = \nabla\left( \dfrac{\rho}{\epsilon}\right)_i - \left(\dfrac{\partial \underline{E}}{\partial x_i} \cdot \nabla \log(\epsilon) + \underline{E} \cdot \dfrac{\partial}{\partial x_i} \nabla \log(\epsilon) \right)
\end{align}

\subsubsection{A 1-D layered media - system coefficients depending on a single coordinate}

Let us now assume that the $\epsilon = \epsilon(z)$, $\mu=\mu(z)$, $\rho = \rho(z)$, and see what can be learned.\\
Using (\ref{eqns:E_Wave_extra_term_log_mu_expanded}),The term (\ref{eqns:E_Wave_extra_term_log_mu}) has the following form, component-wise:
\begin{subequations}
\begin{align}
\left(\nabla \left( \log (\mu) \right) \times \left(\nabla \times \underline{E} \right)\right)_1 &= \dfrac{\mu'(z)}{\mu(z)}\left( -\dfrac{\partial E_1}{\partial z} + \dfrac{\partial E_3}{\partial x}\right) \label{eqns:LogTerm_z_only_Eq1}\\
\left(\nabla \left( \log (\mu) \right) \times \left(\nabla \times \underline{E} \right)\right)_2 &= \dfrac{\mu'(z)}{\mu(z)}\left( -\dfrac{\partial E_2}{\partial z} + \dfrac{\partial E_3}{\partial y}\right) \label{eqns:LogTerm_z_only_Eq2} \\
\left(\nabla \left( \log (\mu) \right) \times \left(\nabla \times \underline{E} \right)\right)_3 &= 0 \label{eqns:LogTerm_z_only_Eq3}
\end{align}
\end{subequations}
Similarly, we use (\ref{eqns:E_Wave_extra_term_Grad_Div_E_expanded}) to bring the term (\ref{eqns:E_Wave_extra_term_Grad_Div_E}) components to
\begin{subequations}
\begin{align}
(\nabla (\nabla \cdot \underline{E}))_1 &= -\dfrac{\epsilon'(z)}{\epsilon(z)} \dfrac{\partial E_3}{\partial x}	\label{eqns:GradDivTerm_z_only_Eq1}\\
(\nabla (\nabla \cdot \underline{E}))_2 &= -\dfrac{\epsilon'(z)}{\epsilon(z)} \dfrac{\partial E_3}{\partial y} \label{eqns:GradDivTerm_z_only_Eq2}\\
(\nabla (\nabla \cdot \underline{E}))_3 &= -\dfrac{\rho(z)}{\epsilon(z)}\epsilon'(z) + \dfrac{\rho'(z)}{\epsilon(z)}-\left( \dfrac{\epsilon''(z)}{\epsilon(z)}-\left( \dfrac{\epsilon'(z)}{\epsilon(z)}\right)^2 \right)E_3 \label{eqns:GradDivTerm_z_only_Eq3}\\
& -\dfrac{\epsilon'(z)}{\epsilon(z)}\dfrac{\partial E_3}{\partial z}	\notag
\end{align}
\end{subequations}

Plugging (\ref{eqns:LogTerm_z_only_Eq3}), (\ref{eqns:GradDivTerm_z_only_Eq3}) into the third component equation in (\ref{eqns:E_Wave_full}) we get an equation in $E_3$ only. Moreover, the first component equation in (\ref{eqns:E_Wave_full}) is $E_1$, $E_3$ - dependent equation, as can be seen by plugging (\ref{eqns:LogTerm_z_only_Eq1}), (\ref{eqns:GradDivTerm_z_only_Eq1}) into it. Since $E_3$ can be found independently, we have an equation in $E_1$. Similar reasoning yields an equation in $E_2$ from the second component equation in (\ref{eqns:E_Wave_full}), thus completing an uncoupling process. The components equations are
\begin{subequations}
\begin{align}
\dfrac{\mu'(z)}{\mu(z)}\left( -\dfrac{\partial E_1}{\partial z} + \dfrac{\partial E_3}{\partial x}\right)-\dfrac{\epsilon'(z)}{\epsilon(z)}\dfrac{\partial E_3}{\partial x}-\nabla^2 E_1 +\epsilon \mu\dfrac{\partial^2 E_1}{\partial t^2} &= -\mu(z)\dfrac{\partial J_1}{\partial t}	\label{eqns:Generalized_wave_z_coeffs_Eq1}\\
\dfrac{\mu'(z)}{\mu(z)}\left( -\dfrac{\partial E_2}{\partial z} + \dfrac{\partial E_3}{\partial y}\right)-\dfrac{\epsilon'(z)}{\epsilon(z)}\dfrac{\partial E_3}{\partial y}-\nabla^2 E_2 + \epsilon \mu\dfrac{\partial^2 E_2}{\partial t^2} &= -\mu(z)\dfrac{\partial J_2}{\partial t}	\label{eqns:Generalized_wave_z_coeffs_Eq2}\\
-\left( \dfrac{\epsilon''(z)}{\epsilon(z)}-\left( \dfrac{\epsilon'(z)}{\epsilon(z)}\right)^2 \right)E_3 -\dfrac{\epsilon'(z)}{\epsilon(z)}\dfrac{\partial E_3}{\partial z} - \nabla^2 E_3 +\epsilon \mu\dfrac{\partial^2 E_3}{\partial t^2} &= -\mu(z)\dfrac{\partial J_3}{\partial t} + \dfrac{\rho(z)}{\epsilon(z)}\epsilon'(z) - \dfrac{\rho'(z)}{\epsilon(z)} \label{eqns:Generalized_wave_z_coeffs_Eq3}
\end{align}
\end{subequations}
and in the time-harmonic case $\underline{E} = \underline{\hat{E}}e^{i \omega t}$, $\underline{J} = \underline{\hat{J}}e^{i \omega t}$:

\begin{subequations}
\begin{align}
\dfrac{\mu'(z)}{\mu(z)}\left( -\dfrac{\partial \hat{E}_1}{\partial z} + \dfrac{\partial \hat{E}_3}{\partial x}\right)-\dfrac{\epsilon'(z)}{\epsilon(z)}\dfrac{\partial \hat{E}_3}{\partial x}-\nabla^2 \hat{E}_1 - \epsilon \mu \omega^2 \hat{E}_1 &= -i\omega \mu(z)\hat{J}_1	\label{eqns:Generalized_Helmholtz_z_coeffs_Eq1}\\
\dfrac{\mu'(z)}{\mu(z)}\left( -\dfrac{\partial \hat{E}_2}{\partial z} + \dfrac{\partial \hat{E}_3}{\partial y}\right)-\dfrac{\epsilon'(z)}{\epsilon(z)}\dfrac{\partial \hat{E}_3}{\partial y}-\nabla^2 \hat{E}_2 - \epsilon \mu \omega^2 \hat{E}_2 &= -i \omega \mu(z)\hat{J}_2	\label{eqns:Generalized_Helmholtz_z_coeffs_Eq2}\\
\left( \dfrac{\epsilon''(z)}{\epsilon(z)}-\left( \dfrac{\epsilon'(z)}{\epsilon(z)}\right)^2 \right)\hat{E}_3 +\dfrac{\epsilon'(z)}{\epsilon(z)}\dfrac{\partial \hat{E}_3}{\partial z} + \nabla^2 \hat{E}_3 +\epsilon \mu \omega^2 \hat{E}_3 &= i \omega \mu(z)\hat{J}_3 - \dfrac{\rho(z)}{\epsilon(z)}\epsilon'(z) + \dfrac{\rho'(z)}{\epsilon(z)} \label{eqns:Generalized_Helmholtz_z_coeffs_Eq3}
\end{align}
\end{subequations}



\subsection{Electric field depending on the direction of propagation coordinate only - homogeneous case}
\subsubsection{Construction of an ODE boundary value problem}
As a first toy problem, let us consider the case where $\hat{E}_3 = \hat{E}_3(z)$, and the source terms $\underline{J}$, $\rho$ vanish. Under these assumptions, equation (\ref{eqns:Generalized_Helmholtz_z_coeffs_Eq3}) becomes a linear, homogeneous ODE with variable coefficients

\begin{align}
\label{eqns:1-D-time-harmonic-E3_z}
\dfrac{d^2 \hat{E}_3}{d z^2} + \dfrac{\epsilon'(z)}{\epsilon(z)}\dfrac{d \hat{E}_3}{d z} + \left( \dfrac{\epsilon''(z)}{\epsilon(z)}-\left( \dfrac{\epsilon'(z)}{\epsilon(z)}\right)^2 + \epsilon(z) \mu \omega^2\right)\hat{E}_3 = 0
\end{align}
We wish to model a plane wave propagation within a heterogeneous layer. The wave may be transmitted into the layer, reflected from it or (typically) a combination of both. The time-harmonic source at the left boundary is thus
\begin{align}
\hat{E}_3(-L) = A e^{i \omega_i}, \qquad A>0, \quad L>0 
\label{eqns:1-D-time-harmonic-z-boundary-source}
\end{align}
where the index $i$ in $\omega_i$ stands for `incidence', as opposed to the Fourier transform variable $\omega$.\\
At the current stage we wish to model the effect of one boundary only. While proper determination of a boundary condition at $+\infty$ is sufficient for an analytic formulation of such a problem, we want to have some numerical perspective as well. A natural way of modeling it is to define a transmission-only boundary condition at some point to the right of $-L$, say $+L$. to avoid reflection, we define a one-sided, right traveling wave at the right boundary. The wave speed is determined by the product $\epsilon \mu$, but note that it should be adapted to a first order PDE formulation:
\begin{align}
\label{eqns:1_D-temporal-traveling-wave-bc}
\left[\dfrac{\partial E_3}{\partial t} +  \dfrac{1}{\sqrt{\epsilon(L)\mu}}\dfrac{\partial E_3}{\partial z} \right ]_{z=L} = 0
\end{align}
and in the time-harmonic case
\begin{align}
\label{eqns:1_D-time-harmonic-traveling-wave-bc}
\left[ i \omega \hat{E}_3 + \dfrac{1}{\sqrt{\epsilon(L)\mu}} \dfrac{\partial \hat{E}_3}{\partial z} \right]_{z = L}=0
\end{align}
  
\subsubsection{Dimensional analysis and the meaning of layer thickness}
One should be aware that the form (\ref{eqns:1-D-time-harmonic-z-boundary-source}) is inadequate for the purpose of parametric and asymptotic analysis. The reason is that series expansions a frequently applied to the solution's coefficients. Using them without care spoils the equation's dimensional consistency. For example, if $\epsilon(z) = \arctan(z)$, series expansion yields $\arctan(z) = z-\dfrac{z^3}{3} + \ldots$. Since $z$ has a length dimension, the series expansion is not dimensionally consistent. Moreover, asymptotic methods for constructing semi-analytic solutions rely heavily on neglecting small terms. To understand the meaning of 'small', parameters should be organized in dimensionless groups. In our case, we wish to determine the relation between the layer's thickness, the incident plane wave frequency and the smooth cutoff function $\epsilon$'s properties: 
\begin{itemize}
\item The function's  dynamic range - the difference between its $-\infty$ and $+\infty$ values, normalized by the $-\infty$ limit to  get an 'effective jump' measure;
\item The function's proximity to a step function, essentially expressed by the size of the function's $z$-derivative at $z=0$ and the rate of its decay toward $\pm \infty$.
\end{itemize}

To obtain a non-dimensional argument for $\epsilon$, we let $\epsilon(z) = \epsilon(z/M)$, where $M$ has units of length. \\
To get non-dimensional groups of parameters, we perform the change of variables $z=M s$, in two steps. First we replace the independent variable in $\hat{E}_3(z))$:
\begin{align}
\label{eqns:1-D-time-harmonic-z-normalized-boundary-source}
\dfrac{1}{M^2}\dfrac{d^2 \hat{E}_3}{d s^2} + 
\dfrac{\epsilon'(z)}{M \epsilon(z)}\dfrac{d \hat{E}_3}{d s} + 
\left( \mu \omega^2\epsilon(z)-\left(\dfrac{\epsilon'(z)}{\epsilon(z)}\right)^2 + \dfrac{\epsilon''(z)}{\epsilon(z)}\right)\hat{E}_3 = 0 
\end{align}
Suppose that $\epsilon_1 = \epsilon(-L)$ and $\Delta \epsilon = \epsilon(L)-\epsilon(-L)$. Let $\epsilon(z) = A+Bc\left( \dfrac{z}{M}\right)$, where $c(z/M)$ is a differentiable, monotonous function of $z$ of the real line, and  $A$, $B$ are determined by the relations
\begin{align*}
\begin{cases}
A + Bc\left( -\dfrac{L}{M}\right) = \epsilon_1	\\
A + Bc\left( \dfrac{L}{M}\right) = (1+ r)\epsilon_1
\end{cases}
\end{align*}
where
\begin{align}
r = \dfrac{\Delta \epsilon}{\epsilon_1} 
\end{align} 
is a dimensionless parameter representing the layer's edge-to-edge difference of dielectric permeability, and
\begin{align}
a = \dfrac{L}{M}
\end{align} 
is a dimensionless parameter for the effective layer width. We solve
for $A$, $B$ and substitute backwards using $z = M s$ and get
\begin{align}
\label{eqns:1-D-time-harmonic-s-dielectric-cutoff}
\epsilon(s) = \dfrac{\epsilon_1 ((r+1) c(-a)-c(a)-r c(s))}{c(-a)-c(a)}
\end{align}

Substituting (\ref{eqns:1-D-time-harmonic-s-dielectric-cutoff}) into (\ref{eqns:1-D-time-harmonic-z-normalized-boundary-source}) while keeping track of the fact that the $\epsilon(z)$ derivatives should be calculated using the chain rule, we get

\begin{align}
\label{eqns:1_D-time-harmonic-s-E_3-dimensional}
\dfrac{d^2\hat{E}_3}{ds^2}&-\dfrac{r c'(s) }{(r+1) c(-a)-c(a)-r c(s)}\dfrac{d\hat{E}_3}{ds}+ \notag\\
&\Big(-\dfrac{r c''(s)}{(r+1) c(-a)-c(a)-r
   c(s)}-\dfrac{r^2 c'(s)^2}{(-(r+1) c(-a)+c(a)+r c(s))^2} \notag \\
   &-\dfrac{\mu  M^2 r \omega ^2
   \epsilon_1 c(s)}{c(-a)-c(a)}+\dfrac{\mu  M^2 r \omega ^2 \epsilon_1
   c(-a)}{c(-a)-c(a)}+\mu  M^2 \omega ^2 \epsilon_1\Big)\hat{E}_3(s)=0
\end{align}
Next we notice that the parameter grouping  
\begin{align}
\label{eqns:1_D-time-harmonic-normalized-freq}
\Omega = M \omega \sqrt{\epsilon_1 \mu}
\end{align}
is dimensionless, as $[\sqrt{\epsilon_1 \mu}] = \left[ \dfrac{1}{v}\right] = \dfrac{\text{time}}{\text{length}}$, while $[M] = \text{length}$ and $[\omega] = \dfrac{\text{rad}}{\text{time}}$. 
Substituting (\ref{eqns:1_D-time-harmonic-normalized-freq}) into (\ref{eqns:1_D-time-harmonic-s-E_3-dimensional}) we get a fully-non-dimensional form of the equation, dependent on three parameters only:
\begin{align}
\label{eqns:1_D-time-harmonic-s-E_3-non-dim}
\dfrac{d^2\hat{E}_3}{ds^2}&-\dfrac{r c'(s) }{(r+1) c(-a)-c(a)-r c(s)}\dfrac{d\hat{E}_3}{ds}+ \notag\\ 
&\Big(-\dfrac{r c''(s)}{(r+1) c(-a)-c(a)-r
   c(s)}-\dfrac{r^2 c'(s)^2}{(-(r+1) c(-a)+c(a)+r c(s))^2} \notag \\
   &-\dfrac{r \Omega ^2
   c(s)}{c(-a)-c(a)}+\dfrac{r \Omega ^2 c(-a)}{c(-a)-c(a)}+\Omega ^2\Big)\hat{E}_3=0
\end{align}

and the boundary conditions (\ref{eqns:1-D-time-harmonic-z-boundary-source}),
 (\ref{eqns:1_D-time-harmonic-traveling-wave-bc}) become
\begin{subequations}
\begin{align}
\label{eqns:1-D-time-harmonic-z-boundary-source-normalized}
&\hat{E}_3(-a) = Ae^{i \omega_i} \\
\label{eqns:1_D-time-harmonic-traveling-wave-bc-normalized}
&\left[ i \Omega \sqrt{1+r} \hat{E}_3 +  \dfrac{d \hat{E}_3}{d s} \right]_{s = a}=0
\end{align}
\end{subequations}


\begin{comment}
Letting the effective layer width $a \ll 1$ and neglecting terms multiplied by it, $\epsilon(a s)$ becomes nearly constant. In terms of perturbation theory, the approximation problem in powers of $a$ is regular and equation (\ref{eqns:1-D-time-harmonic-z-normalized-boundary-source}) has a zero-order approximation by the constant coefficients equation
\begin{align}
\label{eqns:1-D-time-harmonic-z-normalized-boundary-source-regular-per-const-coef}
\dfrac{d^2 \hat{E}_3}{d s^2} +   L^2\epsilon(0) \mu \omega^2 \hat{E}_3 = 0 
\end{align}
On the other hand, $a \gg 1$ yields a singular perturbation problem because naive expansion in powers of $\dfrac{1}{a}$ would yield an algebraic equation as a zero-order approximation for equation (\ref{eqns:1-D-time-harmonic-z-normalized-boundary-source}), which is not compatible with the boundary conditions. In such case, a totally new approximation method is required, based on singular perturbation theory. 

Considering again the easier case $a \ll 1$, we wish to quantify the concept of a layer thickness. To do so, we must take into account the magnitudes of the absolute thickness $L$, the source frequency $\omega$ and the inverse square wave speed $\epsilon \mu = \dfrac{1}{v^2}$.

Since equation (\ref{eqns:1-D-time-harmonic-z-normalized-boundary-source-regular-per-const-coef}) has constant coefficients, we are justified in using relations between the wavelength $\lambda$ and $\omega$:
\begin{align*}
\omega = \dfrac{2\pi v}{\lambda}
\end{align*}
Therefore
\begin{align}
L^2 \epsilon \mu \omega^2 = \dfrac{L^2}{v^2} \cdot \dfrac{(2\pi v)^2}{\lambda^2} = 4\pi^2 \dfrac{L^2}{\lambda^2}
\end{align}
we can therefore expect a solution to be non-oscillatory within a layer when $L \ll \lambda$.\\
In this work, we are dealing with visible light optics, where the wavelength $\lambda \approx 0.5 \cdot 10^{-6}m$. In such case, layers of width $10-100 \mu m$, typical to the semiconductor industry, are considered thick, while biological cells membrane, which is $3-10 nm$ thick, may be considered narrow. 
\end{comment}

 

\subsubsection{The ODE coefficients and discriminant for the $\tan^{-1}(s)$ cutoff  }
First let us construct a smooth, increasing cutoff function based on $\tan^{-1}(s)$. Substituting $\tan^{-1}(s)$ into (\ref{eqns:1-D-time-harmonic-s-dielectric-cutoff}), we get

\begin{align}
\label{eqns:epsilon_atan_cutoff}
\epsilon(s) = \dfrac{1}{2} \epsilon_1 \left(\dfrac{r \tan ^{-1}(s)}{\tan^{-1}(a)}+r+2\right)
\end{align}
The ODE (\ref{eqns:1-D-time-harmonic-E3_z}) has the general form 

\begin{align}
\label{eqns:general_ODE_form}
\hat{E}_3''(s) + c_1(s) \hat{E}_3'(s) + c_0(s)\hat{E}_3(s) = 0
\end{align}
Substituting (\ref{eqns:epsilon_atan_cutoff}) we now get the coefficients
\begin{subequations}
\begin{align}
c_1(s) &= \dfrac{r}{\left(s^2+1\right) \left((r+2) \tan ^{-1}(a)+r \tan ^{-1}(s)\right)} \label{eqns:1-D-time-harmonic-E3_s_Atan_coef_1_r}\\
c_0(s) = &\dfrac{1}{2} \Big(-\dfrac{2 r \left(2 (r+2) s \tan ^{-1}(a)+2 r s \tan
   ^{-1}(s)+r\right)}{\left(s^2+1\right)^2 \left((r+2) \tan ^{-1}(a)+r \tan
   ^{-1}(s)\right)^2}+\notag\\
   &\dfrac{r \Omega ^2 \tan ^{-1}(s)}{\tan ^{-1}(a)}+(r+2) \Omega
   ^2\Big)
 \label{eqns:1-D-time-harmonic-E3_s_Atan_coef_0_r}
\end{align}
\end{subequations}

Let us consider the coefficients shapes and value ranges. In (\ref{eqns:1-D-time-harmonic-E3_s_Atan_coef_1_r}), solving for the $c_1(s)$ denominator vanishing we get
\begin{align}
\label{eqns:1-D-time-harmonic-c_1-Atan-denom-vanish}
r = -\dfrac{2 \tan ^{-1}(a)}{\tan ^{-1}(
   s)+\tan ^{-1}(a)}
\end{align}
but $a>0$ and $|s|\leq a$; therefore
\begin{align*}
\tan ^{-1}(a) > 0, \qquad \tan ^{-1}(s)+\tan ^{-1}(a) \geq 0
\end{align*}
and the right hand side of (\ref{eqns:1-D-time-harmonic-c_1-Atan-denom-vanish}) is negative. But since $r>0$, we get that equation (\ref{eqns:1-D-time-harmonic-c_1-Atan-denom-vanish}) has no solution. Hence $c_1(s)$ and continuous and positive for $-a \leq s \leq a$. Similar reasoning applied to equation (\ref{eqns:1-D-time-harmonic-E3_s_Atan_coef_0_r}) shows that $c_0(s)$ is continuous.\\

\subsubsection{Coefficients profiles - asymptotics in $r$  } 
When $r \rightarrow 0+$, $c_1(s) \rightarrow 0+$ and $c_0(s)\rightarrow \Omega^2$ uniformly. That is intuitive because $r$ reflects the dielectric permittivity relative jump size; when the jump is zero, we get that equation (\ref{eqns:1-D-time-harmonic-z-normalized-boundary-source}) has constant coefficients. The first and second derivative terms $\epsilon'(z)$, $\epsilon''(z)$ in (\ref{eqns:1-D-time-harmonic-E3_z}) vanish.\\
The situation is somewhat more complex when $r \rightarrow \infty$, as we get
\begin{align}
\label{eqns:1-D-time-harmonic-E3_s_Atan_coef_1_r_r_inf}
c_1(s) = \dfrac{1}{\left(s^2+1\right) \left(\tan ^{-1}(a)+\tan ^{-1}(s)\right)}
\end{align}
In this case, the denominator vanishes at a single point $s = -a$, so that $c_1(s)$ approaches positive infinity there. We thus have a strong damping near $s=-1$. At $s=0$ we get

Differentiating (\ref{eqns:1-D-time-harmonic-E3_s_Atan_coef_1_r_r_inf}) in $s$:
\begin{align*}
c'_1(s)=-\dfrac{2 s \tan ^{-1}(a)+2 s \tan ^{-1}(s)+1}{\left(s^2+1\right)^2
   \left(\tan ^{-1}(a)+\tan ^{-1}(s)\right)^2}
\end{align*}
Since $-a \leq s \leq a$, the is always a range of $s$ values in which we can replace the numerator's $\tan^{-1}(s)$ term by $s$ and obtain
\begin{align*}
c_1'(s) \approx -\dfrac{2 s \tan ^{-1}(a)+2 s^2+1}{\left(s^2+1\right)^2 \left(\tan
   ^{-1}(a)+\tan^{-1}(s)\right)^2}
\end{align*}
The denominator is positive, and the numerator vanishes at
\begin{align*}
s_{1,2} = \dfrac{1}{2} \left(-\tan ^{-1}(a) \pm \sqrt{\tan ^{-1}(a)^2-2}\right)
\end{align*}
We see that the derivative change of sign occurs only if $\tan^{-1}(a)>\sqrt{2}$. The points $s_{1,2}$ are not symmetric with respect to origin, but they both approach zero as $a$ increases. \\
Concluding the case of $r \rightarrow \infty$ case, $c_1(s)$ is large near $s=-1$ when $r$ is large. It has also a $\delta$-function-like behavior when $r$ and $a$ are large.\\




\subsubsection{Coefficients profiles - asymptotics in $a$  }

\subsubsection{Coefficients profiles - asymptotics in $\Omega$  }

\subsubsection{Asymptotic profiles of $c_1(s)$}
To understand the asymptotic effect of $r$ variation on $c_1(s)$, we begin by examining the limits $r \rightarrow 0+$, $r \rightarrow \infty$, $a \rightarrow 0+$, $a \rightarrow \infty$.\\

When $a \rightarrow 0+$, we have 
\begin{align}
\label{eqns:1-D-time-harmonic-E3_s_Atan_coef_1_non_dim_a_0}
c_1(s) = \dfrac{r}{r s+r+2}
\end{align}
The numerator vanishes at $s = -1-\dfrac{2}{r}<-1$. Hence $c_1(s)\vert_{a=0}$ is well defined for all $-1 \leq s \leq 1$. Yet, when $r$ is large, the discontinuity point approaches $s=-1$, and we get large values of $c_1(s)$ near $s = -1$. \\
Finally, when $a \rightarrow \infty$, $c_1(s) \rightarrow 0$ uniformly. \begin{figure}
\begin{center}
\includegraphics[width=1.0\textwidth]{plotsCoef1Atan}
\end{center}
\caption {$c_1(s)$ profiles}.\\
Top left: $a=0.01$, $r=0.01$; top right: $a=0.01$, $r=50$; bottom left: $a=10$, $r=50$; bottom right: $a=70$, $r=50$.


\label{fig:c1Atan}
\end{figure}

\begin{figure}
\begin{center}
\includegraphics[width=1.0\textwidth]{plotsCoef1AtanRinf}
\end{center}
\caption {$c_1(s)$ asymptotic profiles - $r \rightarrow \infty$}.\\
Left: $a=1$, right: $a=20$.


\label{fig:c1AtanRinf}
\end{figure}



\subsubsection{Asymptotic profiles for $c_0(s)$} 

We calculate zero and infinity limits for each parameter in equation (\ref{eqns:1-D-time-harmonic-E3_s_Atan_coef_0_r}) assuming that the rest of the parameters are $O(1)$. At this stage, we shall not address the more complex problem of relating the sizes of the parameters pairwise (or triple-wise).\\
Let us begin examining each case.

\begin{enumerate}



\item Asymptotics in $a$:
	\begin{enumerate}
	\item $a$ large: 
	\begin{align}
	\label{eqns:1-D-time-harmonic-E3_s_Atan_coef_0_non_dim_a_inf}
	c_0(s) \approx \dfrac{1}{2} \Omega ^2 (r (1+\text{sgn}(s))+2) = \begin{array}{cc}
	\Big \{ & 
	\begin{array}{cc}
	 \Omega ^2 & s<0 \\
 	(r+1) \Omega ^2 & s \geq 0 \\
	\end{array}
 	\\
	\end{array}
	\end{align}
	
\begin{figure}
\begin{center}
\includegraphics[width=0.8\textwidth]{plotsC0AtanAAsymptotics}
\end{center}
\caption {$c_0(s)$ asymptotic profiles - limits in $a$}.\\
Top: $a \rightarrow \infty$; bottom: $a \rightarrow 0$.\\
Bottom left: $r=0.01$, $\Omega = 1$; bottom right: $r=50$, $\Omega = 1$.
\label{fig:c0AtanAlim}
\end{figure}


An interpretation for this result may be that when the layer's effective width is large, jump size tends to dominate the smoothness of the cutoff function. The jump size is $r \Omega^2$.
	\item $a$ small:
	\begin{align}
	c_0(s) \approx \dfrac{1}{2} \Omega ^2 (r s+r+2)-\dfrac{r^2}{(r s+r
	+2)^2}
	\end{align}
	\end{enumerate}
	calculating the $s$-derivative,
	\begin{align*}
	c_0'(s) = \dfrac{2 r^3}{(r s+r+2)^3}+\dfrac{r \Omega ^2}{2} 
	\end{align*}
	we have that $c_0'(s)>0$ for all $-1 \leq s \leq 1$, hence monotone 		there. When $r$ is large, $c_0(s)$'s discontinuity point approaches 		$s=-1$ from the left.
\item Asymptotics in $r$:
	\begin{enumerate}
	\item $r$ large:\\
	The first and second terms in equation (\ref{eqns:1-D-time-harmonic-E3_s_Atan_coef_0_non_dim}) are bounded when $r \rightarrow \infty$. The third term, on the other hand, grows like $r$. Calculating the limit of $\dfrac{c_0(s)}{r}$, $r \rightarrow \infty$, and multiplying again by $r$ we get the estimate
	\begin{align}
	c_0(s) \approx \dfrac{r\Omega ^2 \left(\tan ^{-1}(a s)+\tan ^{-1}(a)\right)}{2 \tan ^{-1}(a)}
	\end{align}
	When $a$ is large, $c_0(s)$ tends to the step function (\ref{eqns:1-D-time-harmonic-E3_s_Atan_coef_0_non_dim_a_inf}). When $a$ is small, $\tan^{-1}(a) \approx a$, $\tan^{-1}(as) \approx as$, and $c_0(s)$ tends to the linear function 
	\begin{align}
	c_0(s) \approx \dfrac{1}{2} r (s+1) \Omega ^2
	\end{align}
	
\begin{figure} 
\begin{center}
\includegraphics[width=0.8\textwidth]{plotsC0AtanRlargeAsymptotics}
\end{center}
\caption {$\dfrac{c_0(s)}{r}$ asymptotic profiles - $r \rightarrow \infty$}.\\
Left: $a = 0.1$, $\Omega = 50$; Right: $a = 50$, $\Omega=50$.
\label{fig:c0AtanRlim}
\end{figure}

	\item $r$ small:\\
	
\item Asymptotics in $\Omega$
	\begin{enumerate}
	\item $\Omega$ large:\\
	In that case, the first and second terms in (\ref{eqns:1-D-time-harmonic-E3_s_Atan_coef_0_non_dim}) can be neglected. we get
	\begin{align}
	c_0(s) \approx \Omega ^2 \left(\dfrac{r \left(\tan ^{-1}(a s)+\tan ^{-1}(a)\right)}{2 \tan ^{-1}(a)}+1\right)
	\end{align}
	When $a$ is small, we can replace $\tan^{-1}(a)$ by $a$ and $\tan^{-1}(as)$ by $as$. In that case, $c_0(s)$ tends to the linear function
	\begin{align}
	c_0(s) \approx \Omega ^2\left( \dfrac{1}{2} r (s+1) + 1 \right)
	\end{align}
	interpolating $\Omega^2$ at $s=-1$ and  and $\Omega^2(r+1)$ at $s=1$.
	
	\item $\Omega$ small:\\
	Here, the third term in (\ref{eqns:1-D-time-harmonic-E3_s_Atan_coef_0_non_dim}) is dropped. 
	\begin{align}
	\label{eqns:1-D-time-harmonic-E3_s_Atan_coef_0_non_dim_Omega_zero}
	c_0(s) \approx &-\dfrac{a^2 r^2}{\left(a^2 s^2+1\right)^2 \left(r \tan ^{-1}(a s)+(r+2)
   \tan ^{-1}(a)\right)^2} \notag \\
   &-\dfrac{2 a^3 r s}{\left(a^2 s^2+1\right)^2
   \left(r \tan ^{-1}(a s)+(r+2) \tan ^{-1}(a)\right)}
	\end{align}
	
\begin{figure} 
\begin{center}
\includegraphics[width=0.8\textwidth]{plotsC0AtanOmegaAsymptotics}
\end{center}
\caption {$c_0(s)$ asymptotic profiles - $\Omega \rightarrow 0$, $a = 10$, $r = 1$}.

\label{fig:c0AtanOmegalim}
\end{figure}


\end{enumerate}

	When $r$ is large and $a$ is kept $O(1)$, we get a monotone profile with large and negative values near $s=-1$, as demonstrated in the case 1(b). \\
	When $r$ is kept $O(1)$ and $a$ is large, we get a different phenomenon. The dominant term in (\ref{eqns:1-D-time-harmonic-E3_s_Atan_coef_0_non_dim_Omega_zero}) is the second one, which can be thought of as a shifted and scaled version of the function $-\dfrac{s}{(s^2 +1)^2}$. This is an odd function, fast decaying in $s$ and has to peaks, symmetric with respect to the origin - maximum and minimum. For small $s$ values, the denominator's $O(a^4)$ growth is suppressed, and the numerator's $O(a^3)$ a $\delta'(s)$-like behavior. 
	\end{enumerate}

\end{enumerate}

  
\newpage

\subsubsection{Simulations for $\arctan(z)$ cutoff}

Each of the following sequence of figures consists of 4 parts: plots of $c_0(s)$ (\ref{eqns:1-D-time-harmonic-E3_s_Atan_coef_1_r}) and $c_1(s)$ (\ref{eqns:1-D-time-harmonic-E3_s_Atan_coef_0_r}) at the top, a figure of $c_0(s)$ and $c_1(s)$ combined and equation (\ref{eqns:general_ODE_form}) with these coefficients solution at the bottom. In all cases, the left boundary condition () has $A=1$ and $\omega_i=\dfrac{\pi}{4}$.  The right boundary condition (\ref{eqns:1_D-time-harmonic-traveling-wave-bc-normalized}) is written, in terms of the non-dimensional parameter $\Omega$, 
\begin{align}
\label{eqns:1_D-time-harmonic-traveling-wave-bc-non-dim}
\left[i \Omega \sqrt{1+r}\hat{E}_3 + \dfrac{\partial \hat{E}_3}{\partial s} \right]_{s=1}=0
\end{align}

\begin{figure} 
\begin{center}
\includegraphics[width=0.8\textwidth]{plotsCoefsSols1}
\end{center}
\caption {Coefficients and solution - $a = 0.1$, $r = 0.1$, $\Omega = 0.1$}.

\label{fig:atanCoefsPlots1}
\end{figure}

\begin{figure} 
\begin{center}
\includegraphics[width=0.8\textwidth]{plotsCoefsSols2}
\end{center}
\caption {Coefficients and solution - $a = 0.1$, $r = 10$, $\Omega = 0.1$}.

\label{fig:atanCoefsPlots2}
\end{figure}

\begin{figure} 
\begin{center}
\includegraphics[width=0.8\textwidth]{plotsCoefsSols3}
\end{center}
\caption {Coefficients and solution - $a = 0.1$, $r = 0.1$, $\Omega = 10$}.

\label{fig:atanCoefsPlots3}
\end{figure}

\begin{figure} 
\begin{center}
\includegraphics[width=0.8\textwidth]{plotsCoefsSols4}
\end{center}
\caption {Coefficients and solution - $a = 0.1$, $r = 10$, $\Omega = 10$}.

\label{fig:atanCoefsPlots4}
\end{figure}

\begin{figure} 
\begin{center}
\includegraphics[width=0.8\textwidth]{plotsCoefsSols5}
\end{center}
\caption {Coefficients and solution - $a = 10$, $r = 0.1$, $\Omega = 0.1$}.

\label{fig:atanCoefsPlots5}
\end{figure}

\begin{figure} 
\begin{center}
\includegraphics[width=0.8\textwidth]{plotsCoefsSols6}
\end{center}
\caption {Coefficients and solution - $a = 10$, $r = 10$, $\Omega = 0.1$}.

\label{fig:atanCoefsPlots6}
\end{figure}

\begin{figure} 
\begin{center}
\includegraphics[width=0.8\textwidth]{plotsCoefsSols7}
\end{center}
\caption {Coefficients and solution - $a = 10$, $r = 0.1$, $\Omega = 10$}.

\label{fig:atanCoefsPlots7}
\end{figure}

\begin{figure} 
\begin{center}
\includegraphics[width=0.8\textwidth]{plotsCoefsSols8}
\end{center}
\caption {Coefficients and solution - $a = 10$, $r = 10$, $\Omega = 10$}.

\label{fig:atanCoefsPlots8}
\end{figure}



\newpage
%\subsubsection{Simulations for $\tanh(z)$ cutoff}
\begin{comment}
\subsubsection{Passage to a Sturm-form BVP}
Equation (\ref{eqns:1-D-time-harmonic-E3_z}) is a linear, homogeneous real valued variable coefficients second order ODE. We can estimate its solution's period of oscillation using Sturm's comparison theorem \cite{BirkhoffRota1989}:

\begin{thm}
\label{thm:Sturm-Comparison}
Let $f(x)$ and $g(x)$ be nontrivial solutions of the DEs $u''+p(x)u=0$ and $v''+q(x)v=0$, respectively., where $p(x) \geq q(x)$. Then $f(x)$ vanishes at least once between any two zeros of $g(x)$, unless $p(x) \equiv q(x)$ and if $f$ is a constant multiple of $g$. 
\end{thm}
As we can see, the theorem requires the equation's first order derivative coefficient to vanish. Thus equation (\ref{eqns:1-D-time-harmonic-E3_z}) must be transformed to such form. \\
Letting $\hat{E}_3 = \epsilon^{-1/2}u(z)$ we get
\begin{align}
\label{eqns:1-D-harmonic-z-Sturm}
\dfrac{d^2 u}{dz^2} +\dfrac{\omega^2 \epsilon^3(z)-\frac{3}{4}(\epsilon'(z))^2+\frac{1}{2}\epsilon(z)\epsilon''(z)}{\epsilon(z)}u=0
\end{align}
Since $\epsilon(z)$, $\epsilon'(z)$, $\epsilon''(z)$ are bounded and $\epsilon(z)$ is strictly positive, we can take $\omega^2$ large enough so that the right hand side of (\ref{eqns:1-D-harmonic-z-Sturm}) would be larger than $\dfrac{\omega^2}{2\epsilon_2}$ for all $-\delta<z<\delta$.In such case, the solutions for (\ref{eqns:1-D-harmonic-z-Sturm}) oscillate at a frequency of at least $\dfrac{\omega}{2\pi\sqrt{2 \epsilon_2}}$.




\subsection{The one-dimensional case in source-free, non-magnetic media}
$\mathbf{Banergee - J_c and \rho as source}$, Griffiths - Gauss's law

Neglecting magnetic phenomena we have $\mu = 1$. We also assume no free currents - $\underline{J} = 0$, no charge accumulation - $\rho = 0$, and that $\epsilon$ is $x$ - dependent - $\epsilon = \epsilon(x)$. We get the $\underline{E} , \underline{H}$ representation 

\begin{subequations}
	\begin{align}
        \nabla \times \underline{E} &= -\dfrac{\partial \underline{H}}{\partial t}  \label{eqns:ourCurlE_Maxwell}\\
		\nabla \times \underline{H} &=  \epsilon(x)\dfrac{\partial  \underline{E}}{\partial t} \label{eqns:ourCurlH_Maxwell}	\\	
		\nabla \cdot ( \epsilon(x) \underline{E} ) &= 0	\\		
		\nabla \cdot \underline{H} &= 0	
	\end{align}
	\label{eqns:ourCurlMaxell}
\end{subequations}

Applying the vector calculus identity (\ref{eqns:CurlCurlID}) and the $\underline{H}$ - curl equation (\ref{eqns:ourCurlH_Maxwell}) we get
\begin{align*}
\nabla (\nabla \cdot \underline{E}) - \nabla^2\underline{E} + \epsilon(x) \dfrac{\partial^2 \underline{E}}{\partial t^2} = 0
\end{align*}

Next, we use the assumption $\rho = 0$ in equations (\ref{eqns:DivD_Maxwell}), (\ref{eqns:constitutiveDepsE}) combined. We get
\begin{align*}
\nabla \cdot (\epsilon(x) \underline{E}) = 0
\end{align*}
Expanding using the vector calculus formula (\ref{eqns:DivProdID}) we get
\begin{align*}
\epsilon(x) \nabla \cdot \underline{E} + \underline{E} \epsilon'(x) = 0
\end{align*}
Thus 
\begin{align*}
\nabla \cdot \underline{E} = -\underline{E} \cdot \dfrac{\epsilon'(x)}{\epsilon(x)}
\end{align*}




\section{Maxwell's equations in 1-D}
As a first model problem we consider an electromagnetic plane wave hitting an infinite, layered slab of infinite thickness at a right angle. \\
The curl equations  (\ref{eqns:ourCurlE_Maxwell}), (\ref{eqns:ourCurlH_Maxwell}) can be put in a vector notation:

\begin{subequations}
\begin{eqnarray}
	\dfrac{\partial}{\partial t} \mat{E_x \\E_y \\E_z} = 
	\dfrac{1}{\epsilon(x)} \mat{	\partial_y H_z - \partial_z H_y	\\
								\partial_z H_x - \partial_x H_z	\\
								\partial_x H_y - \partial_y H_x	
							}	\\		
	\dfrac{\partial}{\partial t}
						\mat{H_x \\H_y \\H_z} = 
					  -	\mat{	\partial_y E_z - \partial_z E_y	\\
								\partial_z E_x - \partial_x E_z	\\
								\partial_x E_y - \partial_y E_x	
							}	
\end{eqnarray}
\end{subequations}
We assume that the wave front propagates in the $x$ direction. Hence $y$, $z$ derivatives of $\underline{E} , \underline{H}$ vanish. We get 

\begin{subequations}
\begin{eqnarray}
	\dfrac{\partial}{\partial t} \mat{E_x \\E_y \\E_z} = 
	\dfrac{1}{\epsilon(x)} \mat{	0	\\
								 - \partial_x H_z	\\
								\partial_x H_y 	
							}	\\		
	\dfrac{\partial}{\partial t}
						\mat{H_x \\H_y \\H_z} = 
					  	\mat{	0	\\
								  \partial_x E_z	\\
								-\partial_x E_y 
							}	
\end{eqnarray}
\end{subequations}
which can be reduced into two first order PDE systems

\begin{subequations}
\begin{eqnarray}
	\dfrac{\partial}{\partial t} \mat{E_y \\H_z} &= \mat{0 & -1/\epsilon(x)	\\ -1 & 0} \dfrac{\partial}{\partial x} \mat{E_y \\ H_z}		
		\label{eqns:ourMaxwellEyHzPolarized}\\
	\dfrac{\partial}{\partial t} \mat{E_z \\H_y} &= \mat{0 & 1/\epsilon(x)	\\ 1 & 0} \dfrac{\partial}{\partial x} \mat{E_z \\ H_y}
		\label{eqns:ourMaxwellEzHyPolarized}
\end{eqnarray}
\end{subequations}
Systems (\ref{eqns:ourMaxwellEyHzPolarized}), (\ref{eqns:ourMaxwellEzHyPolarized}) are hyperbolic, as the systems matrices have distinct real eigenvalues $\pm \dfrac{1}{\sqrt{\epsilon(x)}}$. Using the method of characteristics one can shown \cite{Gustaffson1995} that since each eigenvalue has the same sign all through the domain, a solution for such a systems is a superposition of a right traveling wave and a left traveling wave. \\

For an analytic solution we therefore have to supply a boundary condition at the left boundary for the right traveling wave and a zero boundary condition at $+\infty$ for the left traveling wave, since we assume that no information is carried from that direction. \\



Since we are interested in visible light optics, the analytic problem left boundary condition should be of the form $A e^{i \left(\omega t + \phi\right)}$, where $A>0$ is the source amplitude, $\omega >0$ is its angular frequency and $-\pi/2 < \phi \leq \pi/2$ is the phase. The complex term is a convenient way to facilitate the use of phasors, and a real part of the solution should be taken after the computation.\\

Numerically speaking, we should make a finite computational domain simulate the problem's infinite spatial domain. In that sense we should typically force artificial boundary conditions, simulating the wave front's continued propagation in the same media, or, in the case of a reflected wave, backward exit to vacuum. Such boundary conditions are called `absorbing boundary conditions'. We'll use the shorthand `ABC'.



\section{Inverse problem for the 1-D one-way wave equation}

Hyperbolic PDE systems of the form (\ref{eqns:ourMaxwellEyHzPolarized}) and (\ref{eqns:ourMaxwellEzHyPolarized}) can be transformed into coupled 1-D one way wave equations
\begin{align*}
u^j_t+a^j(x)u^j_x=0, \qquad j=1,2
\end{align*}
where coupling occurs at the boundary. It therefore makes sense use parameter identification problems for END HERE 
\begin{align*}
\begin{cases}
u_t + a(x) u_x &= 0,	\qquad -L<x< \infty \\
u(x,0) &= 0	\\
u(-L,t) &= g(t)
\end{cases}
\end{align*}
Most of the time we shall be interested in a boundary condition of the form 
\begin{align}
g(t) = A e^{i \left(\omega t + \phi \right)}
\label{eqns:complexExpBC}
\end{align}
When $a(x)=a$ is constant, we have a traveling wave solution 
\begin{align*}
u(x,t) = u(x-at)
\end{align*}
as can be shown by direct differentiation. Applying the initial condition and the left boundary condition we get

\begin{align*}
u(x,t) = g\left( t-\dfrac{x+L}{a} \right)
\end{align*}
and for the boundary condition (\ref{eqns:complexExpBC}), we get
\begin{align}
u(x,t) =A e^{i \left(\omega  \left(t-\frac{x+L}{a}\right)+\phi \right)}
\end{align}

Now assume that
\end{comment} 

\section{Parameter estimation for 1-D boundary value problems}
The inverse problem we wish to solve is the estimation of the parameters defining dielectric profile $\epsilon\left(\dfrac{z}{M}\right)$ from a solution of equation (\ref{eqns:1-D-time-harmonic-z-normalized-boundary-source}) subject to the boundary conditions (\ref{eqns:1-D-time-harmonic-z-boundary-source-normalized}), (\ref{eqns:1_D-time-harmonic-traveling-wave-bc-normalized}). This is a complex project since it involves solving systems of nonlinear equations, where the target parameters may have different scales.\\
It is thus instructive to first analyze the recovery of parameters from  a reasonably simple 'toy model'. We use this procedure to demonstrate the difficulties we may entangle, resulting in general properties of 1-D boundary value problems.


\subsection{The first model: the harmonic oscillator with Dirichlet boundary conditions}
We consider the boundary value problem
\begin{align}
\label{eqns:}
\begin{cases}
&y'' + \omega^2 y =0 \\
y(-a) &= A, \qquad y(a) = 0
\end{cases}
\end{align}
Solving, we get 
\begin{align}
y(x) = \dfrac{1}{2} (A \sec (\omega a ) \cos (\omega x)-A \csc ( \omega a ) \sin ( \omega x))
\end{align}
This solution is valid for $\omega a \neq \pi k, \dfrac{\pi}{2}+ \pi k$, which are, as can be easily verified, the problem's eigenvalues.
 


\subsubsection{Characterization of the forward problem }
\begin{figure} 
\begin{center}
\includegraphics[width=0.8\textwidth]{plotsOscEigFuncsNearEigVal}
\end{center}
\caption {BVP solutions near an eigenvalue - $a = 1$, $A=1$. }

\label{fig:plotsOscEigFuncsNearEigVal}
\end{figure}

\subsubsection{The parameter estimation method}

\subsubsection{Increasing the number of sample points}

\subsubsection{Increasing the BVP interval length}

\subsubsection{Varying the boundary condition amplitude}

\subsubsection{Estimation of $\omega$ near an eigenvalue}

\subsubsection{Sensitivity to noise}



\newpage
\begin{thebibliography}{2d}
\bibitem{ArfkenWeber2005}
Arfken GB. Weber HJ. Mathematical methods for physicists, 6th ed. Elsevier academc press, Amsterdam, 2005
\bibitem{BirkhoffRota1989}
Birkhoff G. Rota GC. Ordinary differential equations, 4th ed. Wiley, New York, 1989.
\bibitem{Gustaffson1995}
Gustaffson B. Kreiss HO. Oliger J. Time dependent problems and difference methods. Wiley, New York, 1995.
\bibitem{Hecht2002}
Hecht E. Optics, 4th ed. Addison Wesley, San Francisco, CA, 2002.
\bibitem{Strikwerda2004}
Strikwerda J. Finite difference schemes and partial differential equation, 2nd ed. SIAM,  Wadsworth and Brooks / Cole, Pacific Grove, CA, 2004
\end{thebibliography}

\end{document}

 








\documentclass[12pt,twoside]{article} 
\usepackage[reqno,tbtags]{amsmath}
\usepackage[active]{srcltx} % not for published version
\usepackage{amsthm}
\usepackage{amsfonts}
\usepackage{amssymb}
\usepackage{graphicx}
\usepackage{subfigure}
\usepackage{verbatim}
% put any defs here
\newcommand{\mat}[2][cccccccccccccccccccccccccccccccccccccccccccc]{\left(\begin{array}{#1}#2 \\ \end{array} \right)}

\newtheorem{lem}{Lemma}[subsubsection]
\newtheorem{thm}{Theorem}[subsection]
\newtheorem{cor}{Corollary}[subsubsection]
\newtheorem{define}{Definition}[subsubsection]


\begin{document}

\title{Numerical schemes for the 1-D wave equation}

\newpage

\section{From Maxwell's equations to generalized Helmholtz equations system in heterogeneous media}
\subsection{Maxwell's equations in MKS units }

To begin, let us state Maxwell's equations in differential form.
\begin{subequations}

	\begin{align}
        \nabla \times \underline{H}-\dfrac{\partial \underline{D}}{\partial t} &= \underline{J}  \label{eqns:CurlH_Maxwell}	\\	
        \nabla \times \underline{E} + \dfrac{\partial \underline{B}}{\partial t}&= 0  \label{eqns:CurlE_Maxwell} \\
		\nabla \cdot \underline{D} &=  \rho	 \label{eqns:DivD_Maxwell}\\		
		\nabla \cdot \underline{B} &= 0	
	\end{align}
\end{subequations}

with the constitutive relations
\begin{subequations}
\begin{align}
        \underline{D} = \epsilon \underline{E} \label{eqns:constitutiveDepsE}\\
        \underline{B} = \mu \underline{H}	\label{eqns:constitutiveBH}\\
        \underline{J} = \sigma \underline{E}
\end{align}
\end{subequations}

Throughout our work we'll need to keep track of the equations elements units.
\begin{itemize}
\item The dielectric permittivity: $[\epsilon (x)] = \dfrac{Farad}{m} =
 \dfrac{C^2}{N m^2}$, \\
 $\epsilon_0 = 8.85*10^{-12}\dfrac{Farad}{m}$
\item The magnetic permeability: $[\mu] = \dfrac{H}{m} = \dfrac{N}{A^2}$, \\  $\mu_0 = 4\pi \cdot 10^{-7}\dfrac{V s}{A m}$
\item The refractive index: $n = \sqrt{\dfrac{\epsilon \mu}{\epsilon_0 \mu_0}}$, dimensionless.
\item The light speed 
	\begin{itemize}
	\item In vacuum: $c = \dfrac{1}{\sqrt{\epsilon_0 \mu_0}}$
	\item In matter: $v = \dfrac{1}{\sqrt{\epsilon \mu}} = \dfrac{c}{n}$
\end{itemize}
\item The wavelength:  $[\lambda] = m$, for visible light $\lambda \approx 0.5 \cdot 10^{-6} m$
\item The wave number in matter: $k = \dfrac{\omega}{v} = \dfrac{2\pi}{\lambda}$, where $\omega$ is the time-harmonic wave frequency, $[k] = \dfrac{\text{rad}}{\text{m}}$
\item The electric field: $[\underline{E}] = \dfrac{N}{C}$
\item The displacement vector: $[\underline{D}] = \dfrac{C}{m^2}$
\item The magnetic $\underline{B}$-field: $[\underline{B}] = \dfrac{V \cdot s}{m^2}$=Weber
\item The magnetic $\underline{H}$-field: $[\underline{H}] = \dfrac{A}{m}$

\end{itemize}

\subsection{Transforming Maxwell's equations into generalized wave equations in heterogeneous media}
\subsubsection{Applying vector analysis identities }
In order to transform Maxwell's equation into a system of wave equations, we need the following set of vector calculus identities:
\begin{subequations}
\label{eqns:vectorCalcIDs}
	\begin{align}
			\nabla \times (\psi \underline{a}) = \nabla \psi \times 	
			\underline{a} + \psi \nabla \times \underline{a}  \label{eqns:ProdCurlID}\\
		\nabla \times \nabla \times \underline{a} = \nabla ( \nabla
			 \cdot \underline{a} ) - \nabla^2\underline{a} \label{eqns:CurlCurlID}\\
		\nabla \cdot (\psi \underline{a}) = \psi \nabla \cdot \underline{a}
			 + \underline{a} \cdot \nabla \psi \label{eqns:DivProdID}
			\end{align}
\end{subequations}
In this section we'll assume only that the dielectric permittivity $\epsilon$ and the magnetic permeability $\mu$ are differentiable functions of the space dimensions $x,y,z$ and are time independent.

We start from equation (\ref{eqns:CurlE_Maxwell})
\begin{align*}
\nabla \times \underline{E} + \dfrac{\partial \underline{B}}{\partial t}&= 0
\end{align*}
Substituting the relation (\ref{eqns:constitutiveBH}) between magnetic fields in conjunction we get
\begin{align*}
\nabla \times \underline{E} + \mu \dfrac{\partial  \underline{H}}{\partial t}&= 0
\end{align*}
For future purposes, let us divide by $\mu$:
\begin{align*}
\dfrac{1}{\mu} \nabla \times \underline{E} + \dfrac{\partial \underline{H}}{\partial t} = 0
\end{align*}

Taking the curl of both sides and changing the order of temporal and spatial differentiation
\begin{align}
\label{eqns:CulCurlMaxwellModified}
\nabla \times \left( \dfrac{1}{\mu} \nabla \times \underline{E} \right)  + \dfrac{\partial (\nabla \times  \underline{H})}{\partial t} = 0
\end{align}
We'll apply thew vector calculus identities (\ref{eqns:vectorCalcIDs}) to decompose the summed terms in (\ref{eqns:CulCurlMaxwellModified}).\\
Using identity (\ref{eqns:ProdCurlID}) we get
\begin{align}
\label{eqns:CurlProdCurlpart}
\nabla \times \left( \dfrac{1}{\mu} \nabla \times \underline{E} \right) = \nabla \left( \dfrac{1}{\mu} \right) \times \left(\nabla \times \underline{E} \right) + \dfrac{1}{\mu} \nabla \times \nabla \times \underline{E}
\end{align}
Now we apply (\ref{eqns:CurlCurlID}) to (\ref{eqns:CurlProdCurlpart}) and get
\begin{align}
\label{eqns:CurlProdCurlExpanded}
\nabla \times \left( \dfrac{1}{\mu} \nabla \times \underline{E} \right) = \nabla \left( \dfrac{1}{\mu} \right) \times \left(\nabla \times \underline{E} \right) + \dfrac{1}{\mu} \left( \nabla (\nabla \cdot \underline{E}) - \nabla^2 \underline{E}\right)
\end{align}
The $\underline{H}$ - term in (\ref{eqns:CulCurlMaxwellModified}) can be replaced using equation (\ref{eqns:CurlH_Maxwell}):
\begin{align}
\label{eqns:Curl_H_replaced_D}
\dfrac{\partial}{\partial t} \left( \nabla \times \underline{H}\right) = \dfrac{\partial^2 \underline{D}}{\partial t^2} + \dfrac{\partial \underline{J}}{\partial t}
\end{align}

and using the constitutive relation (\ref{eqns:constitutiveDepsE}), equation (\ref{eqns:Curl_H_replaced_D}) becomes
\begin{align}
\label{eqns:Curl_H_replaced_E}
\dfrac{\partial}{\partial t} \left( \nabla \times \underline{H}\right) = \epsilon\dfrac{\partial^2 \underline{E}}{\partial t^2} + \dfrac{\partial \underline{J}}{\partial t}
\end{align}

Substituting (\ref{eqns:CurlProdCurlExpanded}), (\ref{eqns:Curl_H_replaced_E}) into (\ref{eqns:CulCurlMaxwellModified}) and multiplying by we get $\mu$ we get an equation in $\underline{E}$:
\begin{align}
\label{eqns:E_Wave_full}
\mu \nabla \left( \dfrac{1}{\mu} \right) \times \left(\nabla \times \underline{E} \right) +  \nabla (\nabla \cdot \underline{E}) - \nabla^2 \underline{E} + \mu \epsilon\dfrac{\partial^2 \underline{E}}{\partial t^2} + \mu \dfrac{\partial \underline{J}}{\partial t} = 0
\end{align}
which can be put in the form 
\begin{align}
\label{eqns:E_Wave_full_compact}
 \nabla \left( \log (\mu) \right) \times \left(\nabla \times \underline{E} \right) +  \nabla (\nabla \cdot \underline{E}) - \nabla^2 \underline{E} + \mu \epsilon \dfrac{\partial^2 \underline{E}}{\partial t^2} + \mu \dfrac{\partial \underline{J}}{\partial t} = 0
\end{align}
where $v$ is the speed of light in matter. We'll refer to the system (\ref{eqns:E_Wave_full}) as the generalized wave equations system.

\subsubsection{Dimensional analysis}
We'll analyze equation (\ref{eqns:E_Wave_full}) term by term.
\begin{align*}
\left[ \mu \nabla \left( \dfrac{1}{\mu} \right) \times \left(\nabla \times \underline{E} \right) \right] = \dfrac{[\mu]}{[\mu] m} \cdot \dfrac{N/C}{m} = \dfrac{N}{C m^2}
\end{align*}
\begin{align*}
\left[\nabla (\nabla \cdot \underline{E}) \right]= \dfrac{N/C}{m^2} = \dfrac{N}{C m^2}
\end{align*}

\begin{align*}
\left[ \nabla^2 \underline{E} \right] = \dfrac{N/C}{m^2} = \dfrac{N}{C m^2}
\end{align*}
Using the relation $\mu \epsilon = \dfrac{1}{v^2}$:
\begin{align*}
\left[ \mu \epsilon\dfrac{\partial^2 \underline{E}}{\partial t^2} \right] = \left[\dfrac{1}{v^2} \dfrac{\partial^2 \underline{E}}{\partial t^2} \right] = \dfrac{sec^2}{m^2}\dfrac{N/C}{sec^2} = \dfrac{N}{C m^2}
\end{align*}
\begin{align*}
[\mu J] = [\mu]\left[\epsilon_0 \dfrac{\partial E}{\partial t}\right] = \dfrac{N}{A^2}\dfrac{C}{N m^2} \cdot \dfrac{N}{C \cdot sec} = \dfrac{N sec^2}{C^2}\dfrac{C}{m^2 sec} = \dfrac{N}{C m^2}
\end{align*}


\subsection{Analysis of the source and material-dependent terms in the generalized wave equations system}

Compared to the standard wave equation $\dfrac{1}{v^2}\dfrac{\partial ^2\underline{E}}{\partial t^2} =  \nabla^2 \underline{E}$, equation (\ref{eqns:E_Wave_full}) has extra terms:
\begin{subequations}
\begin{align}
\nabla \left( \log (\mu) \right) \times \left(\nabla \times \underline{E} \right) \label{eqns:E_Wave_extra_term_log_mu}\\
\nabla (\nabla \cdot \underline{E}) \label{eqns:E_Wave_extra_term_Grad_Div_E} \\
\mu \dfrac{\partial \underline{J}}{\partial t} \label{eqns:E_Wave_extra_term_mu_DJDt} 
\end{align}
\end{subequations}
While the term (\ref{eqns:E_Wave_extra_term_mu_DJDt})is a straight- forward source term, the other two have a more complicated structure. In this section we'll describe the contributions of the terms (\ref{eqns:E_Wave_extra_term_log_mu}), (\ref{eqns:E_Wave_extra_term_Grad_Div_E}) to each equation in the generalized wave equations system.

\subsubsection{The term $\nabla \left( \log (\mu) \right) \times \left(\nabla \times \underline{E} \right)$}

In this work we are primarily interested in visible light optics. Materials which are transparent in visible light are essentially ``nonmagnetic'' (see \cite{Hecht2002}, chapter 3). Evidently, the term (\ref{eqns:E_Wave_extra_term_log_mu}) vanishes if $\mu$ is constant, and specifically for nonmagnetic substances satisfying $\mu \equiv 1$. Still, there exist magnetic substances that are transparent in the infrared and microwave regions of the spectrum. For the sake of completeness, we'll decompose the term to understand its contribution to each wave equation in  the system (\ref{eqns:E_Wave_full}).\\

To handle the repeated cross product we'll use Levi-Civita's permutation symbols $\epsilon_{ijk}$ \cite{ArfkenWeber2005} defined by
\begin{align}
\label{eqns:LeviCivitaSymbol}
\epsilon_{123} = \epsilon_{231} = \epsilon_{312} & = 1	\notag\\
\epsilon_{132} = \epsilon_{213} = \epsilon_{321} &= -1 \\
\text{all other } \epsilon_{ijk} &= 0	\notag
\end{align}
The following lemma is presented as an exercise in \cite{ArfkenWeber2005}, p. 150.
\begin{lem}
\label{lem:LiviCivitaEpsDelta}
Let $\epsilon_{ijk}$ be defined by (\ref{eqns:LeviCivitaSymbol}). Then
\begin{align}
\label{eqns:LiviCivitaEpsilonDelta} 
\epsilon_{ijk} \epsilon_{pqk} = \delta_{ip} \delta_{jq} - \delta_{iq} \delta_{jp}
\end{align} 
where $\delta_{ij}$ is Kronecker's delta.
\end{lem}

\begin{proof}
Noticed that each term in the $k$-sum $\epsilon_{ijk} \epsilon_{pqk}$ is non-vanishing only if $i,j,p,q \neq k$, $i \neq j$, $p \neq q$. For example, for the $k=1$ term we need only to examine the possibilities 
\begin{align*}
i=2,\quad j=3, \quad p=2, \quad q=3	\\
i=3,\quad j=2, \quad p=2, \quad q=3	\\
i=2,\quad j=3, \quad p=3, \quad q=2	\\
i=3,\quad j=2, \quad p=3, \quad q=2
\end{align*} 
Substituting and using (\ref{eqns:LeviCivitaSymbol}) we get that in all cases  $\epsilon_{ij1} \epsilon_{pq1} =  \delta_{ip} \delta_{jq} - \delta_{iq} \delta_{jp}$.\\
Similar results can be shown for $k=2,3$. Finally, the summation (\ref{eqns:LiviCivitaEpsilonDelta}) holds because $i,j,p,q$ substitutions that yield a non-vanishing for a specific value of $k$ cause the other $k$-terms to vanish.
\end{proof}

\begin{lem}
Let $\underline{a}$ be a vector of $x,y,z$-differentiable entries. Then 
\begin{align}
\label{eqns:doubleCrossProdDecomposition}
\left( \underline{a} \times (\nabla \times \underline{E}) \right)_i = \underline{a} \cdot \left( \dfrac{\partial \underline{E}}{\partial x_i}- \nabla E_i \right)
\end{align}
\end{lem}

\begin{proof}
Using the Levi-Civita symbol (\ref{eqns:LeviCivitaSymbol}), 
\begin{align*}
\left( \underline{a} \times (\nabla \times \underline{E}) \right)_i  = \epsilon_{ijk}a_j\left( \epsilon_{kpq}\dfrac{\partial E_q}{\partial x_p} \right) = \epsilon_{ijk} \epsilon_{kpq} a_j \dfrac{\partial E_q}{\partial x_p}
\end{align*}
By lemma (\ref{lem:LiviCivitaEpsDelta})
\begin{align*}
\left( \underline{a} \times (\nabla \times \underline{E}) \right)_i  =  \left( \delta_{ip}\delta{jq} - \delta_{iq}\delta_{jp}\right) a_j \dfrac{\partial E_q}{\partial x_p}
\end{align*}
but
\begin{align*}
\delta_{ip}\delta_{jq}a_j \dfrac{\partial E_q}{\partial x_p} &= a_j \dfrac{\partial E_j}{\partial x_i} \\
\delta_{iq}\delta_{jp}a_j \dfrac{\partial E_q}{\partial x_p} &= a_j \dfrac{\partial E_i}{\partial x_j}
\end{align*}
Hence
\begin{align*}
\left( \underline{a} \times (\nabla \times \underline{E}) \right)_i = a_j\left( \dfrac{\partial E_j}{\partial x_i} - \dfrac{\partial E_i}{\partial x_j} \right)
\end{align*}
Explicit summation yields
\begin{align*}
\left( \underline{a} \times (\nabla \times \underline{E}) \right)_i &= 
a_1\left( \dfrac{\partial E_1}{\partial x_i} - \dfrac{\partial E_i}{\partial x_1} \right) +
a_2\left( \dfrac{\partial E_2}{\partial x_i} - \dfrac{\partial E_i}{\partial x_2} \right) +
a_3\left( \dfrac{\partial E_3}{\partial x_i} - \dfrac{\partial E_i}{\partial x_3} \right) \\
&= \underline{a} \cdot \left( \dfrac{\partial \underline{E}}{\partial x_i}- \nabla E_i \right)
\end{align*}
as required.
\end{proof}

\begin{cor}
The term (\ref{eqns:E_Wave_extra_term_log_mu}) $i$'th component has the form 
\begin{align}
\label{eqns:E_Wave_extra_term_log_mu_expanded}
\left( \nabla \left( \log \mu \right) \times \left(\nabla \times \underline{E} \right) \right)_i = 
\nabla (\log \mu) \cdot \left( \dfrac{\partial \underline{E}}{\partial x_i}- \nabla E_i \right)
\end{align}
\end{cor}





\subsubsection{The term $\nabla (\nabla \cdot \underline{E})$}
To decompose the term (\ref{eqns:E_Wave_extra_term_Grad_Div_E}), let us substitute the constitutive relation (\ref{eqns:constitutiveDepsE}) into equation (\ref{eqns:DivD_Maxwell}). We get
\begin{align*}
\nabla \cdot (\epsilon \underline{E}) = \rho
\end{align*}
Using the product rule for differentiation (\ref{eqns:DivProdID}), 
\begin{align*}
\epsilon\nabla \cdot \underline{E} + \underline{E} \cdot \nabla \epsilon = \rho
\end{align*}
Hence
\begin{align}
\label{eqns:DivE}
\nabla \cdot \underline{E} = \dfrac{\rho}{\epsilon} - \underline{E} \cdot \nabla (\log \epsilon)
\end{align}

Taking the gradient of both sides of (\ref{eqns:DivE}) we get
\begin{align*}
\nabla \left( \nabla \cdot \underline{E} \right)_i &=\nabla\left( \dfrac{\rho}{\epsilon}\right)_i-\dfrac{\partial}{\partial x_i}\left( E_j \dfrac{\partial \log(\epsilon)}{\partial x_j}\right) \\&= \nabla\left( \dfrac{\rho}{\epsilon}\right)_i - \left( \dfrac{\partial E_j}{\partial x_i}\dfrac{\partial \log(\epsilon)}{\partial x_j} + E_j \dfrac{\partial^2 \log(\epsilon)}{\partial x_i \partial x_j}\right)
\end{align*}
Hence
\begin{align}
\label{eqns:E_Wave_extra_term_Grad_Div_E_expanded}
\nabla \left( \nabla \cdot \underline{E} \right)_i = \nabla\left( \dfrac{\rho}{\epsilon}\right)_i - \left(\dfrac{\partial \underline{E}}{\partial x_i} \cdot \nabla \log(\epsilon) + \underline{E} \cdot \dfrac{\partial}{\partial x_i} \nabla \log(\epsilon) \right)
\end{align}

\subsubsection{A 1-D layered media - system coefficients depending on a single coordinate}

Let us now assume that the $\epsilon = \epsilon(z)$, $\mu=\mu(z)$, $\rho = \rho(z)$, and see what can be learned.\\
Using (\ref{eqns:E_Wave_extra_term_log_mu_expanded}),The term (\ref{eqns:E_Wave_extra_term_log_mu}) has the following form, component-wise:
\begin{subequations}
\begin{align}
\left(\nabla \left( \log (\mu) \right) \times \left(\nabla \times \underline{E} \right)\right)_1 &= \dfrac{\mu'(z)}{\mu(z)}\left( -\dfrac{\partial E_1}{\partial z} + \dfrac{\partial E_3}{\partial x}\right) \label{eqns:LogTerm_z_only_Eq1}\\
\left(\nabla \left( \log (\mu) \right) \times \left(\nabla \times \underline{E} \right)\right)_2 &= \dfrac{\mu'(z)}{\mu(z)}\left( -\dfrac{\partial E_2}{\partial z} + \dfrac{\partial E_3}{\partial y}\right) \label{eqns:LogTerm_z_only_Eq2} \\
\left(\nabla \left( \log (\mu) \right) \times \left(\nabla \times \underline{E} \right)\right)_3 &= 0 \label{eqns:LogTerm_z_only_Eq3}
\end{align}
\end{subequations}
Similarly, we use (\ref{eqns:E_Wave_extra_term_Grad_Div_E_expanded}) to bring the term (\ref{eqns:E_Wave_extra_term_Grad_Div_E}) components to
\begin{subequations}
\begin{align}
(\nabla (\nabla \cdot \underline{E}))_1 &= -\dfrac{\epsilon'(z)}{\epsilon(z)} \dfrac{\partial E_3}{\partial x}	\label{eqns:GradDivTerm_z_only_Eq1}\\
(\nabla (\nabla \cdot \underline{E}))_2 &= -\dfrac{\epsilon'(z)}{\epsilon(z)} \dfrac{\partial E_3}{\partial y} \label{eqns:GradDivTerm_z_only_Eq2}\\
(\nabla (\nabla \cdot \underline{E}))_3 &= -\dfrac{\rho(z)}{\epsilon(z)}\epsilon'(z) + \dfrac{\rho'(z)}{\epsilon(z)}-\left( \dfrac{\epsilon''(z)}{\epsilon(z)}-\left( \dfrac{\epsilon'(z)}{\epsilon(z)}\right)^2 \right)E_3 \label{eqns:GradDivTerm_z_only_Eq3}\\
& -\dfrac{\epsilon'(z)}{\epsilon(z)}\dfrac{\partial E_3}{\partial z}	\notag
\end{align}
\end{subequations}

Plugging (\ref{eqns:LogTerm_z_only_Eq3}), (\ref{eqns:GradDivTerm_z_only_Eq3}) into the third component equation in (\ref{eqns:E_Wave_full}) we get an equation in $E_3$ only. Moreover, the first component equation in (\ref{eqns:E_Wave_full}) is $E_1$, $E_3$ - dependent equation, as can be seen by plugging (\ref{eqns:LogTerm_z_only_Eq1}), (\ref{eqns:GradDivTerm_z_only_Eq1}) into it. Since $E_3$ can be found independently, we have an equation in $E_1$. Similar reasoning yields an equation in $E_2$ from the second component equation in (\ref{eqns:E_Wave_full}), thus completing an uncoupling process. The components equations are
\begin{subequations}
\begin{align}
\dfrac{\mu'(z)}{\mu(z)}\left( -\dfrac{\partial E_1}{\partial z} + \dfrac{\partial E_3}{\partial x}\right)-\dfrac{\epsilon'(z)}{\epsilon(z)}\dfrac{\partial E_3}{\partial x}-\nabla^2 E_1 +\epsilon \mu\dfrac{\partial^2 E_1}{\partial t^2} &= -\mu(z)\dfrac{\partial J_1}{\partial t}	\label{eqns:Generalized_wave_z_coeffs_Eq1}\\
\dfrac{\mu'(z)}{\mu(z)}\left( -\dfrac{\partial E_2}{\partial z} + \dfrac{\partial E_3}{\partial y}\right)-\dfrac{\epsilon'(z)}{\epsilon(z)}\dfrac{\partial E_3}{\partial y}-\nabla^2 E_2 + \epsilon \mu\dfrac{\partial^2 E_2}{\partial t^2} &= -\mu(z)\dfrac{\partial J_2}{\partial t}	\label{eqns:Generalized_wave_z_coeffs_Eq2}\\
-\left( \dfrac{\epsilon''(z)}{\epsilon(z)}-\left( \dfrac{\epsilon'(z)}{\epsilon(z)}\right)^2 \right)E_3 -\dfrac{\epsilon'(z)}{\epsilon(z)}\dfrac{\partial E_3}{\partial z} - \nabla^2 E_3 +\epsilon \mu\dfrac{\partial^2 E_3}{\partial t^2} &= -\mu(z)\dfrac{\partial J_3}{\partial t} + \dfrac{\rho(z)}{\epsilon(z)}\epsilon'(z) - \dfrac{\rho'(z)}{\epsilon(z)} \label{eqns:Generalized_wave_z_coeffs_Eq3}
\end{align}
\end{subequations}
and in the time-harmonic case $\underline{E} = \underline{\hat{E}}e^{i \omega t}$, $\underline{J} = \underline{\hat{J}}e^{i \omega t}$:

\begin{subequations}
\begin{align}
\dfrac{\mu'(z)}{\mu(z)}\left( -\dfrac{\partial \hat{E}_1}{\partial z} + \dfrac{\partial \hat{E}_3}{\partial x}\right)-\dfrac{\epsilon'(z)}{\epsilon(z)}\dfrac{\partial \hat{E}_3}{\partial x}-\nabla^2 \hat{E}_1 - \epsilon \mu \omega^2 \hat{E}_1 &= -i\omega \mu(z)\hat{J}_1	\label{eqns:Generalized_Helmholtz_z_coeffs_Eq1}\\
\dfrac{\mu'(z)}{\mu(z)}\left( -\dfrac{\partial \hat{E}_2}{\partial z} + \dfrac{\partial \hat{E}_3}{\partial y}\right)-\dfrac{\epsilon'(z)}{\epsilon(z)}\dfrac{\partial \hat{E}_3}{\partial y}-\nabla^2 \hat{E}_2 - \epsilon \mu \omega^2 \hat{E}_2 &= -i \omega \mu(z)\hat{J}_2	\label{eqns:Generalized_Helmholtz_z_coeffs_Eq2}\\
\left( \dfrac{\epsilon''(z)}{\epsilon(z)}-\left( \dfrac{\epsilon'(z)}{\epsilon(z)}\right)^2 \right)\hat{E}_3 +\dfrac{\epsilon'(z)}{\epsilon(z)}\dfrac{\partial \hat{E}_3}{\partial z} + \nabla^2 \hat{E}_3 +\epsilon \mu \omega^2 \hat{E}_3 &= i \omega \mu(z)\hat{J}_3 - \dfrac{\rho(z)}{\epsilon(z)}\epsilon'(z) + \dfrac{\rho'(z)}{\epsilon(z)} \label{eqns:Generalized_Helmholtz_z_coeffs_Eq3}
\end{align}
\end{subequations}



\subsection{Electric field depending on the direction of propagation coordinate only - homogeneous case}
\subsubsection{Construction of an ODE boundary value problem}
As a first toy problem, let us consider the case where $\hat{E}_3 = \hat{E}_3(z)$, and the source terms $\underline{J}$, $\rho$ vanish. Under these assumptions, equation (\ref{eqns:Generalized_Helmholtz_z_coeffs_Eq3}) becomes a linear, homogeneous ODE with variable coefficients

\begin{align}
\label{eqns:1-D-time-harmonic-E3_z}
\dfrac{d^2 \hat{E}_3}{d z^2} + \dfrac{\epsilon'(z)}{\epsilon(z)}\dfrac{d \hat{E}_3}{d z} + \left( \dfrac{\epsilon''(z)}{\epsilon(z)}-\left( \dfrac{\epsilon'(z)}{\epsilon(z)}\right)^2 + \epsilon(z) \mu \omega^2\right)\hat{E}_3 = 0
\end{align}
We wish to model a plane wave propagation within a heterogeneous layer. The wave may be transmitted into the layer, reflected from it or (typically) a combination of both. The time-harmonic source at the left boundary is thus
\begin{align}
\hat{E}_3(-L) = A e^{i \omega_i}, \qquad A>0, \quad L>0 
\label{eqns:1-D-time-harmonic-z-boundary-source}
\end{align}
where the index $i$ in $\omega_i$ stands for `incidence', as opposed to the Fourier transform variable $\omega$.\\
At the current stage we wish to model the effect of one boundary only. While proper determination of a boundary condition at $+\infty$ is sufficient for an analytic formulation of such a problem, we want to have some numerical perspective as well. A natural way of modeling it is to define a transmission-only boundary condition at some point to the right of $-L$, say $+L$. to avoid reflection, we define a one-sided, right traveling wave at the right boundary. The wave speed is determined by the product $\epsilon \mu$, but note that it should be adapted to a first order PDE formulation:
\begin{align}
\label{eqns:1_D-temporal-traveling-wave-bc}
\left[\dfrac{\partial E_3}{\partial t} +  \dfrac{1}{\sqrt{\epsilon(L)\mu}}\dfrac{\partial E_3}{\partial z} \right ]_{z=L} = 0
\end{align}
and in the time-harmonic case
\begin{align}
\label{eqns:1_D-time-harmonic-traveling-wave-bc}
\left[ i \omega \hat{E}_3 + \dfrac{1}{\sqrt{\epsilon(L)\mu}} \dfrac{\partial \hat{E}_3}{\partial z} \right]_{z = L}=0
\end{align}
  
\subsubsection{Dimensional analysis and the meaning of layer thickness}
One should be aware that the form (\ref{eqns:1-D-time-harmonic-z-boundary-source}) is inadequate for the purpose of parametric and asymptotic analysis. The reason is that series expansions a frequently applied to the solution's coefficients. Using them without care spoils the equation's dimensional consistency. For example, if $\epsilon(z) = \arctan(z)$, series expansion yields $\arctan(z) = z-\dfrac{z^3}{3} + \ldots$. Since $z$ has a length dimension, the series expansion is not dimensionally consistent. Moreover, asymptotic methods for constructing semi-analytic solutions rely heavily on neglecting small terms. To understand the meaning of 'small', parameters should be organized in dimensionless groups. In our case, we wish to determine the relation between the layer's thickness, the incident plane wave frequency and the smooth cutoff function $\epsilon$'s properties: 
\begin{itemize}
\item The function's  dynamic range - the difference between its $-\infty$ and $+\infty$ values, normalized by the $-\infty$ limit to  get an 'effective jump' measure;
\item The function's proximity to a step function, essentially expressed by the size of the function's $z$-derivative at $z=0$ and the rate of its decay toward $\pm \infty$.
\end{itemize}

To obtain a non-dimensional argument for $\epsilon$, we let $\epsilon(z) = \epsilon(z/M)$, where $M$ has units of length. \\
To get non-dimensional groups of parameters, we perform the change of variables $z=M s$, in two steps. First we replace the independent variable in $\hat{E}_3(z))$:
\begin{align}
\label{eqns:1-D-time-harmonic-z-normalized-boundary-source}
\dfrac{1}{M^2}\dfrac{d^2 \hat{E}_3}{d s^2} + 
\dfrac{\epsilon'(z)}{M \epsilon(z)}\dfrac{d \hat{E}_3}{d s} + 
\left( \mu \omega^2\epsilon(z)-\left(\dfrac{\epsilon'(z)}{\epsilon(z)}\right)^2 + \dfrac{\epsilon''(z)}{\epsilon(z)}\right)\hat{E}_3 = 0 
\end{align}
Suppose that $\epsilon_1 = \epsilon(-L)$ and $\Delta \epsilon = \epsilon(L)-\epsilon(-L)$. Let $\epsilon(z) = A+Bc\left( \dfrac{z}{M}\right)$, where $c(z/M)$ is a differentiable, monotonous function of $z$ of the real line, and  $A$, $B$ are determined by the relations
\begin{align*}
\begin{cases}
A + Bc\left( -\dfrac{L}{M}\right) = \epsilon_1	\\
A + Bc\left( \dfrac{L}{M}\right) = (1+ r)\epsilon_1
\end{cases}
\end{align*}
where
\begin{align}
r = \dfrac{\Delta \epsilon}{\epsilon_1} 
\end{align} 
is a dimensionless parameter representing the layer's edge-to-edge difference of dielectric permeability, and
\begin{align}
a = \dfrac{L}{M}
\end{align} 
is a dimensionless parameter for the effective layer width. We solve
for $A$, $B$ and substitute backwards using $z = M s$ and get
\begin{align}
\label{eqns:1-D-time-harmonic-s-dielectric-cutoff}
\epsilon(s) = \dfrac{\epsilon_1 ((r+1) c(-a)-c(a)-r c(s))}{c(-a)-c(a)}
\end{align}

Substituting (\ref{eqns:1-D-time-harmonic-s-dielectric-cutoff}) into (\ref{eqns:1-D-time-harmonic-z-normalized-boundary-source}) while keeping track of the fact that the $\epsilon(z)$ derivatives should be calculated using the chain rule, we get

\begin{align}
\label{eqns:1_D-time-harmonic-s-E_3-dimensional}
\dfrac{d^2\hat{E}_3}{ds^2}&-\dfrac{r c'(s) }{(r+1) c(-a)-c(a)-r c(s)}\dfrac{d\hat{E}_3}{ds}+ \notag\\
&\Big(-\dfrac{r c''(s)}{(r+1) c(-a)-c(a)-r
   c(s)}-\dfrac{r^2 c'(s)^2}{(-(r+1) c(-a)+c(a)+r c(s))^2} \notag \\
   &-\dfrac{\mu  M^2 r \omega ^2
   \epsilon_1 c(s)}{c(-a)-c(a)}+\dfrac{\mu  M^2 r \omega ^2 \epsilon_1
   c(-a)}{c(-a)-c(a)}+\mu  M^2 \omega ^2 \epsilon_1\Big)\hat{E}_3(s)=0
\end{align}
Next we notice that the parameter grouping  
\begin{align}
\label{eqns:1_D-time-harmonic-normalized-freq}
\Omega = M \omega \sqrt{\epsilon_1 \mu}
\end{align}
is dimensionless, as $[\sqrt{\epsilon_1 \mu}] = \left[ \dfrac{1}{v}\right] = \dfrac{\text{time}}{\text{length}}$, while $[M] = \text{length}$ and $[\omega] = \dfrac{\text{rad}}{\text{time}}$. 
Substituting (\ref{eqns:1_D-time-harmonic-normalized-freq}) into (\ref{eqns:1_D-time-harmonic-s-E_3-dimensional}) we get a fully-non-dimensional form of the equation, dependent on three parameters only:
\begin{align}
\label{eqns:1_D-time-harmonic-s-E_3-non-dim}
\dfrac{d^2\hat{E}_3}{ds^2}&-\dfrac{r c'(s) }{(r+1) c(-a)-c(a)-r c(s)}\dfrac{d\hat{E}_3}{ds}+ \notag\\ 
&\Big(-\dfrac{r c''(s)}{(r+1) c(-a)-c(a)-r
   c(s)}-\dfrac{r^2 c'(s)^2}{(-(r+1) c(-a)+c(a)+r c(s))^2} \notag \\
   &-\dfrac{r \Omega ^2
   c(s)}{c(-a)-c(a)}+\dfrac{r \Omega ^2 c(-a)}{c(-a)-c(a)}+\Omega ^2\Big)\hat{E}_3=0
\end{align}

and the boundary conditions (\ref{eqns:1-D-time-harmonic-z-boundary-source}),
 (\ref{eqns:1_D-time-harmonic-traveling-wave-bc}) become
\begin{subequations}
\begin{align}
\label{eqns:1-D-time-harmonic-z-boundary-source-normalized}
&\hat{E}_3(-a) = Ae^{i \omega_i} \\
\label{eqns:1_D-time-harmonic-traveling-wave-bc-normalized}
&\left[ i \Omega \sqrt{1+r} \hat{E}_3 +  \dfrac{d \hat{E}_3}{d s} \right]_{s = a}=0
\end{align}
\end{subequations}


\begin{comment}
Letting the effective layer width $a \ll 1$ and neglecting terms multiplied by it, $\epsilon(a s)$ becomes nearly constant. In terms of perturbation theory, the approximation problem in powers of $a$ is regular and equation (\ref{eqns:1-D-time-harmonic-z-normalized-boundary-source}) has a zero-order approximation by the constant coefficients equation
\begin{align}
\label{eqns:1-D-time-harmonic-z-normalized-boundary-source-regular-per-const-coef}
\dfrac{d^2 \hat{E}_3}{d s^2} +   L^2\epsilon(0) \mu \omega^2 \hat{E}_3 = 0 
\end{align}
On the other hand, $a \gg 1$ yields a singular perturbation problem because naive expansion in powers of $\dfrac{1}{a}$ would yield an algebraic equation as a zero-order approximation for equation (\ref{eqns:1-D-time-harmonic-z-normalized-boundary-source}), which is not compatible with the boundary conditions. In such case, a totally new approximation method is required, based on singular perturbation theory. 

Considering again the easier case $a \ll 1$, we wish to quantify the concept of a layer thickness. To do so, we must take into account the magnitudes of the absolute thickness $L$, the source frequency $\omega$ and the inverse square wave speed $\epsilon \mu = \dfrac{1}{v^2}$.

Since equation (\ref{eqns:1-D-time-harmonic-z-normalized-boundary-source-regular-per-const-coef}) has constant coefficients, we are justified in using relations between the wavelength $\lambda$ and $\omega$:
\begin{align*}
\omega = \dfrac{2\pi v}{\lambda}
\end{align*}
Therefore
\begin{align}
L^2 \epsilon \mu \omega^2 = \dfrac{L^2}{v^2} \cdot \dfrac{(2\pi v)^2}{\lambda^2} = 4\pi^2 \dfrac{L^2}{\lambda^2}
\end{align}
we can therefore expect a solution to be non-oscillatory within a layer when $L \ll \lambda$.\\
In this work, we are dealing with visible light optics, where the wavelength $\lambda \approx 0.5 \cdot 10^{-6}m$. In such case, layers of width $10-100 \mu m$, typical to the semiconductor industry, are considered thick, while biological cells membrane, which is $3-10 nm$ thick, may be considered narrow. 
\end{comment}

 

\subsubsection{The ODE coefficients and discriminant for the $\tan^{-1}(s)$ cutoff  }
First let us construct a smooth, increasing cutoff function based on $\tan^{-1}(s)$. Substituting $\tan^{-1}(s)$ into (\ref{eqns:1-D-time-harmonic-s-dielectric-cutoff}), we get

\begin{align}
\label{eqns:epsilon_atan_cutoff}
\epsilon(s) = \dfrac{1}{2} \epsilon_1 \left(\dfrac{r \tan ^{-1}(s)}{\tan^{-1}(a)}+r+2\right)
\end{align}
The ODE (\ref{eqns:1-D-time-harmonic-E3_z}) has the general form 

\begin{align}
\label{eqns:general_ODE_form}
\hat{E}_3''(s) + c_1(s) \hat{E}_3'(s) + c_0(s)\hat{E}_3(s) = 0
\end{align}
Substituting (\ref{eqns:epsilon_atan_cutoff}) we now get the coefficients
\begin{subequations}
\begin{align}
c_1(s) &= \dfrac{r}{\left(s^2+1\right) \left((r+2) \tan ^{-1}(a)+r \tan ^{-1}(s)\right)} \label{eqns:1-D-time-harmonic-E3_s_Atan_coef_1_r}\\
c_0(s) = &\dfrac{1}{2} \Big(-\dfrac{2 r \left(2 (r+2) s \tan ^{-1}(a)+2 r s \tan
   ^{-1}(s)+r\right)}{\left(s^2+1\right)^2 \left((r+2) \tan ^{-1}(a)+r \tan
   ^{-1}(s)\right)^2}+\notag\\
   &\dfrac{r \Omega ^2 \tan ^{-1}(s)}{\tan ^{-1}(a)}+(r+2) \Omega
   ^2\Big)
 \label{eqns:1-D-time-harmonic-E3_s_Atan_coef_0_r}
\end{align}
\end{subequations}

Let us consider the coefficients shapes and value ranges. In (\ref{eqns:1-D-time-harmonic-E3_s_Atan_coef_1_r}), solving for the $c_1(s)$ denominator vanishing we get
\begin{align}
\label{eqns:1-D-time-harmonic-c_1-Atan-denom-vanish}
r = -\dfrac{2 \tan ^{-1}(a)}{\tan ^{-1}(
   s)+\tan ^{-1}(a)}
\end{align}
but $a>0$ and $|s|\leq a$; therefore
\begin{align*}
\tan ^{-1}(a) > 0, \qquad \tan ^{-1}(s)+\tan ^{-1}(a) \geq 0
\end{align*}
and the right hand side of (\ref{eqns:1-D-time-harmonic-c_1-Atan-denom-vanish}) is negative. But since $r>0$, we get that equation (\ref{eqns:1-D-time-harmonic-c_1-Atan-denom-vanish}) has no solution. Hence $c_1(s)$ and continuous and positive for $-a \leq s \leq a$. Similar reasoning applied to equation (\ref{eqns:1-D-time-harmonic-E3_s_Atan_coef_0_r}) shows that $c_0(s)$ is continuous.\\

\subsubsection{Coefficients profiles - asymptotics in $r$  } 
When $r \rightarrow 0+$, $c_1(s) \rightarrow 0+$ and $c_0(s)\rightarrow \Omega^2$ uniformly. That is intuitive because $r$ reflects the dielectric permittivity relative jump size; when the jump is zero, we get that equation (\ref{eqns:1-D-time-harmonic-z-normalized-boundary-source}) has constant coefficients. The first and second derivative terms $\epsilon'(z)$, $\epsilon''(z)$ in (\ref{eqns:1-D-time-harmonic-E3_z}) vanish.\\
The situation is somewhat more complex when $r \rightarrow \infty$, as we get
\begin{align}
\label{eqns:1-D-time-harmonic-E3_s_Atan_coef_1_r_r_inf}
c_1(s) = \dfrac{1}{\left(s^2+1\right) \left(\tan ^{-1}(a)+\tan ^{-1}(s)\right)}
\end{align}
In this case, the denominator vanishes at a single point $s = -a$, so that $c_1(s)$ approaches positive infinity there. We thus have a strong damping near $s=-1$. At $s=0$ we get

Differentiating (\ref{eqns:1-D-time-harmonic-E3_s_Atan_coef_1_r_r_inf}) in $s$:
\begin{align*}
c'_1(s)=-\dfrac{2 s \tan ^{-1}(a)+2 s \tan ^{-1}(s)+1}{\left(s^2+1\right)^2
   \left(\tan ^{-1}(a)+\tan ^{-1}(s)\right)^2}
\end{align*}
Since $-a \leq s \leq a$, the is always a range of $s$ values in which we can replace the numerator's $\tan^{-1}(s)$ term by $s$ and obtain
\begin{align*}
c_1'(s) \approx -\dfrac{2 s \tan ^{-1}(a)+2 s^2+1}{\left(s^2+1\right)^2 \left(\tan
   ^{-1}(a)+\tan^{-1}(s)\right)^2}
\end{align*}
The denominator is positive, and the numerator vanishes at
\begin{align*}
s_{1,2} = \dfrac{1}{2} \left(-\tan ^{-1}(a) \pm \sqrt{\tan ^{-1}(a)^2-2}\right)
\end{align*}
We see that the derivative change of sign occurs only if $\tan^{-1}(a)>\sqrt{2}$. The points $s_{1,2}$ are not symmetric with respect to origin, but they both approach zero as $a$ increases. \\
Concluding the case of $r \rightarrow \infty$ case, $c_1(s)$ is large near $s=-1$ when $r$ is large. It has also a $\delta$-function-like behavior when $r$ and $a$ are large.\\




\subsubsection{Coefficients profiles - asymptotics in $a$  }

\subsubsection{Coefficients profiles - asymptotics in $\Omega$  }

\subsubsection{Asymptotic profiles of $c_1(s)$}
To understand the asymptotic effect of $r$ variation on $c_1(s)$, we begin by examining the limits $r \rightarrow 0+$, $r \rightarrow \infty$, $a \rightarrow 0+$, $a \rightarrow \infty$.\\

When $a \rightarrow 0+$, we have 
\begin{align}
\label{eqns:1-D-time-harmonic-E3_s_Atan_coef_1_non_dim_a_0}
c_1(s) = \dfrac{r}{r s+r+2}
\end{align}
The numerator vanishes at $s = -1-\dfrac{2}{r}<-1$. Hence $c_1(s)\vert_{a=0}$ is well defined for all $-1 \leq s \leq 1$. Yet, when $r$ is large, the discontinuity point approaches $s=-1$, and we get large values of $c_1(s)$ near $s = -1$. \\
Finally, when $a \rightarrow \infty$, $c_1(s) \rightarrow 0$ uniformly. \begin{figure}
\begin{center}
\includegraphics[width=1.0\textwidth]{plotsCoef1Atan}
\end{center}
\caption {$c_1(s)$ profiles}.\\
Top left: $a=0.01$, $r=0.01$; top right: $a=0.01$, $r=50$; bottom left: $a=10$, $r=50$; bottom right: $a=70$, $r=50$.


\label{fig:c1Atan}
\end{figure}

\begin{figure}
\begin{center}
\includegraphics[width=1.0\textwidth]{plotsCoef1AtanRinf}
\end{center}
\caption {$c_1(s)$ asymptotic profiles - $r \rightarrow \infty$}.\\
Left: $a=1$, right: $a=20$.


\label{fig:c1AtanRinf}
\end{figure}



\subsubsection{Asymptotic profiles for $c_0(s)$} 

We calculate zero and infinity limits for each parameter in equation (\ref{eqns:1-D-time-harmonic-E3_s_Atan_coef_0_r}) assuming that the rest of the parameters are $O(1)$. At this stage, we shall not address the more complex problem of relating the sizes of the parameters pairwise (or triple-wise).\\
Let us begin examining each case.

\begin{enumerate}



\item Asymptotics in $a$:
	\begin{enumerate}
	\item $a$ large: 
	\begin{align}
	\label{eqns:1-D-time-harmonic-E3_s_Atan_coef_0_non_dim_a_inf}
	c_0(s) \approx \dfrac{1}{2} \Omega ^2 (r (1+\text{sgn}(s))+2) = \begin{array}{cc}
	\Big \{ & 
	\begin{array}{cc}
	 \Omega ^2 & s<0 \\
 	(r+1) \Omega ^2 & s \geq 0 \\
	\end{array}
 	\\
	\end{array}
	\end{align}
	
\begin{figure}
\begin{center}
\includegraphics[width=0.8\textwidth]{plotsC0AtanAAsymptotics}
\end{center}
\caption {$c_0(s)$ asymptotic profiles - limits in $a$}.\\
Top: $a \rightarrow \infty$; bottom: $a \rightarrow 0$.\\
Bottom left: $r=0.01$, $\Omega = 1$; bottom right: $r=50$, $\Omega = 1$.
\label{fig:c0AtanAlim}
\end{figure}


An interpretation for this result may be that when the layer's effective width is large, jump size tends to dominate the smoothness of the cutoff function. The jump size is $r \Omega^2$.
	\item $a$ small:
	\begin{align}
	c_0(s) \approx \dfrac{1}{2} \Omega ^2 (r s+r+2)-\dfrac{r^2}{(r s+r
	+2)^2}
	\end{align}
	\end{enumerate}
	calculating the $s$-derivative,
	\begin{align*}
	c_0'(s) = \dfrac{2 r^3}{(r s+r+2)^3}+\dfrac{r \Omega ^2}{2} 
	\end{align*}
	we have that $c_0'(s)>0$ for all $-1 \leq s \leq 1$, hence monotone 		there. When $r$ is large, $c_0(s)$'s discontinuity point approaches 		$s=-1$ from the left.
\item Asymptotics in $r$:
	\begin{enumerate}
	\item $r$ large:\\
	The first and second terms in equation (\ref{eqns:1-D-time-harmonic-E3_s_Atan_coef_0_non_dim}) are bounded when $r \rightarrow \infty$. The third term, on the other hand, grows like $r$. Calculating the limit of $\dfrac{c_0(s)}{r}$, $r \rightarrow \infty$, and multiplying again by $r$ we get the estimate
	\begin{align}
	c_0(s) \approx \dfrac{r\Omega ^2 \left(\tan ^{-1}(a s)+\tan ^{-1}(a)\right)}{2 \tan ^{-1}(a)}
	\end{align}
	When $a$ is large, $c_0(s)$ tends to the step function (\ref{eqns:1-D-time-harmonic-E3_s_Atan_coef_0_non_dim_a_inf}). When $a$ is small, $\tan^{-1}(a) \approx a$, $\tan^{-1}(as) \approx as$, and $c_0(s)$ tends to the linear function 
	\begin{align}
	c_0(s) \approx \dfrac{1}{2} r (s+1) \Omega ^2
	\end{align}
	
\begin{figure} 
\begin{center}
\includegraphics[width=0.8\textwidth]{plotsC0AtanRlargeAsymptotics}
\end{center}
\caption {$\dfrac{c_0(s)}{r}$ asymptotic profiles - $r \rightarrow \infty$}.\\
Left: $a = 0.1$, $\Omega = 50$; Right: $a = 50$, $\Omega=50$.
\label{fig:c0AtanRlim}
\end{figure}

	\item $r$ small:\\
	
\item Asymptotics in $\Omega$
	\begin{enumerate}
	\item $\Omega$ large:\\
	In that case, the first and second terms in (\ref{eqns:1-D-time-harmonic-E3_s_Atan_coef_0_non_dim}) can be neglected. we get
	\begin{align}
	c_0(s) \approx \Omega ^2 \left(\dfrac{r \left(\tan ^{-1}(a s)+\tan ^{-1}(a)\right)}{2 \tan ^{-1}(a)}+1\right)
	\end{align}
	When $a$ is small, we can replace $\tan^{-1}(a)$ by $a$ and $\tan^{-1}(as)$ by $as$. In that case, $c_0(s)$ tends to the linear function
	\begin{align}
	c_0(s) \approx \Omega ^2\left( \dfrac{1}{2} r (s+1) + 1 \right)
	\end{align}
	interpolating $\Omega^2$ at $s=-1$ and  and $\Omega^2(r+1)$ at $s=1$.
	
	\item $\Omega$ small:\\
	Here, the third term in (\ref{eqns:1-D-time-harmonic-E3_s_Atan_coef_0_non_dim}) is dropped. 
	\begin{align}
	\label{eqns:1-D-time-harmonic-E3_s_Atan_coef_0_non_dim_Omega_zero}
	c_0(s) \approx &-\dfrac{a^2 r^2}{\left(a^2 s^2+1\right)^2 \left(r \tan ^{-1}(a s)+(r+2)
   \tan ^{-1}(a)\right)^2} \notag \\
   &-\dfrac{2 a^3 r s}{\left(a^2 s^2+1\right)^2
   \left(r \tan ^{-1}(a s)+(r+2) \tan ^{-1}(a)\right)}
	\end{align}
	
\begin{figure} 
\begin{center}
\includegraphics[width=0.8\textwidth]{plotsC0AtanOmegaAsymptotics}
\end{center}
\caption {$c_0(s)$ asymptotic profiles - $\Omega \rightarrow 0$, $a = 10$, $r = 1$}.

\label{fig:c0AtanOmegalim}
\end{figure}


\end{enumerate}

	When $r$ is large and $a$ is kept $O(1)$, we get a monotone profile with large and negative values near $s=-1$, as demonstrated in the case 1(b). \\
	When $r$ is kept $O(1)$ and $a$ is large, we get a different phenomenon. The dominant term in (\ref{eqns:1-D-time-harmonic-E3_s_Atan_coef_0_non_dim_Omega_zero}) is the second one, which can be thought of as a shifted and scaled version of the function $-\dfrac{s}{(s^2 +1)^2}$. This is an odd function, fast decaying in $s$ and has to peaks, symmetric with respect to the origin - maximum and minimum. For small $s$ values, the denominator's $O(a^4)$ growth is suppressed, and the numerator's $O(a^3)$ a $\delta'(s)$-like behavior. 
	\end{enumerate}

\end{enumerate}

  
\newpage

\subsubsection{Simulations for $\arctan(z)$ cutoff}

Each of the following sequence of figures consists of 4 parts: plots of $c_0(s)$ (\ref{eqns:1-D-time-harmonic-E3_s_Atan_coef_1_r}) and $c_1(s)$ (\ref{eqns:1-D-time-harmonic-E3_s_Atan_coef_0_r}) at the top, a figure of $c_0(s)$ and $c_1(s)$ combined and equation (\ref{eqns:general_ODE_form}) with these coefficients solution at the bottom. In all cases, the left boundary condition () has $A=1$ and $\omega_i=\dfrac{\pi}{4}$.  The right boundary condition (\ref{eqns:1_D-time-harmonic-traveling-wave-bc-normalized}) is written, in terms of the non-dimensional parameter $\Omega$, 
\begin{align}
\label{eqns:1_D-time-harmonic-traveling-wave-bc-non-dim}
\left[i \Omega \sqrt{1+r}\hat{E}_3 + \dfrac{\partial \hat{E}_3}{\partial s} \right]_{s=1}=0
\end{align}

\begin{figure} 
\begin{center}
\includegraphics[width=0.8\textwidth]{plotsCoefsSols1}
\end{center}
\caption {Coefficients and solution - $a = 0.1$, $r = 0.1$, $\Omega = 0.1$}.

\label{fig:atanCoefsPlots1}
\end{figure}

\begin{figure} 
\begin{center}
\includegraphics[width=0.8\textwidth]{plotsCoefsSols2}
\end{center}
\caption {Coefficients and solution - $a = 0.1$, $r = 10$, $\Omega = 0.1$}.

\label{fig:atanCoefsPlots2}
\end{figure}

\begin{figure} 
\begin{center}
\includegraphics[width=0.8\textwidth]{plotsCoefsSols3}
\end{center}
\caption {Coefficients and solution - $a = 0.1$, $r = 0.1$, $\Omega = 10$}.

\label{fig:atanCoefsPlots3}
\end{figure}

\begin{figure} 
\begin{center}
\includegraphics[width=0.8\textwidth]{plotsCoefsSols4}
\end{center}
\caption {Coefficients and solution - $a = 0.1$, $r = 10$, $\Omega = 10$}.

\label{fig:atanCoefsPlots4}
\end{figure}

\begin{figure} 
\begin{center}
\includegraphics[width=0.8\textwidth]{plotsCoefsSols5}
\end{center}
\caption {Coefficients and solution - $a = 10$, $r = 0.1$, $\Omega = 0.1$}.

\label{fig:atanCoefsPlots5}
\end{figure}

\begin{figure} 
\begin{center}
\includegraphics[width=0.8\textwidth]{plotsCoefsSols6}
\end{center}
\caption {Coefficients and solution - $a = 10$, $r = 10$, $\Omega = 0.1$}.

\label{fig:atanCoefsPlots6}
\end{figure}

\begin{figure} 
\begin{center}
\includegraphics[width=0.8\textwidth]{plotsCoefsSols7}
\end{center}
\caption {Coefficients and solution - $a = 10$, $r = 0.1$, $\Omega = 10$}.

\label{fig:atanCoefsPlots7}
\end{figure}

\begin{figure} 
\begin{center}
\includegraphics[width=0.8\textwidth]{plotsCoefsSols8}
\end{center}
\caption {Coefficients and solution - $a = 10$, $r = 10$, $\Omega = 10$}.

\label{fig:atanCoefsPlots8}
\end{figure}



\newpage
%\subsubsection{Simulations for $\tanh(z)$ cutoff}
\begin{comment}
\subsubsection{Passage to a Sturm-form BVP}
Equation (\ref{eqns:1-D-time-harmonic-E3_z}) is a linear, homogeneous real valued variable coefficients second order ODE. We can estimate its solution's period of oscillation using Sturm's comparison theorem \cite{BirkhoffRota1989}:

\begin{thm}
\label{thm:Sturm-Comparison}
Let $f(x)$ and $g(x)$ be nontrivial solutions of the DEs $u''+p(x)u=0$ and $v''+q(x)v=0$, respectively., where $p(x) \geq q(x)$. Then $f(x)$ vanishes at least once between any two zeros of $g(x)$, unless $p(x) \equiv q(x)$ and if $f$ is a constant multiple of $g$. 
\end{thm}
As we can see, the theorem requires the equation's first order derivative coefficient to vanish. Thus equation (\ref{eqns:1-D-time-harmonic-E3_z}) must be transformed to such form. \\
Letting $\hat{E}_3 = \epsilon^{-1/2}u(z)$ we get
\begin{align}
\label{eqns:1-D-harmonic-z-Sturm}
\dfrac{d^2 u}{dz^2} +\dfrac{\omega^2 \epsilon^3(z)-\frac{3}{4}(\epsilon'(z))^2+\frac{1}{2}\epsilon(z)\epsilon''(z)}{\epsilon(z)}u=0
\end{align}
Since $\epsilon(z)$, $\epsilon'(z)$, $\epsilon''(z)$ are bounded and $\epsilon(z)$ is strictly positive, we can take $\omega^2$ large enough so that the right hand side of (\ref{eqns:1-D-harmonic-z-Sturm}) would be larger than $\dfrac{\omega^2}{2\epsilon_2}$ for all $-\delta<z<\delta$.In such case, the solutions for (\ref{eqns:1-D-harmonic-z-Sturm}) oscillate at a frequency of at least $\dfrac{\omega}{2\pi\sqrt{2 \epsilon_2}}$.




\subsection{The one-dimensional case in source-free, non-magnetic media}
$\mathbf{Banergee - J_c and \rho as source}$, Griffiths - Gauss's law

Neglecting magnetic phenomena we have $\mu = 1$. We also assume no free currents - $\underline{J} = 0$, no charge accumulation - $\rho = 0$, and that $\epsilon$ is $x$ - dependent - $\epsilon = \epsilon(x)$. We get the $\underline{E} , \underline{H}$ representation 

\begin{subequations}
	\begin{align}
        \nabla \times \underline{E} &= -\dfrac{\partial \underline{H}}{\partial t}  \label{eqns:ourCurlE_Maxwell}\\
		\nabla \times \underline{H} &=  \epsilon(x)\dfrac{\partial  \underline{E}}{\partial t} \label{eqns:ourCurlH_Maxwell}	\\	
		\nabla \cdot ( \epsilon(x) \underline{E} ) &= 0	\\		
		\nabla \cdot \underline{H} &= 0	
	\end{align}
	\label{eqns:ourCurlMaxell}
\end{subequations}

Applying the vector calculus identity (\ref{eqns:CurlCurlID}) and the $\underline{H}$ - curl equation (\ref{eqns:ourCurlH_Maxwell}) we get
\begin{align*}
\nabla (\nabla \cdot \underline{E}) - \nabla^2\underline{E} + \epsilon(x) \dfrac{\partial^2 \underline{E}}{\partial t^2} = 0
\end{align*}

Next, we use the assumption $\rho = 0$ in equations (\ref{eqns:DivD_Maxwell}), (\ref{eqns:constitutiveDepsE}) combined. We get
\begin{align*}
\nabla \cdot (\epsilon(x) \underline{E}) = 0
\end{align*}
Expanding using the vector calculus formula (\ref{eqns:DivProdID}) we get
\begin{align*}
\epsilon(x) \nabla \cdot \underline{E} + \underline{E} \epsilon'(x) = 0
\end{align*}
Thus 
\begin{align*}
\nabla \cdot \underline{E} = -\underline{E} \cdot \dfrac{\epsilon'(x)}{\epsilon(x)}
\end{align*}




\section{Maxwell's equations in 1-D}
As a first model problem we consider an electromagnetic plane wave hitting an infinite, layered slab of infinite thickness at a right angle. \\
The curl equations  (\ref{eqns:ourCurlE_Maxwell}), (\ref{eqns:ourCurlH_Maxwell}) can be put in a vector notation:

\begin{subequations}
\begin{eqnarray}
	\dfrac{\partial}{\partial t} \mat{E_x \\E_y \\E_z} = 
	\dfrac{1}{\epsilon(x)} \mat{	\partial_y H_z - \partial_z H_y	\\
								\partial_z H_x - \partial_x H_z	\\
								\partial_x H_y - \partial_y H_x	
							}	\\		
	\dfrac{\partial}{\partial t}
						\mat{H_x \\H_y \\H_z} = 
					  -	\mat{	\partial_y E_z - \partial_z E_y	\\
								\partial_z E_x - \partial_x E_z	\\
								\partial_x E_y - \partial_y E_x	
							}	
\end{eqnarray}
\end{subequations}
We assume that the wave front propagates in the $x$ direction. Hence $y$, $z$ derivatives of $\underline{E} , \underline{H}$ vanish. We get 

\begin{subequations}
\begin{eqnarray}
	\dfrac{\partial}{\partial t} \mat{E_x \\E_y \\E_z} = 
	\dfrac{1}{\epsilon(x)} \mat{	0	\\
								 - \partial_x H_z	\\
								\partial_x H_y 	
							}	\\		
	\dfrac{\partial}{\partial t}
						\mat{H_x \\H_y \\H_z} = 
					  	\mat{	0	\\
								  \partial_x E_z	\\
								-\partial_x E_y 
							}	
\end{eqnarray}
\end{subequations}
which can be reduced into two first order PDE systems

\begin{subequations}
\begin{eqnarray}
	\dfrac{\partial}{\partial t} \mat{E_y \\H_z} &= \mat{0 & -1/\epsilon(x)	\\ -1 & 0} \dfrac{\partial}{\partial x} \mat{E_y \\ H_z}		
		\label{eqns:ourMaxwellEyHzPolarized}\\
	\dfrac{\partial}{\partial t} \mat{E_z \\H_y} &= \mat{0 & 1/\epsilon(x)	\\ 1 & 0} \dfrac{\partial}{\partial x} \mat{E_z \\ H_y}
		\label{eqns:ourMaxwellEzHyPolarized}
\end{eqnarray}
\end{subequations}
Systems (\ref{eqns:ourMaxwellEyHzPolarized}), (\ref{eqns:ourMaxwellEzHyPolarized}) are hyperbolic, as the systems matrices have distinct real eigenvalues $\pm \dfrac{1}{\sqrt{\epsilon(x)}}$. Using the method of characteristics one can shown \cite{Gustaffson1995} that since each eigenvalue has the same sign all through the domain, a solution for such a systems is a superposition of a right traveling wave and a left traveling wave. \\

For an analytic solution we therefore have to supply a boundary condition at the left boundary for the right traveling wave and a zero boundary condition at $+\infty$ for the left traveling wave, since we assume that no information is carried from that direction. \\



Since we are interested in visible light optics, the analytic problem left boundary condition should be of the form $A e^{i \left(\omega t + \phi\right)}$, where $A>0$ is the source amplitude, $\omega >0$ is its angular frequency and $-\pi/2 < \phi \leq \pi/2$ is the phase. The complex term is a convenient way to facilitate the use of phasors, and a real part of the solution should be taken after the computation.\\

Numerically speaking, we should make a finite computational domain simulate the problem's infinite spatial domain. In that sense we should typically force artificial boundary conditions, simulating the wave front's continued propagation in the same media, or, in the case of a reflected wave, backward exit to vacuum. Such boundary conditions are called `absorbing boundary conditions'. We'll use the shorthand `ABC'.



\section{Inverse problem for the 1-D one-way wave equation}

Hyperbolic PDE systems of the form (\ref{eqns:ourMaxwellEyHzPolarized}) and (\ref{eqns:ourMaxwellEzHyPolarized}) can be transformed into coupled 1-D one way wave equations
\begin{align*}
u^j_t+a^j(x)u^j_x=0, \qquad j=1,2
\end{align*}
where coupling occurs at the boundary. It therefore makes sense use parameter identification problems for END HERE 
\begin{align*}
\begin{cases}
u_t + a(x) u_x &= 0,	\qquad -L<x< \infty \\
u(x,0) &= 0	\\
u(-L,t) &= g(t)
\end{cases}
\end{align*}
Most of the time we shall be interested in a boundary condition of the form 
\begin{align}
g(t) = A e^{i \left(\omega t + \phi \right)}
\label{eqns:complexExpBC}
\end{align}
When $a(x)=a$ is constant, we have a traveling wave solution 
\begin{align*}
u(x,t) = u(x-at)
\end{align*}
as can be shown by direct differentiation. Applying the initial condition and the left boundary condition we get

\begin{align*}
u(x,t) = g\left( t-\dfrac{x+L}{a} \right)
\end{align*}
and for the boundary condition (\ref{eqns:complexExpBC}), we get
\begin{align}
u(x,t) =A e^{i \left(\omega  \left(t-\frac{x+L}{a}\right)+\phi \right)}
\end{align}

Now assume that
\end{comment} 

\section{Parameter estimation for 1-D boundary value problems}
The inverse problem we wish to solve is the estimation of the parameters defining dielectric profile $\epsilon\left(\dfrac{z}{M}\right)$ from a solution of equation (\ref{eqns:1-D-time-harmonic-z-normalized-boundary-source}) subject to the boundary conditions (\ref{eqns:1-D-time-harmonic-z-boundary-source-normalized}), (\ref{eqns:1_D-time-harmonic-traveling-wave-bc-normalized}). This is a complex project since it involves solving systems of nonlinear equations, where the target parameters may have different scales.\\
It is thus instructive to first analyze the recovery of parameters from  a reasonably simple 'toy model'. We use this procedure to demonstrate the difficulties we may entangle, resulting in general properties of 1-D boundary value problems.


\subsection{The first model: the harmonic oscillator with Dirichlet boundary conditions}
We consider the boundary value problem
\begin{align}
\label{eqns:}
\begin{cases}
&y'' + \omega^2 y =0 \\
y(-a) &= A, \qquad y(a) = 0
\end{cases}
\end{align}
Solving, we get 
\begin{align}
y(x) = \dfrac{1}{2} (A \sec (\omega a ) \cos (\omega x)-A \csc ( \omega a ) \sin ( \omega x))
\end{align}
%This solution is valid for $\omega a \neq \pi k, \dfrac{\pi}{2}+ \pi k$, which are, as can be easily verified, the problem's eigenvalues.
 


\subsubsection{Characterization of the forward problem }







\newpage
\begin{thebibliography}{2d}
\bibitem{ArfkenWeber2005}
Arfken GB. Weber HJ. Mathematical methods for physicists, 6th ed. Elsevier academc press, Amsterdam, 2005
\bibitem{BirkhoffRota1989}
Birkhoff G. Rota GC. Ordinary differential equations, 4th ed. Wiley, New York, 1989.
\bibitem{Gustaffson1995}
Gustaffson B. Kreiss HO. Oliger J. Time dependent problems and difference methods. Wiley, New York, 1995.
\bibitem{Hecht2002}
Hecht E. Optics, 4th ed. Addison Wesley, San Francisco, CA, 2002.
\bibitem{Strikwerda2004}
Strikwerda J. Finite difference schemes and partial differential equation, 2nd ed. SIAM,  Wadsworth and Brooks / Cole, Pacific Grove, CA, 2004
\end{thebibliography}

\end{document}

 








