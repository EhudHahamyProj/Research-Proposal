\documentclass[12pt,twoside]{report} 
\usepackage[titletoc]{appendix}
\usepackage[reqno,tbtags]{amsmath}
\usepackage[active]{srcltx} % not for published version
\usepackage{amsthm}
\usepackage{amsfonts}
\usepackage{amssymb}
\usepackage{graphicx}
\usepackage{subfigure}
\usepackage{verbatim}
\usepackage{mathtools}
% put any defs here
%\newcommand{\mat}[2][cccccccccccccccccccccccccccccccccccccccccccc]{\left(\begin{array}{#1}#2 \\ \end{array} \right)}

\newtheorem{lem}{Lemma}[subsubsection]
\newtheorem{thm}{Theorem}[subsection]
\newtheorem{cor}{Corollary}[subsubsection]
\newtheorem{define}{Definition}[subsubsection]
\newcommand{\HRule}{\rule{\linewidth}{0.5mm}}



\begin{document}

\begin{titlepage}



\begin{center}

\textsc{ \LARGE
Tel Aviv University \\[1.5cm]
}
\textsc{\Large
Raymond and Beverly Sackler Faculty of Exact Sciences \\ [0.4cm]
School of Mathematical Sciences \\
Department of Applied Mathematics
} \\[1.0cm]
\textsc{Research proposal}

\HRule \\[0.4cm]
{ \huge \bfseries The inverse medium problem for time-harmonic Maxwell's equations \\[0.4cm] } 
\HRule \\[1.5cm]
\hfill \break

\noindent
\begin{minipage}{0.4\textwidth}
\begin{flushleft} \large
\emph{Author:}\\
Ehud \textsc{Hahamy}, \\
id. 032025140
\end{flushleft}
\end{minipage}%
\begin{minipage}{0.4\textwidth}
\begin{flushright} \large
\emph{Supervisor:} \\
Prof. ~Nir \textsc{Sochen}
\end{flushright}
\end{minipage}

\vfill
{\large \today}




\end{center}

%\HRule \\[0.4cm]{
%The inverse medium problem for time-harmonic Maxwell's equations
%}





\end{titlepage}


\begin{abstract}

We study the electromagnetic wave propagation in inhomogeneous media, considered as the forward problem for the `inverse medium problem' in which a structured dielectric premittivity parameters are to be recovered from field measurements. \\
In this proposal, dimensional analysis is applied to a medium's time-independent parameters. Embedding in the time-harmonic Maxwell's equations system leads to a framework in which asymptotic expansions of the solutions reveal the propagating wave's qualitative behavior.\\
Careful dimensional grouping yields valid comparisons of parameter magnitudes. Numerical simulations demonstrate variation of solution features corresponding to different asymptotic assumptions.\\
Solution behavior patterns generated by sigmoid-shaped dielectric profiles in the 1-D case are discussed, and a basic parameter recovery technique is demonstrated.


 
%The framework is applied to a dimensionless dielectric permittivity jump size parameter reconstruction.

\end{abstract}

\newpage

\chapter{Introduction}
\section{Methodology  }
\label{sec:methodology}

In this proposal we consider the so-called `inverse medium problem' for electromagnetic fields from an asymptotic perspective. We wish to recover the parameters defining a medium's smooth, structured, non-constant dielectric permittivity profile using measurements of electromagnetic wave propagating in it.\\
An `asymptotic' approach means that some statement regarding the magnitude (either `small' or `large') of one parameter at least is assumed. We also assume a time-harmonic source. By `structure' we mean that the dielectric profile has the approximate shape of some closed-form function, possibly subject to noise. \\
Application for such formulation can be found tumor detection, biological tissue and membrane characterization, non-destructive defect location in CMOS, MEMES and nano-lithography, and to many other fields.\\

The inverse medium problem is commonly approached using methods involving the solution of systems of nonlinear equations (a strategy we take here), machine learning techniques, etc. Many such methods require some prior information regarding the possible data behavior. An important example is the class of gradient-based methods for solution of optimization problems, where the prior information is manifested by an initial guess of the target values, close enough for the method to converge.\\
Most of the computational results presented here are aimed to acquire some relation between the parameters defining a sigmoid-shaped dielectric permittivity profile and the solution's spatial behavior. Yet, the proposed framework can be applied, in principle, to a wide variety of profile classes.\\

As the dielectric profile may be quite general, one cannot expect closed-form solution for Maxwell's equations. However, when some statement regarding the relative sizes of the profile parameters is made, one can apply asymptotic methods to derive a uniform approximation of the solution. Asymptotic solutions are of interest since they tend to be relatively simple and, when properly used, emphasize behavior dependencies otherwise obscure.\\

Magnitude statements are often made regarding the medium's thickness, the incident wave's frequency and the size of jump in dielectric permittivity across an interface. Here we introduce a unified framework in which all these quantities are relative to each other, so that the actual quantities to be varied are dimensionless. We thus obtain a meaningful interpretation of `small' and `large' quantities. In fact, distinction between fine and crude dielectric profile details is necessary for any useful definition of inverse mapping resolution. Indeed, sensitivity analysis for an inversion scheme should measure the effect of `small' variations of the parameters to be recovered, typically using measurements taken at a `large' distance from the object.\\
This interpretation has special importance in the inverse scattering theory, which is often partitioned into two distinct research domains: the inverse medium problem handling smoothly varying profiles, and the inverse obstacle problem, typically assuming piecewise-constant profiles. From our perspective, this separation is only a matter of scale. We assume that, looking close enough, any jump in permittivity can be considered regular. We thus model all objects using smooth functions, with singularity being an extreme, `asymptotic', case where a certain dimensionless parameter tends to infinity. \\


Asymptotic and numerical solutions are validated by trial-and-error interlacing comparisons. Clearly, one must verify that as little artifacts as possible interfere with the solutions due to computational error. Parasitic boundary effects are notorious for causing error in the field of numerical solution of PDEs. With this problem in mind, absorbing boundary conditions of impedance type were combined in the model's basic formulation, so that both asymptotic and numerical solutions are computed within the same model problem.\\
   
\section{Simplifying assumptions}
Maxwell's equations are a  system of six coupled first order PDEs, with the vector components of the electric and magnetic fields are the unknowns to be computed. The system can be transformed into a set of three coupled wave equations. Without loss of generality, the magnetic field's components are eliminated in the process, so that the electric field is to be determined. \\

We present an argument showing that when the system's coefficients depend on the direction of propagation (usually denoted $z$) solely, the system still uncouples in the following sense: the equation for the electric field's $z$- component is independent of the other two equations. It can be solved by itself and its solution becomes an aggregate of the $x$ and $y$ components equations. The resulting equations are independent of each other.\\

Still, each of the resulting wave equations contains a three-dimensional Laplacian, complicating the solution process.\\
In this proposal we aim to introduce the concepts and tools needed to solve the inverse medium problem without diving into the technical subtleties of the full solution. We shall thus handle only the $z$-component equations with its Laplacian reduced to a second $z$-derivative, resulting in a 1-D wave equation with variable coefficients. \\
This equation can be further simplified into a second order ODE boundary value problem using a one-sided temporal Fourier transform. 

\section{Dimensional analysis}
\label{sec:IntoDimAnaslysis}
In section \ref{sec:methodology} we mentioned that asymptotic analysis relies on a statement regarding the size of the model parameters. As the notion of magnitude is relative, dimensional analysis must be performed. \\
Further, our model equations coefficients are determined by a set of parameters, each diffenernt in physical dimensions. Hence comparison of magnitudes can only take place using dimensional grouping.

\section{Previous work}
Both subjects discussed in this proposal, namely electromagnetic waves propagation in inhomogeneous media and the inverse medium problem, are vast. The selection of references made here is thus necessarily a matter of personal interest and assumed relevance to future work.

\subsection{General perturbation theory}
The basic mathematical machinary used in this proposal is asymptotic analysis. This classical topic is covered in many texts. \\
The book by Bender and Orszag \cite{BenderOrszag1999} is a standard reference, with many examples and clear, intuitive explanations.\\
Nayfeh's book \cite{Nayfeh1981} has many detailed examples, and considered a good reference for systematic organization of complex calculations.\\
Murdock's book \cite{Murdock1991} is organized as a complement to Nayfeh's book. It contains Many rigorous derivations and error analysis for methods illustrated in other textbooks.

\subsection{Asymptotic methods for wave propagation}
Joseph Keller, one of the founders of the high-frequency asymptotic theory of the Helmholtz equation and Maxwell's equations, has written a very readable book chapter in \cite{Keller1995} summarizing the topics of wave-fronts and ray theory relating physical and geometric optics. The theory is applied to representative problems involving reflection, transmission and diffraction in both homogeneous and inhomogeneous media. The WKB method is applied to the equations, obtaining eikonal equations. The nonlinear equations can be solved using the method of characteristics.\\
Some of the finer computational details of applying the WKB method to wave equations are skipped in this text. The case of inhomogeneous refractive index is described in \cite{SecklerKeller1959}.\\

Relating this proposal Keller's work, we notice that the WKB method asymptotic assumtion is high temporal frequency. No asymptotic statement is made regarding the diffraction index or the dielectric permittivity.\\

A book by Kravtsov and Orlov \cite{KravstovOrlov1990} specifically handles wave propagation in inhomogeneous media from a geometrical optics point of view. Perturbation techniques are applied to the eikonal equation's spatial media. and many numerical examples are presented. \\

The previous mentioned two books are monographs, surveying the material with high detail and examples, but are not meant to be used for systematic, academic course-style style studying. Joseph Keller's student, Norman Bleistein, published a textbook \cite{Bleistein1984} organized for that purpose. A large part of the book is devoted to asymptotic methods of approximating the temporal Fourier integral transforming the wave equation into Helmholtz equation and vice versa. These topics are not dealt with at \cite{Keller1995}, \cite{KravstovOrlov1990}, but much  of the methods applicability depends on them. 

\subsection{Regularization of PDE coefficients}
Continuing the line of thought presented in section \ref{sec:methodology}, PDE coefficients are assumed non-constant and differentiable in this work. We aim to understand the solution's behavior under certain assumptions about the parameters determining the coefficients shapes. \\
Some aspects of this problem had been studied in a different context. Numerical solutions of PDEs with discontinuous coefficients tend to have low convergence rates. Methods to solve this problem by local smoothing of the coefficients were introduced by Peskin \cite{Peskin1977, Peskin2002}. Tornberg and Engquist \cite{TornbergEngquist2003} used decomposition of error into an analytic part and numerical part to facilitate analysis. They have also introduced high order conditions for controlling the numerical error based on the regularity of the modified coefficients, the regularization kernel width and the numerical step size.\\
Kashdan and Turkel \cite{KashdanTurkel2006} implemented the Tornberg-Engquist regularization method as a framework for applying a fourth-order numerical scheme to the time-harmonic Maxwell equation in a medium composed of two homogeneous materials with an interface in which the dielectric coefficient is discontinuous.

\subsection{The inverse medium problem for elelctromagnetic waves}
David Colton and Rainer Kress are two of the most prominent authors in the filed of inverse acoustic and electromagnetic scattering theory. With a body of work starting at the 1970's, including  hundreds of published papers and over ten books (written together and apart), even a shallow survey of their contribution to this filed is way out of the scope of this proposal. Their writing is dense; reading required a considerable previous background in the theory of wave equations, Maxwell's equations, functional analysis, perturbation theory, etc. \\
Selecting one text which is directly relevant to this proposal's subject matter, the monograph \cite{ColtonKress1998} has a chapter handling the inverse medium problem. The point of view taken there is, as often in Colton and Kress' writing, of extracting information about the subject matter (in this case, parameters defining the structure of an inhomogeneous medium) from a so-called `far field pattern'. Other chapters of interest in the monograph handle the forward problem in both acoustic and electromagnetic cases, and a presentation of the theory of ill-posed problems from a scattering theory perspective. 

A good place to start the study of the Colton-Kress theory is the introductionary textbook \cite{Kirsch2011}, written by Kress's former student, Andreas Kirsch. This text contains no direct treatment of the inverse medium problem, but provides elementary treatment of the main topics of the Colton-Kress theory: the far-field pattern, boundary values at infinity, and some non-trivial notions in complex and functional analysis.\\

The problem of recovering a slab's complex refractive index from transmission and reflection spectra was addressed in several works. Nichelatti \cite{Nichelatti2002}, for example, obtained a formula for a slab's assumed constant refractive index in terms of measured reflection and transmission spectra, valid when $k^2 \ll n$, where $n$, $k$ are the refractive index real and imaginary parts, respectively.

\subsection{Inverse problems for the wave equation - stability characterization}
The important subject of ill posedness is frequently mentioned in the inverse problems literature and elsewhere, but an elementary, examples-motivated and yet mathematically concise introduction for the subject is quite rare.\\
A short and illustrative resource is a chapter in Kress' book \cite{Kress1997}. Ill posed finite dimensional problems and the commonly-used solution for this problem - Tikhonov regularization - are analyzed from a singular value decomposition point of view. Criteria for selecting an optimal 
regularization parameter are discussed.\\

The stability of the inverse medium problem the 1-D wave equation 
\begin{align*}
\rho(z) \dfrac{\partial ^2 y}{\partial t^2}-\dfrac{\partial}{\partial z}\left( \mu(z) \dfrac{\partial y}{\partial z}\right) = 0
\end{align*}
with boundary excitation at $z=0$ and zero initial conditions in which one tries to acquire as much information as possible about the functions $\rho(z)$, $\mu(z)$ is considered in \cite{Bamberger1979}. The solution is measured at the boundary $z=0$ for all time $t$ values. It is shown that there exist sets of couple $\sigma = (\rho(z), \mu(z))$ that are indistinguishable by a boundary measurement $Y(t)$, thus forming equivalence classes. A distance between such equivalence classes is defined so that it is weak enough to make any portion $\Sigma =\{ \sigma: \sigma_{\min} \leq \sigma \leq \sigma_{\max}\}$ of the space compact, but strong enough to ensure the Lipschitz continuity of the mapping $\sigma \rightarrow Y$. That ensures the existence of a solution for the inverse problem set as an optimization problem of $\Sigma$. The fact that the defined distance is much weaker than the usual $L^2$ norm explains the inverse problem's tendency to instability. \\

Recently, Bao, Li et al. \cite{BaoLi2005} studied the problem of inverse scattering from inhomogeneous media in two and three dimensions from a perspective close to ours. They study the stability properties of the forward problem as well as the continuity, Fr\`echet differentiability and continuation properties of the inverse mapping. One motivation \cite{BaoHouLi2007} is to characterize the performance of optimization techniques for solving inverse problems. The study of the forward problem yields a good initial guess for such methods. \\
The Bao-Li has given rise to in-object \cite{ItoJinZou2012} and far-field \cite{LiZou2012} sampling techniques. Estimation of error for a linearized inverse medium mapping is presented in \cite{Bao2010}.





\chapter{The generalized Helmholtz equations in heterogeneous media}

\section{Maxell's equation in MKS units}
We state Maxwell's equations in differential form.
\begin{subequations}

	\begin{align}
        \nabla \times \overrightarrow{H}-\dfrac{\partial \overrightarrow{D}}{\partial t} &= \overrightarrow{J}  \label{eqns:CurlH_Maxwell}	\\	
        \nabla \times \overrightarrow{E} + \dfrac{\partial \overrightarrow{B}}{\partial t}&= 0  \label{eqns:CurlE_Maxwell} \\
		\nabla \cdot \overrightarrow{D} &=  \rho	 \label{eqns:DivD_Maxwell}\\		
		\nabla \cdot \overrightarrow{B} &= 0	
	\end{align}
\end{subequations}
$\overrightarrow{H}$ is the magnetic field, $\overrightarrow{E}$ is the electic field, $\overrightarrow{D}$ is the electric displacement vector, $\overrightarrow{B}$ is the magnetic flux density, $\overrightarrow{J}$ is the electric current density, and $\rho$ is the electric charge density.\\
In this work we assume the linear constitutive relations 
\begin{subequations}
\begin{align}
        \overrightarrow{D} = \epsilon \overrightarrow{E} \label{eqns:constitutiveDepsE}\\
        \overrightarrow{B} = \mu \overrightarrow{H}	\label{eqns:constitutiveBH}\\
        \overrightarrow{J} = \sigma \overrightarrow{E}
\end{align}
\end{subequations}
The dielectric permittivity $\epsilon$, the magnetic permeability $\mu$ and the conductivity $\sigma$ are assumed a differentiable function of $x$, $y$, $z$ and time independent.\\
The PDE subsystem (\ref{eqns:CurlH_Maxwell}), (\ref{eqns:CurlE_Maxwell}) is time dependent. Hence initial conditions should be set to the vector quantities $\overrightarrow{H}$, $\overrightarrow{D}$, $\overrightarrow{E}$ and $\overrightarrow{B}$. Note, however, that using the constitutive relations (\ref{eqns:constitutiveDepsE}), (\ref{eqns:constitutiveBH}), determination of
\begin{subequations}
\begin{align}
\label{eqns:MaxwellSysInitCondE}
\overrightarrow{E}(x,y,z,0) = \overrightarrow{E}_0(x,y,z,t)\\
\label{eqns:MaxwellSysInitCondH}
\overrightarrow{H}(x,y,z,0) = \overrightarrow{H}_0(x,y,z,t)
\end{align}
\end{subequations}
is sufficient.


\section{Transforming Maxwell's equations into generalized wave equations in heterogeneous media}
\label{sec:TransMaxwellToWave}


We use a standard step of applying certain vector calculus formulas to Maxwell's equations and to the constitutive relations (see full derivation in appendix \ref{append:GeneralizedWaveDeriv}). We get a coupling of three second order PDEs in the electric field $\overrightarrow{E}$'s components. 
\begin{align}
\label{eqns:E_Wave_full}
\mu \nabla \left( \dfrac{1}{\mu} \right) \times \left(\nabla \times \overrightarrow{E} \right) +  \nabla (\nabla \cdot \overrightarrow{E}) - \nabla^2 \overrightarrow{E} + \mu \epsilon\dfrac{\partial^2 \overrightarrow{E}}{\partial t^2} + \mu \dfrac{\partial \overrightarrow{J}}{\partial t} = 0
\end{align}
The magnetic field $\overrightarrow{H}$ components were eliminated. We'll refer to the system (\ref{eqns:E_Wave_full}) as the generalized wave equations system.\\
System (\ref{eqns:E_Wave_full}) is of second order in time. Hence initial conditions for $\overrightarrow{E}$ and for $\dfrac{\partial \overrightarrow{E}}{\partial t}$ are required for all $(x,y,z)$ in the equation's spatial domain. The initial condition for $\overrightarrow{E}$ is given by (\ref{eqns:MaxwellSysInitCondE}). For later convenience we denote $\overrightarrow{E}_0(x,y,z) = \overrightarrow{f}(x,y,z)$. \\
To determine the initial condition for $\dfrac{\partial E}{\partial t}$ we use the initial condition (\ref{eqns:MaxwellSysInitCondE}) along with 
(\ref{eqns:CurlE_Maxwell}) and (\ref{eqns:constitutiveDepsE}), so that
\begin{align*}
\dfrac{\partial \overrightarrow{E}}{\partial t}(x,y,z,0) = -\overrightarrow{J}(x,y,z,0) + \dfrac{1}{\epsilon} \nabla \times \overrightarrow{H}_0(x,y,z)
\end{align*}
We shall later denote $\overrightarrow{g}(x,y,z) =  \dfrac{\partial \overrightarrow{E}}{\partial t}(x,y,z,0)$.\\

In summary, we get

\begin{subequations}
\begin{align}
\label{eqns:GeneralizedWaveEqInitCondsE}
\overrightarrow{f}(x,y,z) &= \overrightarrow{E}(x,y,z,0) = \overrightarrow{E}_0(x,y,z)	\\
\label{eqns:GeneralizedWaveEqInitCondsDEDT}
\overrightarrow{g}(x,y,z) &=\dfrac{\partial \overrightarrow{E}}{\partial t}(x,y,z,0) = -\overrightarrow{J}(x,y,z,0) + \dfrac{1}{\epsilon} \nabla \times \overrightarrow{H}_0(x,y,z)
\end{align}
\end{subequations}

\section{Analysis of the source and material-dependent terms in the generalized wave equations system}

Compared to the standard wave equation $\epsilon \mu\dfrac{\partial ^2\overrightarrow{E}}{\partial t^2} =  \nabla^2 \overrightarrow{E}$, equation (\ref{eqns:E_Wave_full}) has extra terms:
\begin{subequations}
\begin{align}
\mu \nabla \left( \dfrac{1}{\mu} \right) \times \left(\nabla \times \overrightarrow{E} \right) \label{eqns:E_Wave_extra_term_log_mu}\\
\nabla (\nabla \cdot \overrightarrow{E}) \label{eqns:E_Wave_extra_term_Grad_Div_E} \\
\mu \dfrac{\partial \overrightarrow{J}}{\partial t} \label{eqns:E_Wave_extra_term_mu_DJDt} 
\end{align}
\end{subequations}
While the term (\ref{eqns:E_Wave_extra_term_mu_DJDt})is a straight- forward source term, the other two have a more complicated structure. In this section we'll describe the contributions of the terms (\ref{eqns:E_Wave_extra_term_log_mu}), (\ref{eqns:E_Wave_extra_term_Grad_Div_E}) to each equation in the generalized wave equations system.

\subsection{The term $\mu \nabla \left( \dfrac{1}{\mu} \right) \times \left(\nabla \times \vec{E} \right)$}

In this work we are primarily interested in visible light optics. Materials which are transparent in visible light are essentially ``nonmagnetic'' (see \cite{Hecht2002}, chapter 3). Evidently, the term (\ref{eqns:E_Wave_extra_term_log_mu}) vanishes if $\mu$ is constant, and specifically for nonmagnetic substances satisfying $\mu \equiv 1$. Still, there exist magnetic substances that are transparent in the infrared and microwave regions of the spectrum. For the sake of completeness, we'll decompose the term into a component-by component form to understand its contribution to each wave equation in  the system (\ref{eqns:E_Wave_full}).\\
The full derivation is involves a non-trivial application of the Levi-Civita symbol calculus. Here we quote only the final result

\begin{align}
\label{eqns:E_Wave_extra_term_log_mu_expanded}
\left(\mu \nabla \left( \dfrac{1}{\mu}\right) \times \left(\nabla \times \overrightarrow{E} \right) \right)_i = 
\mu \nabla \left(\dfrac{1}{\mu} \right) \cdot \left( \dfrac{\partial \overrightarrow{E}}{\partial x_i}- \nabla E_i \right)
\end{align}

and the full derivation can be found in appendix \ref{append:GradLogTermComponenetwiseDeriv}.






\subsection{The term $\nabla (\nabla \cdot \vec{E})$}
To decompose the term (\ref{eqns:E_Wave_extra_term_Grad_Div_E}), let us substitute the constitutive relation (\ref{eqns:constitutiveDepsE}) into equation (\ref{eqns:DivD_Maxwell}). We get
\begin{align*}
\nabla \cdot (\epsilon \overrightarrow{E}) = \rho
\end{align*}
Using the product rule for differentiation (\ref{eqns:DivProdID}), 
\begin{align*}
\epsilon\nabla \cdot \overrightarrow{E} + \overrightarrow{E} \cdot \nabla \epsilon = \rho
\end{align*}
Hence
\begin{align}
\label{eqns:DivE}
\nabla \cdot \overrightarrow{E} =\dfrac{1}{\epsilon} \left(\rho - \overrightarrow{E} \cdot \nabla \epsilon\right)
\end{align}

Taking the gradient of both sides of (\ref{eqns:DivE}) we get
\begin{align*}
\nabla \left( \nabla \cdot \overrightarrow{E} \right)_i &=\nabla\left( \dfrac{\rho}{\epsilon}\right)_i-\dfrac{\partial}{\partial x_i}\left( E_j \dfrac{\partial \log(\epsilon)}{\partial x_j}\right) \\&= \nabla\left( \dfrac{\rho}{\epsilon}\right)_i - \left( \dfrac{\partial E_j}{\partial x_i}\dfrac{\partial \log(\epsilon)}{\partial x_j} + E_j \dfrac{\partial^2 \log(\epsilon)}{\partial x_i \partial x_j}\right)
\end{align*}
Hence
\begin{align}
\label{eqns:E_Wave_extra_term_Grad_Div_E_expanded}
\nabla \left( \nabla \cdot \overrightarrow{E} \right)_i = \nabla\left( \dfrac{\rho}{\epsilon}\right)_i - \left(\dfrac{\partial \overrightarrow{E}}{\partial x_i} \cdot \nabla \log(\epsilon) + \overrightarrow{E} \cdot \dfrac{\partial}{\partial x_i} \nabla \log(\epsilon) \right)
\end{align}

\subsection{A 1-D layered media - system coefficients depending on a single coordinate}

Let us now assume that the $\epsilon = \epsilon(z)$, $\mu=\mu(z)$, $\rho = \rho(z)$, and see what can be learned.\\
Using (\ref{eqns:E_Wave_extra_term_log_mu_expanded}),The term (\ref{eqns:E_Wave_extra_term_log_mu}) has the following form, component-wise:
\begin{subequations}
\begin{align}
\left(  \mu \nabla \left(\dfrac{1}{\mu} \right)  \times \left(\nabla \times \overrightarrow{E} \right)\right)_1 &= \dfrac{\mu'(z)}{\mu(z)}\left( -\dfrac{\partial E_1}{\partial z} + \dfrac{\partial E_3}{\partial x}\right) \label{eqns:LogTerm_z_only_Eq1}\\
\left(  \mu \nabla \left( \dfrac{1}{\mu} \right)  \times \left(\nabla \times \overrightarrow{E} \right)\right)_2 &= \dfrac{\mu'(z)}{\mu(z)}\left( -\dfrac{\partial E_2}{\partial z} + \dfrac{\partial E_3}{\partial y}\right) \label{eqns:LogTerm_z_only_Eq2} \\
\left(  \mu \nabla \left( \dfrac{1}{\mu} \right)  \times \left(\nabla \times \overrightarrow{E} \right)\right)_3 &= 0 \label{eqns:LogTerm_z_only_Eq3}
\end{align}
\end{subequations}
Similarly, we use (\ref{eqns:E_Wave_extra_term_Grad_Div_E_expanded}) to bring the term (\ref{eqns:E_Wave_extra_term_Grad_Div_E}) components to
\begin{subequations}
\begin{align}
(\nabla (\nabla \cdot \overrightarrow{E}))_1 &= -\dfrac{\epsilon'(z)}{\epsilon(z)} \dfrac{\partial E_3}{\partial x}	\label{eqns:GradDivTerm_z_only_Eq1}\\
(\nabla (\nabla \cdot \overrightarrow{E}))_2 &= -\dfrac{\epsilon'(z)}{\epsilon(z)} \dfrac{\partial E_3}{\partial y} \label{eqns:GradDivTerm_z_only_Eq2}\\
(\nabla (\nabla \cdot \overrightarrow{E}))_3 &= -\dfrac{\rho(z)}{\epsilon(z)}\epsilon'(z) + \dfrac{\rho'(z)}{\epsilon(z)}-\left( \dfrac{\epsilon''(z)}{\epsilon(z)}-\left( \dfrac{\epsilon'(z)}{\epsilon(z)}\right)^2 \right)E_3 \label{eqns:GradDivTerm_z_only_Eq3}\\
& -\dfrac{\epsilon'(z)}{\epsilon(z)}\dfrac{\partial E_3}{\partial z}	\notag
\end{align}
\end{subequations}

Plugging (\ref{eqns:LogTerm_z_only_Eq3}), (\ref{eqns:GradDivTerm_z_only_Eq3}) into the third component equation in (\ref{eqns:E_Wave_full}) we get an equation in $E_3$ only. Moreover, the first component equation in (\ref{eqns:E_Wave_full}) is $E_1$, $E_3$ - dependent equation, as can be seen by plugging (\ref{eqns:LogTerm_z_only_Eq1}), (\ref{eqns:GradDivTerm_z_only_Eq1}) into it. Since $E_3$ can be found independently, we have an equation in $E_1$. Similar reasoning yields an equation in $E_2$ from the second component equation in (\ref{eqns:E_Wave_full}), thus completing an uncoupling process. The components equations are
\begin{subequations}
\begin{align}
\label{eqns:Generalized_wave_z_coeffs_Eq1}
\dfrac{\mu'(z)}{\mu(z)}\left( -\dfrac{\partial E_1}{\partial z} + \dfrac{\partial E_3}{\partial x}\right)-\dfrac{\epsilon'(z)}{\epsilon(z)}\dfrac{\partial E_3}{\partial x}-\nabla^2 E_1 +\epsilon(z) \mu(z)\dfrac{\partial^2 E_1}{\partial t^2} &\\ 
=-\mu(z)\dfrac{\partial J_1}{\partial t} \notag\\
\label{eqns:Generalized_wave_z_coeffs_Eq2}
\dfrac{\mu'(z)}{\mu(z)}\left( -\dfrac{\partial E_2}{\partial z} + \dfrac{\partial E_3}{\partial y}\right)-\dfrac{\epsilon'(z)}{\epsilon(z)}\dfrac{\partial E_3}{\partial y}-\nabla^2 E_2 + \epsilon(z) \mu(z)\dfrac{\partial^2 E_2}{\partial t^2} &\\
= -\mu(z)\dfrac{\partial J_2}{\partial t} \notag\\
\label{eqns:Generalized_wave_z_coeffs_Eq3}	
-\left( \dfrac{\epsilon''(z)}{\epsilon(z)}-\left( \dfrac{\epsilon'(z)}{\epsilon(z)}\right)^2 \right)E_3 -\dfrac{\epsilon'(z)}{\epsilon(z)}\dfrac{\partial E_3}{\partial z} - \nabla^2 E_3 +\epsilon(z) \mu(z)\dfrac{\partial^2 E_3}{\partial t^2} &\\
= -\mu(z)\dfrac{\partial J_3}{\partial t} + \dfrac{\rho(z)}{\epsilon(z)}\epsilon'(z) - \dfrac{\rho'(z)}{\epsilon(z)} \notag 
\end{align}
\end{subequations}
The PDE set can be written in the frequency domain using the one-sided Fourier transform

\begin{subequations}
\begin{align}
\label{eqns:time-harmonic-E-field-trasns}
 \overrightarrow{\hat{E}}(x,y,z,\omega) = \int_0^\infty \overrightarrow{E}(x,y,z,t)e^{i \omega t}dt\\
\label{eqns:time-harmonic-J-current-trasns}
  \overrightarrow{\hat{J}}(x,y,z,\omega) = \int_0^\infty \overrightarrow{J}(x,y,z,t)e^{i \omega t}dt
\end{align}
\end{subequations}

into
\begin{subequations}
\begin{align}
\label{eqns:Generalized_Helmholtz_z_coeffs_Eq1}
\dfrac{\mu'(z)}{\mu(z)}\left( -\dfrac{\partial \hat{E}_1}{\partial z} + \dfrac{\partial \hat{E}_3}{\partial x}\right)-\dfrac{\epsilon'(z)}{\epsilon(z)}\dfrac{\partial \hat{E}_3}{\partial x}-\nabla^2 \hat{E}_1 - \epsilon(z) \mu(z) \omega^2 \hat{E}_1 &\\
= i\omega \mu(z)\hat{J}_1 + i \omega f_1(x,y,z) - g_1(x,y,z) \notag\\
\label{eqns:Generalized_Helmholtz_z_coeffs_Eq2}
\dfrac{\mu'(z)}{\mu(z)}\left( -\dfrac{\partial \hat{E}_2}{\partial z} + \dfrac{\partial \hat{E}_3}{\partial y}\right)-\dfrac{\epsilon'(z)}{\epsilon(z)}\dfrac{\partial \hat{E}_3}{\partial y}-\nabla^2 \hat{E}_2 - \epsilon(z) \mu(z) \omega^2 \hat{E}_2 &\\= i \omega \mu(z)\hat{J}_2 + i \omega f_2(x,y,z)-g_2(x,y,z) \notag\\
\label{eqns:Generalized_Helmholtz_z_coeffs_Eq3}
\left( \dfrac{\epsilon''(z)}{\epsilon(z)}-\left( \dfrac{\epsilon'(z)}{\epsilon(z)}\right)^2 \right)\hat{E}_3 +\dfrac{\epsilon'(z)}{\epsilon(z)}\dfrac{\partial \hat{E}_3}{\partial z} + \nabla^2 \hat{E}_3 +\epsilon(z) \mu(z) \omega^2 \hat{E}_3 &\\
= i \omega \mu(z)\hat{J}_3 - \dfrac{\rho(z)}{\epsilon(z)}\epsilon'(z) + \dfrac{\rho'(z)}{\epsilon(z)} +\notag \\ i \omega f_3(x,y,z)- g_3(x,y,z)\notag 
\end{align}
\end{subequations}
Unlike (\ref{eqns:Generalized_wave_z_coeffs_Eq1}) -   (\ref{eqns:Generalized_wave_z_coeffs_Eq3}), the system (\ref{eqns:Generalized_Helmholtz_z_coeffs_Eq1}) -
(\ref{eqns:Generalized_Helmholtz_z_coeffs_Eq3}) is time- independent, and therefore cannot directly account for initial conditions. This is the reason for the application of the one-sided Fourier representation (\ref{eqns:time-harmonic-E-field-trasns}), transforming the initial conditions  contributions to the equations forcing terms. \\

Characterization of the conditions on the parameters defining the PDE systems (\ref{eqns:Generalized_Helmholtz_z_coeffs_Eq1}) -
(\ref{eqns:Generalized_Helmholtz_z_coeffs_Eq3}) coefficients allowing inversion of the solutions for the system (\ref{eqns:Generalized_Helmholtz_z_coeffs_Eq1}) -
(\ref{eqns:Generalized_Helmholtz_z_coeffs_Eq3}) back into time-dependent solutions of the system (\ref{eqns:Generalized_wave_z_coeffs_Eq1}) -   (\ref{eqns:Generalized_wave_z_coeffs_Eq3})
is a research goal outside the scope of this proposal. \\ However, to make sure that the integrals (\ref{eqns:time-harmonic-E-field-trasns}), (\ref{eqns:time-harmonic-J-current-trasns}) converge, $\omega$ may taken to be conplex-valued, so that its imaginary part would impose exponential decay on the solution. In such case, and under certain assumptions about the decay of $\overrightarrow{\hat{E}}(x,y,z,\omega)$, $\overrightarrow{\hat{J}}(x,y,z,\omega)$ in $\omega$, these functions are analytic in $\omega$ and their inverse Fourier transform can, in principle, be calculated using complex variable contour integration.

\chapter{The electric field in wave propagation direction}

\section{Construction of an ODE boundary value problem}
As a first toy problem, let us consider the case where $E_3 = \hat{E}_3(z)$, the source terms $\overrightarrow{J}$, $\rho$ and the initial conditions $\overrightarrow{f}$, $\overrightarrow{g}$ vanish, and the magnetic permeability $\mu(z)$ is constant.  Under these assumptions, equation (\ref{eqns:Generalized_wave_z_coeffs_Eq3}) becomes
\begin{align}
\label{eqns:1-D-spatio-temporal-E3_z}
\dfrac{\partial^2 E_3}{\partial z^2} + \dfrac{\epsilon'(z)}{\epsilon(z)}\dfrac{\partial E_3}{\partial z} + \left( \dfrac{\epsilon''(z)}{\epsilon(z)}-\left( \dfrac{\epsilon'(z)}{\epsilon(z)}\right)^2 \right)E_3   = \epsilon(z)\mu\dfrac{\partial ^2 E_3}{\partial t^2}
\end{align}


We wish to model a plane wave propagation within a heterogeneous layer. Since the reduced model (\ref{eqns:1-D-time-harmonic-E3_z}) is spatially one-dimensional, we model an incidence plane wave by determination of the solution's value at the left boundary. Temporally we assume that the incidence wave hits the boundary at time zero, so that
\begin{align}
\label{eqns:1-D-temporal-source-BC-E3_z}
E_3(-L,t) = f(t)H(t)
\end{align}
where $H(t)$ is the Heaviside step function. \\
At the current stage we wish to model the effect of one boundary source only. While proper determination of a boundary condition at $+\infty$ is sufficient for an analytic formulation of such a problem, we want to have some numerical perspective as well. A natural way of modeling it is to define a transmission-only boundary condition at some point to the right of $-L$, say $+L$. We avoid a numerical artifact of reflection from the artificially-defined boundary  by defining a one-sided, right traveling wave at the right boundary. The wave speed is determined by the product $\epsilon \mu$, but note that it should be adapted to a first order PDE formulation:
\begin{align}
\label{eqns:1_D-temporal-traveling-wave-bc}
\left[\dfrac{\partial E_3}{\partial t} +  \dfrac{1}{\sqrt{\epsilon(L)\mu}}\dfrac{\partial E_3}{\partial z} \right ]_{z=L} = 0
\end{align}



The frequency domain counterpart of equation (\ref{eqns:1-D-spatio-temporal-E3_z}) is the linear, homogeneous ODE with variable coefficients

\begin{align}
\label{eqns:1-D-time-harmonic-E3_z}
\dfrac{d^2 \hat{E}_3}{d z^2} + \dfrac{\epsilon'(z)}{\epsilon(z)}\dfrac{d \hat{E}_3}{d z} + \left( \dfrac{\epsilon''(z)}{\epsilon(z)}-\left( \dfrac{\epsilon'(z)}{\epsilon(z)}\right)^2 + \epsilon(z) \mu \omega^2\right)\hat{E}_3 = 0
\end{align}
which is the Laplacian - reduced version of (\ref{eqns:Generalized_Helmholtz_z_coeffs_Eq3}) with zero contribution from the initial conditions.\\
 
For a full definition of the problem, we must add boundary conditions. 


 The Fourier-transformed analog of the source at the left boundary is 
\begin{align}
\hat{E}_3(-L,\omega) = A(\omega) e^{i \phi(\omega)}, \qquad A(\omega)\geq 0 
\label{eqns:1-D-time-harmonic-z-boundary-source}
\end{align}
where the $\phi$ is the incidence wave's phase. This is an amplitude-phase representation of the complex-valued Fourier transform of the left boundary source term
\begin{align}
F(\omega) = \int_{-\infty}^\infty f(t)H(t)e^{i \omega t}dt = \int_0^\infty f(t)e^{i\omega t}dt
\end{align}

As for the right boundary condition (\ref{eqns:1_D-temporal-traveling-wave-bc}), it is readily transformed into
\begin{align}
\label{eqns:1_D-time-harmonic-traveling-wave-bc}
\left[ i \omega \hat{E}_3 + \dfrac{1}{\sqrt{\epsilon(L)\mu}} \dfrac{\partial \hat{E}_3}{\partial z} \right]_{z = L}=0
\end{align}

It should be stressed that in order to solve the time-dependent problem (\ref{eqns:1-D-spatio-temporal-E3_z}), (\ref{eqns:1-D-temporal-source-BC-E3_z}) (with vanishing initial conditions), one should (theoretically) solve the BVP
(\ref{eqns:1-D-time-harmonic-E3_z}), (\ref{eqns:1-D-time-harmonic-z-boundary-source}), (\ref{eqns:1_D-time-harmonic-traveling-wave-bc}) for each frequency $\omega$ and the perform an inverse Fourier transform.\\


\section{Dimensional analysis }
Continuing the discussion in section \ref{sec:IntoDimAnaslysis}, one should be aware that the form (\ref{eqns:1-D-time-harmonic-z-boundary-source}) is inadequate for the purpose of parametric and asymptotic analysis. The reason is that series expansions a frequently applied to the solution's coefficients. Using them without care spoils the equation's dimensional consistency. For example, if $\epsilon(z) = \arctan(z)$, series expansion yields $\arctan(z) = z-\dfrac{z^3}{3} + \ldots$. Since $z$ has a length dimension, the series expansion is dimensionally inconsistent. Moreover, asymptotic methods for constructing semi-analytic solutions rely heavily on neglecting small terms. To understand the meaning of 'small', parameters should be organized in dimensionless groups. In our case, we wish to determine the relation between the layer's thickness, the incident plane wave frequency and the dielectric profile function $\epsilon$ properties.
\begin{comment} 
\begin{itemize}
\item The dielectric profile's dynamic range - the difference between its $-\infty$ and $+\infty$ values, normalized by the $-\infty$ limit to  get an 'effective jump' measure;
\item The dielectric profile's proximity to a step function, essentially expressed by the size of the function's $z$-derivative at $z=0$ and the rate of its decay toward $\pm \infty$.
\end{itemize}
\end{comment}

To obtain a non-dimensional argument for $\epsilon$, we let $\epsilon(z) = \epsilon(z/M)$, where $M$ has units of length. \\
To get non-dimensional groups of parameters, we apply the change of variables $z= M s$ to equation (\ref{eqns:1-D-time-harmonic-E3_z}) and get
\begin{align}
\label{eqns:1-D-time-harmonic-z-normalized-boundary-source}
\dfrac{d^2\hat{E}_3}{ds^2} + \left( \dfrac{1}{\epsilon}\dfrac{d \epsilon}{ds}\right)\dfrac{d\hat{E}_3}{ds}
+ \left( \dfrac{1}{\epsilon}\dfrac{d^2\epsilon}{ds^2}
-\dfrac{1}{\epsilon^2}\left( \dfrac{d\epsilon}{ds}\right)^2 + \epsilon \mu \omega^2\right)\hat{E}_3 = 0
\end{align}

Let $\epsilon_1 = \epsilon(-L)$ and $\Delta \epsilon = \epsilon(L)-\epsilon(-L)$. These values can be interpolated across the interval $[-L,L]$ using any non-constant function. For our purposes we use a smooth, monotone function $c(s)$, to which we refer as the `cutoff function'. it term of $c(s)$, 
\begin{align}
\label{eqns:1-D-time-harmonic-s-dielectric-cutoff}
\epsilon(s) = \dfrac{\epsilon_1 ((r+1) c(-a)-c(a)-r c(s))}{c(-a)-c(a)}
\end{align}
where 
\begin{align}
r = \dfrac{\Delta \epsilon}{\epsilon_1}
\end{align} 
is a dimensionless parameter representing the layer's edge-to-edge difference of dielectric permeability, and
\begin{align}
a = \dfrac{L}{M}
\end{align} 
is a dimensionless parameter for the effective layer width.\\
Clearly, when $c(s)$ is monotone and differentiable, so is $\epsilon(s)$.\\


Next, we notice that the parameter grouping  
\begin{align}
\label{eqns:1_D-time-harmonic-normalized-freq}
\Omega = M \omega \sqrt{\epsilon_1 \mu}
\end{align}
is dimensionless, as $[\sqrt{\epsilon_1 \mu}] = \left[ \dfrac{1}{v}\right] = \dfrac{\text{time}}{\text{length}}$, while $[M] = \text{length}$ and $[\omega] = \dfrac{\text{rad}}{\text{time}}$. \\


Substituting (\ref{eqns:1-D-time-harmonic-s-dielectric-cutoff}) into (\ref{eqns:1-D-time-harmonic-z-normalized-boundary-source}), we get the dimensionless form 


\begin{align}
\label{eqns:1_D-time-harmonic-s-E_3-non-dim}
\dfrac{d^2\hat{E}_3}{ds^2}+c_1(s)\dfrac{d\hat{E}_3}{ds}+ c_0(s)\hat{E}_3(s) &=0 
\end{align}
where, letting $D = (r+1) c(-a)-c(a)-r c(s)$
\begin{subequations}
\begin{align}
\label{eqns:1_D-time-harmonic-s-E_3-non-dim-coef-1}
c_1(s) = -\dfrac{r c'(s) }{D}& \\
\label{eqns:1_D-time-harmonic-s-E_3-non-dim-coef-0}
c_0(s) = -\dfrac{r c''(s)}{D}&-\dfrac{r^2 c'(s)^2}{D^2} 
   +\dfrac{r \Omega ^2
   (c(s)-c(-a))}{c(a)-c(-a)}+\Omega ^2 
\end{align}
\end{subequations}
in which the coefficient are independent of dimensional parameters. In this grouping we also reduced the number of parameters to three.\\ 
The coefficients $c_1(s)$ and $c_0(s)$ are differentiable in the interval $-a<s<a$, as can be shown by finding the denominators zeros in $r$.
The boundary conditions (\ref{eqns:1-D-time-harmonic-z-boundary-source}),
 (\ref{eqns:1_D-time-harmonic-traveling-wave-bc}) become
\begin{subequations}
\begin{align}
\label{eqns:1-D-time-harmonic-z-boundary-source-normalized}
&\hat{E}_3(-a) = Ae^{i \phi} \\
\label{eqns:1_D-time-harmonic-traveling-wave-bc-normalized}
&\left[ i \Omega \sqrt{1+r} \hat{E}_3 +  \dfrac{d \hat{E}_3}{d s} \right]_{s = a}=0
\end{align}
\end{subequations}


Let us interpret the quantity $D$ and equations (\ref{eqns:1_D-time-harmonic-s-E_3-non-dim-coef-1}), (\ref{eqns:1_D-time-harmonic-s-E_3-non-dim-coef-0}) further. Suppose that $c(s)$ is a weighted combination of $c(-a)$ and $c(a)$, that is
\begin{align}
c(s) = (1-\alpha(s))c(-a) + \alpha(s) c(a) 
\end{align}
\label{eqns:1_D-time-harmonic-cuoff-convex-combination}
Specifically, if $c(s)$ is monotone, then $0 \leq \alpha(s) \leq 1$. \\
Substituting (\ref{eqns:1_D-time-harmonic-cuoff-convex-combination}) into (\ref{eqns:1_D-time-harmonic-s-E_3-non-dim-coef-1}), (\ref{eqns:1_D-time-harmonic-s-E_3-non-dim-coef-0}) we get
%Source: 2012-07-02-Matched-Asymptotics-non-oblique\2014-06-28-General-cutoff-\coefficients-convex-combinations.nb
\begin{align*}
D = -(c(a)-c(-a)) (1+r \alpha (s))
\end{align*}
The denominator grows linearly in absolute value with $\alpha(s)$, hence the coefficient $c_1(s)$ decreases as $\alpha(s)$ grows, while the effect on $c_0(s)$ is more complicated. Letting $R = c(a)-c(-a)$, 

\begin{subequations}
\begin{align}
c_1(s) &= \dfrac{r \left(c(-a)+R \alpha '(s)\right)}{R(r \alpha (s)+1)} \\
c_0(s)&= \Omega ^2  (1 + r \alpha(s)) + \\
&\dfrac{r \left(c(-a)+R \alpha '(s)\right) \left(-r c(-a)-r R
   \alpha '(s)+r R \alpha (s)+R\right)}{R^2(1+r  \alpha
   (s))^2}   \notag
\end{align}
\end{subequations}
From equation (\ref{eqns:1_D-time-harmonic-cuoff-convex-combination}), $\alpha(s)$ is a linear function of $c(s)$. Hence if $c'(s) \rightarrow 0$, as in the sigmoidal case, we get $\alpha'(s) \rightarrow 0$. 

Assuming a sigmoidal $c(s)$ we can deduce the following:
\begin{itemize}
\item When $a$ is large enough, the term $R \alpha'(s)$ can be neglected as $s \rightarrow a$. 
\item If, in addition, $r$ is large, and $\Omega^2$ is at least $O(1)$, we have 
\begin{align}
c_0(s)\approx \Omega ^2  (1 + r \alpha(s)), \qquad s \rightarrow a
\end{align}
and $\Big \vert \dfrac{c_1(s)}{c_0(s)} \Big \vert \ll 1 $, so that oscillation would be a `dominant' feature over attenuation. 
\end{itemize}
One goal of our research is to make the above qualitative description precise by perturbation expansions. 





\begin{comment}
Letting the effective layer width $a \ll 1$ and neglecting terms multiplied by it, $\epsilon(a s)$ becomes nearly constant. In terms of perturbation theory, the approximation problem in powers of $a$ is regular and equation (\ref{eqns:1-D-time-harmonic-z-normalized-boundary-source}) has a zero-order approximation by the constant coefficients equation
\begin{align}
\label{eqns:1-D-time-harmonic-z-normalized-boundary-source-regular-per-const-coef}
\dfrac{d^2 \hat{E}_3}{d s^2} +   L^2\epsilon(0) \mu \omega^2 \hat{E}_3 = 0 
\end{align}
On the other hand, $a \gg 1$ yields a singular perturbation problem because naive expansion in powers of $\dfrac{1}{a}$ would yield an algebraic equation as a zero-order approximation for equation (\ref{eqns:1-D-time-harmonic-z-normalized-boundary-source}), which is not compatible with the boundary conditions. In such case, a totally new approximation method is required, based on singular perturbation theory. 

Considering again the easier case $a \ll 1$, we wish to quantify the concept of a layer thickness. To do so, we must take into account the magnitudes of the absolute thickness $L$, the source frequency $\omega$ and the inverse square wave speed $\epsilon \mu = \dfrac{1}{v^2}$.

Since equation (\ref{eqns:1-D-time-harmonic-z-normalized-boundary-source-regular-per-const-coef}) has constant coefficients, we are justified in using relations between the wavelength $\lambda$ and $\omega$:
\begin{align*}
\omega = \dfrac{2\pi v}{\lambda}
\end{align*}
Therefore
\begin{align}
L^2 \epsilon \mu \omega^2 = \dfrac{L^2}{v^2} \cdot \dfrac{(2\pi v)^2}{\lambda^2} = 4\pi^2 \dfrac{L^2}{\lambda^2}
\end{align}
we can therefore expect a solution to be non-oscillatory within a layer when $L \ll \lambda$.\\
In this work, we are dealing with visible light optics, where the wavelength $\lambda \approx 0.5 \cdot 10^{-6}m$. In such case, layers of width $10-100 \mu m$, typical to the semiconductor industry, are considered thick, while biological cells membrane, which is $3-10 nm$ thick, may be considered narrow. 
\end{comment}

 

\chapter{The ODE coefficients and solutions for $c(s) = \tan^{-1}(s)$   }

\section{Construction of the ODE coefficients}
First let us construct a smooth, increasing cutoff function based on $\tan^{-1}(s)$. Substituting $\tan^{-1}(s)$ into (\ref{eqns:1-D-time-harmonic-s-dielectric-cutoff}), we get

\begin{align}
\label{eqns:epsilon_atan_cutoff}
\epsilon(s) = \dfrac{1}{2} \epsilon_1 \left(\dfrac{r \tan ^{-1}(s)}{\tan^{-1}(a)}+r+2\right)
\end{align}
The ODE (\ref{eqns:1-D-time-harmonic-E3_z}) has the general form (\ref{eqns:1_D-time-harmonic-s-E_3-non-dim}). Substituting (\ref{eqns:epsilon_atan_cutoff}) we now get the coefficients
\begin{subequations}
\begin{align}
c_1(s) &= \dfrac{r}{\left(s^2+1\right) \left((r+2) \tan ^{-1}(a)+r \tan ^{-1}(s)\right)} \label{eqns:1-D-time-harmonic-E3_s_Atan_coef_1_r}\\
c_0(s) = &\dfrac{1}{2} \Big(-\dfrac{2 r \left(2 (r+2) s \tan ^{-1}(a)+2 r s \tan
   ^{-1}(s)+r\right)}{\left(s^2+1\right)^2 \left((r+2) \tan ^{-1}(a)+r \tan
   ^{-1}(s)\right)^2}+\notag\\
   &\dfrac{r \Omega ^2 \tan ^{-1}(s)}{\tan ^{-1}(a)}+(r+2) \Omega
   ^2\Big)
 \label{eqns:1-D-time-harmonic-E3_s_Atan_coef_0_r}
\end{align}
\end{subequations}

As mentioned before, since the cutoff function is monotone and differentiable, so is $\epsilon(s)$. The coefficients $c_1(s)$ and $c_0(s)$ are differentiable for $-a<s<a$.
 
\section{Example for the perturbation expansion approach- asymptotic solution for $r \ll 1$  }
\label{sec:perExpansionSmall_r} 
We aim to establish an analytic approach for explaining solution shapes when the dimensionless parameters $r$, $a$ and $\Omega$ are either very small od very large. As a simple example, we consider the case $r \ll 1$, where $a$ and $\Omega$ are $O(1)$.\\
When $r \rightarrow 0+$, $c_1(s) \rightarrow 0+$ and $c_0(s)\rightarrow \Omega^2$ uniformly. That is intuitive because $r$ reflects the dielectric permittivity relative jump size; when the jump is zero, we get that equation (\ref{eqns:1-D-time-harmonic-z-normalized-boundary-source}) has constant coefficients. The first and second derivative terms $\epsilon'(z)$, $\epsilon''(z)$ in (\ref{eqns:1-D-time-harmonic-E3_z}) vanish.\\
In this case, regular asymptotic expansion easily yields the solution behavior. \\
We expand $\hat{E}_3 = u_0(s)+r u_1(s)+ \ldots$. The coefficients (\ref{eqns:1-D-time-harmonic-E3_s_Atan_coef_1_r}), (\ref{eqns:1-D-time-harmonic-E3_s_Atan_coef_0_r}) and the boundary conditions (\ref{eqns:1-D-time-harmonic-z-boundary-source-normalized}), (\ref{eqns:1_D-time-harmonic-traveling-wave-bc-normalized}) are expanded into power series in $r$. Equating like powers we get the zero order approximation of the problem



\begin{align}
\label{eqns:1-D-time-harmonic-E3_s_Atan_coefs-zero-order-per}
&u_0''+\Omega^2 u_0 = 0 \notag\\
u_0(-a) = A e^{i \phi}, &\qquad i \Omega u_0 + u_0'(a) = 0
\end{align} 
We thus expect the solution to be well approximated by an $\Omega$ frequency sine. This is indeed the case as shown in figure (\ref{fig:plotsAtanrSmall}).

\begin{figure}
% Source - PhD\Reports\2012-07-02-Matched-Asymptotics-non-oblique\2014-06-16-1-D-Maxwell-r-small.nb
\begin{center}
\includegraphics[width=1.0\textwidth]{plotsAtanrSmall}
\end{center}
\caption {Solution for $r$ small }.\\
Top left - numerical solution for (\ref{eqns:1_D-time-harmonic-s-E_3-non-dim}) with $\tan^{-1}(s)$-based coefficients (\ref{eqns:1-D-time-harmonic-E3_s_Atan_coef_1_r}), (\ref{eqns:1-D-time-harmonic-E3_s_Atan_coef_0_r}).\\
Top right - the zero-order perturbation solution for (\ref{eqns:1-D-time-harmonic-E3_s_Atan_coefs-zero-order-per})\\
Bottom left - the absolute value of the difference between the numerical and perturbation solutions.\\
Bottom right - numerical and perturbation solution plotted together.\\ 
Parameters set to $r=0.01$, $A=1$, $\phi = \pi/4$, $a=1$, $\Omega=10$.
\label{fig:plotsAtanrSmall}
\end{figure}


\begin{comment}

\subsubsection{Passage to a Sturm-form BVP}
Equation (\ref{eqns:1-D-time-harmonic-E3_z}) is a linear, homogeneous real valued variable coefficients second order ODE. We can estimate its solution's period of oscillation using Sturm's comparison theorem \cite{BirkhoffRota1989}:

\begin{thm}
\label{thm:Sturm-Comparison}
Let $f(x)$ and $g(x)$ be nontrivial solutions of the DEs $u''+p(x)u=0$ and $v''+q(x)v=0$, respectively., where $p(x) \geq q(x)$. Then $f(x)$ vanishes at least once between any two zeros of $g(x)$, unless $p(x) \equiv q(x)$ and if $f$ is a constant multiple of $g$. 
\end{thm}
As we can see, the theorem requires the equation's first order derivative coefficient to vanish. Thus equation (\ref{eqns:1-D-time-harmonic-E3_z}) must be transformed to such form. \\
Letting $\hat{E}_3 = \epsilon^{-1/2}u(z)$ we get
\begin{align}
\label{eqns:1-D-harmonic-z-Sturm}
\dfrac{d^2 u}{dz^2} +\dfrac{\omega^2 \epsilon^3(z)-\frac{3}{4}(\epsilon'(z))^2+\frac{1}{2}\epsilon(z)\epsilon''(z)}{\epsilon(z)}u=0
\end{align}
Since $\epsilon(z)$, $\epsilon'(z)$, $\epsilon''(z)$ are bounded and $\epsilon(z)$ is strictly positive, we can take $\omega^2$ large enough so that the right hand side of (\ref{eqns:1-D-harmonic-z-Sturm}) would be larger than $\dfrac{\omega^2}{2\epsilon_2}$ for all $-\delta<z<\delta$.In such case, the solutions for (\ref{eqns:1-D-harmonic-z-Sturm}) oscillate at a frequency of at least $\dfrac{\omega}{2\pi\sqrt{2 \epsilon_2}}$.




\end{comment}

\newpage



\section{Relating Coefficient asymptotic profiles to solution shapes}
The solution behavior presented in section (\ref{sec:perExpansionSmall_r}) is remarkably simple in the sense that when a single parameter is asymptotically eliminated we get a `well behaved' regular perturbation boundary value problem.\\
From an analytic solutions point of view, the `good' attributes of the small $r$ case are that both the ODE and the boundary conditions are well approximated by a constant coefficients quantities. \\
We wish to point out that this is not the general case. As our methodology is exploration of the problem's response to asymptotically small or large parameter values, we shall often entangle singular perturbation problems, problems where even the zeroth order approximation equations are variable-valued, and problems where the boundary conditions coefficients are large. \\
Such problems are harder to analyze, and their exhaustive analysis is one of the research goals. In this proposal we'll take an alternative approach: since we managed to write the ODE coefficients in terms of three dimensionless parameters, we can, in principle, fix each pair among the possible three as $O(1)$ quantities and let the third be small or large - a total of six possibilities. We can also consider the configurations where each parameter is either small or large, and get eight more possibilities. \\
Clearly, since the number of possibilities is large, we shall consider here only a few examples producing non-trivial solution behavior.\\
For each configuration analyzed we shall examine the coefficients $c_0(s)$, $c_1(s)$ shapes. Sometimes, as in the small $r$ case - but not always - there exists a clear interpretation of the solution behavior in terms of the coefficients shapes. We shall point these cases out and leave the rest of the cases for future work. 
\newpage

\subsection{The $c_1(s)$  coefficient profiles - asymptotics in $r$}

The case $r \ll 1$ was analyzed in section \ref{sec:perExpansionSmall_r}, where it was shown that $c_1(s) \approx 0$ as $r \rightarrow 0$.\\
The situation is somewhat more complex when $r \rightarrow \infty$, as we get
\begin{align}
\label{eqns:1-D-time-harmonic-E3_s_Atan_coef_1_r_r_inf}
c_1(s) = \dfrac{1}{\left(s^2+1\right) \left(\tan ^{-1}(a)+\tan ^{-1}(s)\right)}
\end{align}
Here $c_1(s)$ is large near $s=-a$. When $a$ is large, $c_1(s)$ is large near $s=0$ as well. See figures \ref{fig:c1Atan} for some plots of $c_1(s)$ with small / large but finite $r$ values, and \ref{fig:c1AtanRinf} for asymptotic profiles.

\begin{figure}
\begin{center}
\includegraphics[width=1.0\textwidth]{plotsCoef1Atan}
\end{center}
\caption {$c_1(s)$ profiles}.\\
Top left: $a=0.01$, $r=0.01$; top right: $a=0.01$, $r=50$; bottom left: $a=10$, $r=50$; bottom right: $a=70$, $r=50$.


\label{fig:c1Atan}
\end{figure}

\begin{figure}
\begin{center}
\includegraphics[width=1.0\textwidth]{plotsCoef1AtanRinf}
\end{center}
\caption {$c_1(s)$ asymptotic profiles - $r \rightarrow \infty$}.\\
Left: $a=1$, right: $a=20$.


\label{fig:c1AtanRinf}
\end{figure}




\begin{comment}
\subsection{The $c_1(s)$ coefficient profiles - asymptotics in $a$  }
When $a \rightarrow 0+$, we have 
\begin{align}
\label{eqns:1-D-time-harmonic-E3_s_Atan_coef_1_non_dim_a_0}
c_1(s) = \dfrac{r}{r s+r+2}
\end{align}
The numerator vanishes at $s = -1-\dfrac{2}{r}<-1$. Hence $c_1(s)\vert_{a=0}$ is well defined for all $-1 \leq s \leq 1$. Yet, when $r$ is large, the discontinuity point approaches $s=-1$, and we get large values of $c_1(s)$ near $s = -1$. \\
Finally, when $a \rightarrow \infty$, $c_1(s) \rightarrow 0$ uniformly.
\end{comment}



\subsection{The $c_0(s)$ coefficient profiles - asymptotics in $a$  } 

\begin{enumerate}
\item $a$ large: 
\begin{align}
\label{eqns:1-D-time-harmonic-E3_s_Atan_coef_0_non_dim_a_inf}
	c_0(s) \approx \dfrac{1}{2} \Omega ^2 (r (1+\text{sgn}(s))+2) = \begin{array}{cc}
	\Big \{ & 
	\begin{array}{cc}
	 \Omega ^2 & s<0 \\
 	(r+1) \Omega ^2 & s \geq 0 \\
	\end{array}
 	\\
	\end{array}
\end{align}
	
\begin{figure}
\begin{center}
\includegraphics[width=0.8\textwidth]{plotsC0AtanAAsymptotics}
\end{center}
\caption {$c_0(s)$ asymptotic profiles - limits in $a$}.\\
Top: $a \rightarrow \infty$; bottom: $a \rightarrow 0$.\\
Bottom left: $r=0.01$, $\Omega = 1$; bottom right: $r=50$, $\Omega = 1$.
\label{fig:c0AtanAlim}
\end{figure}


An interpretation for this result may be that when the layer's effective width is large, jump size tends to dominate the smoothness of the cutoff function. The jump size is $r \Omega^2$.
\item $a$ small:
\begin{align}
c_0(s) \approx \dfrac{1}{2} \Omega ^2 (r s+r+2)-\dfrac{r^2}{(r s+r +2)^2}
\end{align}
$c_0(s)$ can be shown to be increasing for $-1\leq s \leq 1$.  When $r$ is large, $c_0(s)$'s discontinuity point approaches	$s=-1$ from the left.
\end{enumerate}
\begin{comment}
	calculating the $s$-derivative,
	\begin{align*}
	c_0'(s) = \dfrac{2 r^3}{(r s+r+2)^3}+\dfrac{r \Omega ^2}{2} 
	\end{align*}
\end{comment}
See figure \ref{fig:c0AtanAlim} for both cases.



\subsection{The $c_0(s)$ coefficient profiles - asymptotics in $r$  } 

\begin{enumerate}
\item $r$ large:\\
Calculating the limit of $\dfrac{c_0(s)}{r}$, $r \rightarrow \infty$, and multiplying again by $r$ we get the estimate
	\begin{align}
	c_0(s) \approx \dfrac{r\Omega ^2 \left(\tan ^{-1}(a s)+\tan ^{-1}(a)\right)}{2 \tan ^{-1}(a)}
	\end{align}
	When $a$ is large, $c_0(s)$ tends to the step function (\ref{eqns:1-D-time-harmonic-E3_s_Atan_coef_0_non_dim_a_inf}). When $a$ is small, $\tan^{-1}(a) \approx a$, $\tan^{-1}(as) \approx as$, and $c_0(s)$ tends to the linear function 
	\begin{align}
	c_0(s) \approx \dfrac{1}{2} r (s+1) \Omega ^2
	\end{align}
See figure \ref{fig:c0AtanRlim}.	
\begin{figure} 
\begin{center}
\includegraphics[width=0.8\textwidth]{plotsC0AtanRlargeAsymptotics}
\end{center}
\caption {$\dfrac{c_0(s)}{r}$ asymptotic profiles - $r \rightarrow \infty$}.\\
Left: $a = 0.1$, $\Omega = 50$; Right: $a = 50$, $\Omega=50$.
\label{fig:c0AtanRlim}
\end{figure}



\item The small $r$ case was treated earlier. 
\end{enumerate}



\subsection{The $c_0(s)$ coefficient profiles - asymptotics in $\Omega$  } 

\begin{enumerate}
\item $\Omega$ large:\\
	In that case, 
	\begin{align}
	c_0(s) \approx \Omega ^2 \left(\dfrac{r \left(\tan ^{-1}(a s)+\tan ^{-1}(a)\right)}{2 \tan ^{-1}(a)}+1\right)
	\end{align}
	When $a$ is small, we can replace $\tan^{-1}(a)$ by $a$ and $\tan^{-1}(as)$ by $as$, so that $c_0(s)$ tends to the linear function
	\begin{align}
	c_0(s) \approx \Omega ^2\left( \dfrac{1}{2} r (s+1) + 1 \right)
	\end{align}
	interpolating $\Omega^2$ at $s=-1$ and  and $\Omega^2(r+1)$ at $s=1$.
\begin{figure} 
\begin{center}
\includegraphics[width=0.8\textwidth]{plotsC0AtanOmegaAsymptotics}
\end{center}
\caption {$c_0(s)$ asymptotic profiles - $\Omega \rightarrow 0$, $a = 10$, $r = 1$}.

\label{fig:c0AtanOmegalim}
\end{figure}
	
\item $\Omega$ small:\\
Here, 
	\begin{align}
	\label{eqns:1-D-time-harmonic-E3_s_Atan_coef_0_non_dim_Omega_zero}
	c_0(s) \approx &-\dfrac{a^2 r^2}{\left(a^2 s^2+1\right)^2 \left(r \tan ^{-1}(a s)+(r+2)
   \tan ^{-1}(a)\right)^2} \notag \\
   &-\dfrac{2 a^3 r s}{\left(a^2 s^2+1\right)^2
   \left(r \tan ^{-1}(a s)+(r+2) \tan ^{-1}(a)\right)}
	\end{align}
When $r$ is large and $a$ is kept $O(1)$, we get a monotone profile with large and negative values near $s=-1$, as demonstrated in the earlier. \\
	When $r$ is kept $O(1)$ and $a$ is large, we get a different phenomenon. The dominant term in (\ref{eqns:1-D-time-harmonic-E3_s_Atan_coef_0_non_dim_Omega_zero}) is the second one, which can be thought of as a shifted and scaled version of the function $-\dfrac{s}{(s^2 +1)^2}$. This is an odd function, fast decaying in $s$ and has to peaks, symmetric with respect to the origin - maximum and minimum. For small $s$ values, the denominator's $O(a^4)$ growth is suppressed, and the numerator's $O(a^3)$ a $\delta'(s)$-like behavior. See figure \ref{fig:c0AtanOmegalim}.
	



\end{enumerate}



\newpage

\section{Simulations for $\tan^{-1}(s)$ cutoff}

Each of the following sequence of figures consists of 4 parts: plots of $c_0(s)$ (\ref{eqns:1-D-time-harmonic-E3_s_Atan_coef_1_r}) and $c_1(s)$ (\ref{eqns:1-D-time-harmonic-E3_s_Atan_coef_0_r}) at the top, a figure of $c_0(s)$ and $c_1(s)$ combined and equation (\ref{eqns:1_D-time-harmonic-s-E_3-non-dim}) solution with these coefficients solution at the bottom. In all cases, the left boundary condition (\ref{eqns:1-D-time-harmonic-z-boundary-source-normalized}) has $A=1$ and $\phi=\dfrac{\pi}{4}$.  

\begin{figure}[!htb] 
\begin{center}
\includegraphics[width=0.8\textwidth]{plotsCoefsSols1}
\end{center}
\caption {Coefficients and solution - $a = 0.1$, $r = 0.1$, $\Omega = 0.1$}.

\label{fig:atanCoefsPlots1}
\end{figure}

\begin{figure} 
\begin{center}
\includegraphics[width=0.8\textwidth]{plotsCoefsSols2}
\end{center}
\caption {Coefficients and solution - $a = 0.1$, $r = 10$, $\Omega = 0.1$}.

\label{fig:atanCoefsPlots2}
\end{figure}

\begin{figure} 
\begin{center}
\includegraphics[width=0.8\textwidth]{plotsCoefsSols3}
\end{center}
\caption {Coefficients and solution - $a = 0.1$, $r = 0.1$, $\Omega = 10$}.

\label{fig:atanCoefsPlots3}
\end{figure}

\begin{figure} 
\begin{center}
\includegraphics[width=0.8\textwidth]{plotsCoefsSols4}
\end{center}
\caption {Coefficients and solution - $a = 0.1$, $r = 10$, $\Omega = 10$}.

\label{fig:atanCoefsPlots4}
\end{figure}

\begin{figure} 
\begin{center}
\includegraphics[width=0.8\textwidth]{plotsCoefsSols5}
\end{center}
\caption {Coefficients and solution - $a = 10$, $r = 0.1$, $\Omega = 0.1$}.

\label{fig:atanCoefsPlots5}
\end{figure}

\begin{figure} 
\begin{center}
\includegraphics[width=0.8\textwidth]{plotsCoefsSols6}
\end{center}
\caption {Coefficients and solution - $a = 10$, $r = 10$, $\Omega = 0.1$}.

\label{fig:atanCoefsPlots6}
\end{figure}

\begin{figure} 
\begin{center}
\includegraphics[width=0.8\textwidth]{plotsCoefsSols7}
\end{center}
\caption {Coefficients and solution - $a = 10$, $r = 0.1$, $\Omega = 10$}.

\label{fig:atanCoefsPlots7}
\end{figure}

\begin{figure} 
\begin{center}
\includegraphics[width=0.8\textwidth]{plotsCoefsSols8}
\end{center}
\caption {Coefficients and solution - $a = 10$, $r = 10$, $\Omega = 10$}.

\label{fig:atanCoefsPlots8}
\end{figure}



\newpage
%\subsubsection{Simulations for $\tanh(z)$ cutoff}

\chapter{Parameter estimation for 1-D boundary value problems}
The inverse problem we wish to solve is the estimation of the parameters $r$, $a$, $\Omega$, $A$, $\phi$ from a solution of equation (\ref{eqns:1-D-time-harmonic-z-normalized-boundary-source}) with the cutoff function $c(s) = \tan^{-1}(s)$ subject to the boundary conditions (\ref{eqns:1-D-time-harmonic-z-boundary-source-normalized}), (\ref{eqns:1_D-time-harmonic-traveling-wave-bc-normalized}). This is a complex project since it involves solving systems of nonlinear equations, where the target parameters may have different scales.\\
It is thus instructive to first analyze the recovery of parameters from  a reasonably simple 'toy model'. We use this procedure to demonstrate the difficulties we may entangle, resulting in general properties of 1-D boundary value problems.

\section{Uniqueness of the solution}

We consider a problem formulated as follows:\\
Let $\overrightarrow{\lambda}, \overrightarrow{\lambda}_0 \in \mathbb{R}^d$ be elements of a parameter space, and
\begin{subequations}
\begin{align}
\label{eqns:1-D-BVP-general}
y''(x) + P(x, \overrightarrow{\lambda}_0)y'(x) + Q(x, \overrightarrow{\lambda}_0)y(x) =0 \\
\label{eqns:1-D-BC-travelWave}
y(-a) = Ae^{i\phi}, \qquad i \Omega y(a) + y'(a)=0
\end{align}
\end{subequations}
is a 1-D boundary value problem, where $P(x, \overrightarrow{\lambda})$, $Q(x, \overrightarrow{\lambda})$ are known, smooth function of the space variable $x$ and of the parameters $\lambda_1, \ldots \lambda_n$. $\Omega$ is s real number.\\

\begin{lem}
The solution for the problem (\ref{eqns:1-D-BVP-general}), (\ref{eqns:1-D-BC-travelWave}) is unique for all $\overrightarrow{\lambda} \in \mathbb{R}^d $ and $\Omega \neq 0$.
\end{lem}
\begin{proof}
It is sufficient to show that $y \equiv 0$ is the only solution of (\ref{eqns:1-D-BVP-general}), (\ref{eqns:1-D-BC-travelWave}) in the homogeneous case $A=0$.\\
Let $y_1$, $y_2$ be real, independent solutions of the ODE (\ref{eqns:1-D-BVP-general}) satisfying $y_1(-a)=1$, $y_1'(-a)=0$, $y_2(-a)=0, y_2'(-a)=1$. Then the homogeneous solution $y$ satisfies $y=C_2 y_2$.\\
If $C_2 = 0$ we are done. Otherwise, at the left boundary $a$ we have
\begin{align*}
i\Omega y_2(a) + y_2'(a) = 0 
\end{align*}
but $\Omega$, $y_2(a)$ and $y_2'(a)$ are real, and $\Omega \neq 0$. Hence $y_2(a) = 0$, $y_2'(a)=0$. By the uniqueness theorem for linear ODEs with continuous coefficients, $y_2 \equiv 0$ - a contradiction. Hence $C_2$ must vanish and $y \equiv 0$, as required. 
\end{proof}


\section{The nonlinear regression approach for the inverse problem}
We wish to recover $\overrightarrow{\lambda}_0$ from sampled points $y_i$ of the solution for the boundary values problem (\ref{eqns:1-D-BVP-general}).\\
To do so, we define the global error
\begin{align}
\label{eqns:1-D-L2-error}
Err(\overrightarrow{\lambda}) = \dfrac{1}{2}\sum_{\mathclap{i=-n+1}}^{n-1} \Big \vert D_2 y_i + P(x_i, \overrightarrow{\lambda})D_1 y_i + Q(x_i, \overrightarrow{\lambda})y_i \Big \vert ^2
\end{align}
where $D_1 y_i = \dfrac{y_{i+1}-y_{i-1}}{2h}$, $D_2 y_i= \dfrac{y_{i+1}-2y_i +y_{i-1}}{h^2}$ and $h = \dfrac{a}{n}$. Note that the absolute value is in the complex sense.\\
Since $y_i$ are exact values of the solution at the points $x_i$, the functional $Err$ has a minimizing argument $\overrightarrow{\lambda}_0$ at which it vanishes. As a local minimum, $\overrightarrow{\lambda}_0$ is characterized by 
\begin{align}
\dfrac{\partial}{\partial \lambda_j} Err(\overrightarrow{\lambda}_0) = 0, \qquad 1 \leq j \leq d
\end{align}
We thus get a set of $d$ nonlinear equations in the $\overrightarrow{\lambda}$ components
\begin{align}
\label{eqns:nonLinear-sys-1-D-BVP-inverse}
0=\sum_{\mathclap{i=-n+1}}^{n-1}\Re \Big(\left(D_2 y_i + P(x_i, \overrightarrow{\lambda})D_1 y_i + Q(x_i, \overrightarrow{\lambda})y_i \right)\times\notag\\ 
\left( \dfrac{\partial P(x_i, \overrightarrow{\lambda})}{\partial \lambda_j}D_1y_i^*  + \dfrac{\partial Q(x_i, \overrightarrow{\lambda})}{\partial \lambda_j}y_i^*\right)\Big)
\end{align}
where $\Re(\cdot)$ is the real part of a complex number, and $(\cdot)^*$ is the complex conjugate.\\
Suppose that we have an initial guess for $\overrightarrow{\lambda}_0$. Then we can apply iterative schemes for solution of nonlinear equations to the system (\ref{eqns:nonLinear-sys-1-D-BVP-inverse}). Such schemes are based on root-finding methods applied to the error function's gradient. \\

Let us consider one such problem. Suppose that $a=1$, $\Omega=1$, $A=1$ and $\phi=\pi/4$ are known. We generate the values of $y_i$ numerically applying $r=10$, but using a computer algebra system we can write the error function (\ref{eqns:1-D-L2-error}) parametrically, as an $r$-dependent expression. That enables us to differentiate $Err$ directly in $r$. As can be seen in figure , The derivative vanishes exactly  at $r=10$, corresponding to the $r$-dependent minimum error.  

\begin{figure}[!htb] 
\begin{center}
\includegraphics[width=0.8\textwidth]{plots1-D-Inverse-error}
\end{center}
\caption{Top - the $r$-dependent error function, bottom - the error function's $r$-derivative, vanishing at $r=10$.}


\label{fig:inverseR-1-D-err-and-analytic-dErr}
\end{figure}

\chapter{Future work}
\section{Asymptotic analysis}
Our first goal is to derive uniform asymptotic expansions for all the cases presented in the 1-D case, so that the forward problem solution features would be predicted analytically from magnitude statements. We shall start with the $\tan^{-1}(z)$-based cutoff and try to obtain a general theory for smooth, monotone cutoff functions.  \\
Second, we shall apply asymptotic expansions of integrals to the inverse temporal Fourier transform of the 1-D Helmholtz equation for high frequencies to get insight about the behavior of the forward problem solutions when the wave number is large compared to both layer width and dielectric interface jump. Again, we shall start with the $\tan^{-1}(z)$-based cutoff solutions  and aim to generalize. \\
With experience gained from the degenerate $z$-component equation, we shall expand solutions for the non-degenerate case, with two and three dimensional Laplacians to be accounted for. Due to the complexity of the problem, we shall restrict ourselves to simple, slab-like geometries. With these expansions we shall repeat the feature acquisition process. Time permitting, we may try expanding the full Maxwell equations system with various cutoff functions.

\section{Numerical solvers}
As mentioned before, a good way of validating asymptotic solutions is numerical simulation. In order to quantify the solution's quality, relatively-simple numerical solvers will be written so that numerical error may be explicitly estimated and taken into account in the comparison procedure.

\section{Resolution of inversion methods} 
We wish to use our detailed dimensional analysis to account for continuity, Fr\'echet differentiability and error estimates for inverse mappings. Our aim is to answer questions about the effective size of structured media features that may be reconstructed by a certain inversion method. Thus we shall equip the inversion procedure with a global sense of resolution relative to incident wave number.\\
This measure would be effective especially in extreme cases. Typical cases would be:
\begin{itemize}
\item Media having either very small or very large widths compared to the wave number;
\item Media having either very small or very large refractive index jump across and interface compared to the wave number;
\item Combinations of both above cases.
\end{itemize} 
The reference object size would be measured in terms of a dimensionless wave number.




\begin{comment}
\section{Illustration of parameter estimation using the regression technique for a 1-D harmonic oscillator Dirichlet problem without friction}
Parameter estimation for problem (\ref{eqns:1_D-time-harmonic-s-E_3-non-dim}) with the coefficients (\ref{eqns:1-D-time-harmonic-E3_s_Atan_coef_1_r}), (\ref{eqns:1-D-time-harmonic-E3_s_Atan_coef_0_r}) and boundary conditions (\ref{eqns:1-D-time-harmonic-z-boundary-source-normalized}), (\ref{eqns:1_D-time-harmonic-traveling-wave-bc-normalized}) is quite involved computationally. In this proposal we shall illustrate the process for the problem

\begin{align}
y''(x) + \omega^2 y(x) =0 \notag\\
y(-a) =A, \qquad y(a) = 0
\end{align}

The main reason for using this problem is that its eigenvalues and eigenfunctions can be easily calculated so that we can readily see the effect of the ill-posedness on the parameter estimation stability.\\
The problem's eigenvalues are $\kappa_k = \dfrac{\pi k}{2}$, $k \in \mathbb{Z}$. \\

\subsection{Relating accuracy to number of sampling points}
We demonstrate the error in recovering $\omega$ when $a$, $A$ are known. Let $\omega = 1$, $a=1$, $A = 1$. Figure \ref{fig:inverseOmega1DvarySPnum} shows the error in estimating $\omega$ as a depending on the number of sampling points in $\log_{10}$ scale. As can be seen from the plot, increase of one order of magnitude in sampling points reduces error in two orders of magnitude.

\begin{figure}[!htb] 
\begin{center}
\includegraphics[width=0.8\textwidth]{plotsInverseOmega1DvarySPnum}
\end{center}

\caption {Error in recovering $\omega = 1$ from the Dirichlet harmonic solution with varying number of sampling points.The error $\omega_{\text{err}} = |\omega_{\text{rec}}-1|$} where $\omega_{\text{rec}}$ is the recovered value of $\omega$. The parameters $a=1$, $A=1$ are given.\\ 

\label{fig:inverseOmega1DvarySPnum}
\end{figure}

\subsection{Relating accuracy to BVP interval length}
Keeping the number of sampling points fixed, one should expect a deterioration of parameter estimation accuracy as the interval length $a$ increases.\\
This rate of deterioration is demonstrated in figure \ref{fig:inverseOmega1DvaryLength}. As can be seen, oncrease of one order of magnitude in length causes two orders deterioration of two orders of magnitude in the recovered $\omega$ accuracy. 
It thus seems reasonable that keeping a fixed proportion of sampling points number and interval length should preserve parameter recovery accuracy, up to some tolerance.




\begin{figure}[!htb] 
\begin{center}
\includegraphics[width=0.8\textwidth]{plotsInverseOmega1DvaryLength}
\label{fig:inverseOmega1DvaryLength}
\end{center}

\caption {Error in recovering $\omega = 1$ from the Dirichlet harmonic solution with varying interval length.The error $\omega_{\text{err}} = |\omega_{\text{rec}}-1|$} where $\omega_{\text{rec}}$ is the recovered value of $\omega$. The parameter $A=1$ is given, and the number of sampling points is fixed to 100. 

\label{fig:inverseOmega1DvaryLength}
\end{figure}


\subsection{Relating accuracy to left boundary condition amplitude}

In this case we increase the left boundary condition amplitude $A$ by 11 orders of magnitude, from $10^{5}$ to $10^5$. All other parameters are kept fixed. As can be seen in figure \ref{fig:inverseOmega1DvaryLeftBCamp}, the amplitude variation has no effect on the $\omega$ recovery accuracy. This result bocomes clear when we look at the $\omega$ recovery formula

\begin{align}
\label{eqns:inverseOmegaSquare1-D-regression}
\omega^2 = -\dfrac{n^2 \sum _{i=1-n}^{n-1} y_i (y_{i-1}-2
   y_i+y_{i+1})}{a^2 \sum _{i=1-n}^{n-1} y_i^2}
\end{align}
Notice that the solution $y(x)$, and hence the samples $y(i)$  depend linearly on $A$. Since both the numerator and the denominator are pure quadratic in $y_i$, the expression (\ref{eqns:inverseOmegaSquare1-D-regression}) is independent of $A$. 


\begin{figure}[!htb] 
\begin{center}
\includegraphics[width=0.8\textwidth]{plotsInverseOmega1DvaryLeftBCamp}
\label{fig:inverseOmega1DvaryLeftBCamp}
\end{center}
\end{figure}

\subsection{Recovering $\omega$ near an eigenvalue}

Here we show the effect of the numerical instability of the forward problem near an eigenvalue on the parameter estimation process. As can be seen in figure \ref{fig:inverseOmega1DvaryOmegaNearEig}, the calculated $\omega$ is up to three orders of magnitude larger than the true value.\\
This result demonstrates the need to assess any BVP eigenvalues before the parameter estimation process is applied, as the results would be meaningless near these values. 

\begin{figure}[!htb] 
\begin{center}
\includegraphics[width=0.8\textwidth]{plotsInverseOmega1DvaryOmegaNearEig}
\label{fig:inverseOmega1DvaryOmegaNearEig}
\end{center}
\end{figure}


\end{comment}




 


\begin{comment}
plotsInverseOmega1DvarySPnum

\subsection{The one-dimensional case in source-free, non-magnetic media}
$\mathbf{Banergee - J_c and \rho as source}$, Griffiths - Gauss's law

Neglecting magnetic phenomena we have $\mu = 1$. We also assume no free currents - $\overrightarrow{J} = 0$, no charge accumulation - $\rho = 0$, and that $\epsilon$ is $x$ - dependent - $\epsilon = \epsilon(x)$. We get the $\overrightarrow{E} , \overrightarrow{H}$ representation 

\begin{subequations}
	\begin{align}
        \nabla \times \overrightarrow{E} &= -\dfrac{\partial \overrightarrow{H}}{\partial t}  \label{eqns:ourCurlE_Maxwell}\\
		\nabla \times \overrightarrow{H} &=  \epsilon(x)\dfrac{\partial  \overrightarrow{E}}{\partial t} \label{eqns:ourCurlH_Maxwell}	\\	
		\nabla \cdot ( \epsilon(x) \overrightarrow{E} ) &= 0	\\		
		\nabla \cdot \overrightarrow{H} &= 0	
	\end{align}
	\label{eqns:ourCurlMaxell}
\end{subequations}

Applying the vector calculus identity (\ref{eqns:CurlCurlID}) and the $\overrightarrow{H}$ - curl equation (\ref{eqns:ourCurlH_Maxwell}) we get
\begin{align*}
\nabla (\nabla \cdot \overrightarrow{E}) - \nabla^2\overrightarrow{E} + \epsilon(x) \dfrac{\partial^2 \overrightarrow{E}}{\partial t^2} = 0
\end{align*}

Next, we use the assumption $\rho = 0$ in equations (\ref{eqns:DivD_Maxwell}), (\ref{eqns:constitutiveDepsE}) combined. We get
\begin{align*}
\nabla \cdot (\epsilon(x) \overrightarrow{E}) = 0
\end{align*}
Expanding using the vector calculus formula (\ref{eqns:DivProdID}) we get
\begin{align*}
\epsilon(x) \nabla \cdot \overrightarrow{E} + \overrightarrow{E} \epsilon'(x) = 0
\end{align*}
Thus 
\begin{align*}
\nabla \cdot \overrightarrow{E} = -\overrightarrow{E} \cdot \dfrac{\epsilon'(x)}{\epsilon(x)}
\end{align*}





\section{Inverse problem for the 1-D one-way wave equation}

Hyperbolic PDE systems of the form (\ref{eqns:ourMaxwellEyHzPolarized}) and (\ref{eqns:ourMaxwellEzHyPolarized}) can be transformed into coupled 1-D one way wave equations
\begin{align*}
u^j_t+a^j(x)u^j_x=0, \qquad j=1,2
\end{align*}
where coupling occurs at the boundary. It therefore makes sense use parameter identification problems for END HERE 
\begin{align*}
\begin{cases}
u_t + a(x) u_x &= 0,	\qquad -L<x< \infty \\
u(x,0) &= 0	\\
u(-L,t) &= g(t)
\end{cases}
\end{align*}
Most of the time we shall be interested in a boundary condition of the form 
\begin{align}
g(t) = A e^{i \left(\omega t + \phi \right)}
\label{eqns:complexExpBC}
\end{align}
When $a(x)=a$ is constant, we have a traveling wave solution 
\begin{align*}
u(x,t) = u(x-at)
\end{align*}
as can be shown by direct differentiation. Applying the initial condition and the left boundary condition we get

\begin{align*}
u(x,t) = g\left( t-\dfrac{x+L}{a} \right)
\end{align*}
and for the boundary condition (\ref{eqns:complexExpBC}), we get
\begin{align}
u(x,t) =A e^{i \left(\omega  \left(t-\frac{x+L}{a}\right)+\phi \right)}
\end{align}

Now assume that
\end{comment} 


\begin{appendices}
\chapter{Derivation of the generalized wave equations from Maxwell's equations}
\label{append:GeneralizedWaveDeriv}
In order to transform Maxwell's equation into a system of wave equations, we need the following set of vector calculus identities:
\begin{subequations}
\label{eqns:vectorCalcIDs}
	\begin{align}
			\nabla \times (\psi \overrightarrow{a}) = \nabla \psi \times 	
			\overrightarrow{a} + \psi \nabla \times \overrightarrow{a}  \label{eqns:ProdCurlID}\\
		\nabla \times \nabla \times \overrightarrow{a} = \nabla ( \nabla
			 \cdot \overrightarrow{a} ) - \nabla^2\overrightarrow{a} \label{eqns:CurlCurlID}\\
		\nabla \cdot (\psi \overrightarrow{a}) = \psi \nabla \cdot \overrightarrow{a}
			 + \overrightarrow{a} \cdot \nabla \psi \label{eqns:DivProdID}
			\end{align}
\end{subequations}

We start from equation (\ref{eqns:CurlE_Maxwell})
\begin{align*}
\nabla \times \overrightarrow{E} + \dfrac{\partial \overrightarrow{B}}{\partial t}&= 0
\end{align*}
Substituting the relation (\ref{eqns:constitutiveBH}) between magnetic fields in conjunction we get
\begin{align*}
\nabla \times \overrightarrow{E} + \mu \dfrac{\partial  \overrightarrow{H}}{\partial t}&= 0
\end{align*}
For future purposes, let us divide by $\mu$:
\begin{align*}
\dfrac{1}{\mu} \nabla \times \overrightarrow{E} + \dfrac{\partial \overrightarrow{H}}{\partial t} = 0
\end{align*}

Taking the curl of both sides and changing the order of temporal and spatial differentiation
\begin{align}
\label{eqns:CulCurlMaxwellModified}
\nabla \times \left( \dfrac{1}{\mu} \nabla \times \overrightarrow{E} \right)  + \dfrac{\partial (\nabla \times  \overrightarrow{H})}{\partial t} = 0
\end{align}
We'll apply thew vector calculus identities (\ref{eqns:vectorCalcIDs}) to decompose the summed terms in (\ref{eqns:CulCurlMaxwellModified}).\\
Using identity (\ref{eqns:ProdCurlID}) we get
\begin{align}
\label{eqns:CurlProdCurlpart}
\nabla \times \left( \dfrac{1}{\mu} \nabla \times \overrightarrow{E} \right) = \nabla \left( \dfrac{1}{\mu} \right) \times \left(\nabla \times \overrightarrow{E} \right) + \dfrac{1}{\mu} \nabla \times \nabla \times \overrightarrow{E}
\end{align}
Now we apply (\ref{eqns:CurlCurlID}) to (\ref{eqns:CurlProdCurlpart}) and get
\begin{align}
\label{eqns:CurlProdCurlExpanded}
\nabla \times \left( \dfrac{1}{\mu} \nabla \times \overrightarrow{E} \right) = \nabla \left( \dfrac{1}{\mu} \right) \times \left(\nabla \times \overrightarrow{E} \right) + \dfrac{1}{\mu} \left( \nabla (\nabla \cdot \overrightarrow{E}) - \nabla^2 \overrightarrow{E}\right)
\end{align}
The $\overrightarrow{H}$ - term in (\ref{eqns:CulCurlMaxwellModified}) can be replaced using equation (\ref{eqns:CurlH_Maxwell}):
\begin{align}
\label{eqns:Curl_H_replaced_D}
\dfrac{\partial}{\partial t} \left( \nabla \times \overrightarrow{H}\right) = \dfrac{\partial^2 \overrightarrow{D}}{\partial t^2} + \dfrac{\partial \overrightarrow{J}}{\partial t}
\end{align}

and using the constitutive relation (\ref{eqns:constitutiveDepsE}), equation (\ref{eqns:Curl_H_replaced_D}) becomes
\begin{align}
\label{eqns:Curl_H_replaced_E}
\dfrac{\partial}{\partial t} \left( \nabla \times \overrightarrow{H}\right) = \epsilon\dfrac{\partial^2 \overrightarrow{E}}{\partial t^2} + \dfrac{\partial \overrightarrow{J}}{\partial t}
\end{align}

Substituting (\ref{eqns:CurlProdCurlExpanded}), (\ref{eqns:Curl_H_replaced_E}) into (\ref{eqns:CulCurlMaxwellModified}) and multiplying by we get $\mu$ we get the generalized wave equations system (\ref{eqns:E_Wave_full}) stated in sction \ref{sec:TransMaxwellToWave}. 
\begin{align}
\mu \nabla \left( \dfrac{1}{\mu} \right) \times \left(\nabla \times \overrightarrow{E} \right) +  \nabla (\nabla \cdot \overrightarrow{E}) - \nabla^2 \overrightarrow{E} + \mu \epsilon\dfrac{\partial^2 \overrightarrow{E}}{\partial t^2} + \mu \dfrac{\partial \overrightarrow{J}}{\partial t} = 0
\end{align}
%which can be put in the form 
%\begin{align}
%\label{eqns:E_Wave_full_compact}
% \nabla \left( \log (\mu) \right) \times \left(\nabla \times \overrightarrow{E} \right) +  \nabla (\nabla \cdot \overrightarrow{E}) - \nabla^2 \overrightarrow{E} + \mu \epsilon \dfrac{\partial^2 \overrightarrow{E}}{\partial t^2} + \mu \dfrac{\partial \overrightarrow{J}}{\partial t} = 0
%\end{align}

\chapter{Component-wise expansion of the term $\left(\mu \nabla \left( \dfrac{1}{\mu}\right) \times \left(\nabla \times \vec{E} \right) \right)$}
\label{append:GradLogTermComponenetwiseDeriv}
To handle the repeated cross product we'll use Levi-Civita's permutation symbols $\epsilon_{ijk}$ \cite{ArfkenWeber2005} defined by
\begin{align}
\label{eqns:LeviCivitaSymbol}
\epsilon_{123} = \epsilon_{231} = \epsilon_{312} & = 1	\notag\\
\epsilon_{132} = \epsilon_{213} = \epsilon_{321} &= -1 \\
\text{all other } \epsilon_{ijk} &= 0	\notag
\end{align}
The following lemma is presented as an exercise in \cite{ArfkenWeber2005}, p. 150.
\begin{lem}
\label{lem:LiviCivitaEpsDelta}
Let $\epsilon_{ijk}$ be defined by (\ref{eqns:LeviCivitaSymbol}). Then
\begin{align}
\label{eqns:LiviCivitaEpsilonDelta} 
\epsilon_{ijk} \epsilon_{pqk} = \delta_{ip} \delta_{jq} - \delta_{iq} \delta_{jp}
\end{align} 
where $\delta_{ij}$ is Kronecker's delta.
\end{lem}

\begin{proof}
Noticed that each term in the $k$-sum $\epsilon_{ijk} \epsilon_{pqk}$ is non-vanishing only if $i,j,p,q \neq k$, $i \neq j$, $p \neq q$. For example, for the $k=1$ term we need only to examine the possibilities 
\begin{align*}
i=2,\quad j=3, \quad p=2, \quad q=3	\\
i=3,\quad j=2, \quad p=2, \quad q=3	\\
i=2,\quad j=3, \quad p=3, \quad q=2	\\
i=3,\quad j=2, \quad p=3, \quad q=2
\end{align*} 
Substituting and using (\ref{eqns:LeviCivitaSymbol}) we get that in all cases  $\epsilon_{ij1} \epsilon_{pq1} =  \delta_{ip} \delta_{jq} - \delta_{iq} \delta_{jp}$.\\
Similar results can be shown for $k=2,3$. Finally, the summation (\ref{eqns:LiviCivitaEpsilonDelta}) holds because $i,j,p,q$ substitutions that yield a non-vanishing for a specific value of $k$ cause the other $k$-terms to vanish.
\end{proof}

\begin{lem}
Let $\overrightarrow{a}$ be a vector of $x,y,z$-differentiable entries. Then 
\begin{align}
\label{eqns:doubleCrossProdDecomposition}
\left( \overrightarrow{a} \times (\nabla \times \overrightarrow{E}) \right)_i = \overrightarrow{a} \cdot \left( \dfrac{\partial \overrightarrow{E}}{\partial x_i}- \nabla E_i \right)
\end{align}
\end{lem}

\begin{proof}
Using the Levi-Civita symbol (\ref{eqns:LeviCivitaSymbol}), 
\begin{align*}
\left( \overrightarrow{a} \times (\nabla \times \overrightarrow{E}) \right)_i  = \epsilon_{ijk}a_j\left( \epsilon_{kpq}\dfrac{\partial E_q}{\partial x_p} \right) = \epsilon_{ijk} \epsilon_{kpq} a_j \dfrac{\partial E_q}{\partial x_p}
\end{align*}
By lemma (\ref{lem:LiviCivitaEpsDelta})
\begin{align*}
\left( \overrightarrow{a} \times (\nabla \times \overrightarrow{E}) \right)_i  =  \left( \delta_{ip}\delta{jq} - \delta_{iq}\delta_{jp}\right) a_j \dfrac{\partial E_q}{\partial x_p}
\end{align*}
but
\begin{align*}
\delta_{ip}\delta_{jq}a_j \dfrac{\partial E_q}{\partial x_p} &= a_j \dfrac{\partial E_j}{\partial x_i} \\
\delta_{iq}\delta_{jp}a_j \dfrac{\partial E_q}{\partial x_p} &= a_j \dfrac{\partial E_i}{\partial x_j}
\end{align*}
Hence
\begin{align*}
\left( \overrightarrow{a} \times (\nabla \times \overrightarrow{E}) \right)_i = a_j\left( \dfrac{\partial E_j}{\partial x_i} - \dfrac{\partial E_i}{\partial x_j} \right)
\end{align*}
Explicit summation yields
\begin{align*}
\left( \overrightarrow{a} \times (\nabla \times \overrightarrow{E}) \right)_i &= 
a_1\left( \dfrac{\partial E_1}{\partial x_i} - \dfrac{\partial E_i}{\partial x_1} \right) +
a_2\left( \dfrac{\partial E_2}{\partial x_i} - \dfrac{\partial E_i}{\partial x_2} \right) +
a_3\left( \dfrac{\partial E_3}{\partial x_i} - \dfrac{\partial E_i}{\partial x_3} \right) \\
&= \overrightarrow{a} \cdot \left( \dfrac{\partial \overrightarrow{E}}{\partial x_i}- \nabla E_i \right)
\end{align*}
as required.
\end{proof}

\begin{cor}
The term (\ref{eqns:E_Wave_extra_term_log_mu}) $i$'th component has the form (\ref{eqns:E_Wave_extra_term_log_mu_expanded}):
\begin{align*}
\left(\mu \nabla \left( \dfrac{1}{\mu}\right) \times \left(\nabla \times \overrightarrow{E} \right) \right)_i = 
\mu \nabla \left(\dfrac{1}{\mu} \right) \cdot \left( \dfrac{\partial \overrightarrow{E}}{\partial x_i}- \nabla E_i \right)
\end{align*}
\end{cor}




\end{appendices}



\newpage
\begin{thebibliography}{2d}
\bibitem{ArfkenWeber2005}
Arfken GB. Weber HJ. Mathematical methods for physicists, 6th ed. Elsevier academc press, Amsterdam, 2005.
\bibitem{Bamberger1979}
Bamberger A. Chavent G. Lailly P. About the stability of the inverse problem in 1-D wave eqaitions - application to the interpretation of sesmic profiles. Appl. Math. Optim. \textbf{5}, 1-47, 1979.
\bibitem{Bao2010}
Error estimates for the recursive linearization of the inverse medium problems. J. Comput. Math. \textbf{28}(6), 1-20, 2010.
\bibitem{BaoHouLi2007}
Bao G. Hou S. Li P. Inverse scattering by a continuation method with initial guess from a direct imaging algorithm. J. Comput. Phys. \textbf{227}, 755-762, 2007.
\bibitem{BaoLi2005}
Bao G. Li P. Inverse medium scattering for the Helmholtz equation at fixed frequency. Inverse problems \textbf{21}, 1621-1641, 2005.
\bibitem{BenderOrszag1999}
Bender CM. Orszag SA. Advanced mathematical methods for scientists and engineers 1. Asymptotic methods and perturbation theory, 2nd ed. Springer, New York, 1999.
\bibitem{BirkhoffRota1989}
Birkhoff G. Rota GC. Ordinary differential equations, 4th ed. Wiley, New York, 1989.
%\bibitem{Gustaffson1995}
%Gustaffson B. Kreiss HO. Oliger J. Time dependent problems and difference methods. Wiley, New York, 1995.
\bibitem{Bleistein1984}
Bleistein N. Mathematical method for wave phenomena. Academic Press, Orlando, 1984.
\bibitem{ColtonKress1998}
Colton D. Kress R. Inverse acoustic and electromagnetic scattering theory. Springer, Berlin, 1998 
\bibitem{Hecht2002}
Hecht E. Optics, 4th ed. Addison Wesley, San Francisco, CA, 2002.
\bibitem{ItoJinZou2012}
Ito K. Jin B. Zou J. A direct sampling method to an inverse medium scattering problem. Inverse problems \textbf{28}, 2012.
\bibitem{KashdanTurkel2006}
Kashdan E. Turkel E. A high-order accurate method for frequency domain Maxwell equations with discontinuous coefficients. J. Sci. Comput. \textbf{27}, Nos. 1-3, 2006.  
\bibitem{Keller1995}
Keller JB. Asymptotic methods for partial differential equations: the reduced wave equations and Maxwell's equations In: Keller JB, McLaughlin DW, Papanicolaou GC eds. Surveys in applied mathematics 1. Plenum, New York, 1995.
\bibitem{Kirsch2011}
Kirsch A. An introduction to the mathematical theory of inverse problems, 2nd ed. Springer, New York, 2011.
\bibitem{KravstovOrlov1990}
Kravstov YA. Orlov YI. Geometrical optics of inhomogeneous media. Springer, Berlin, 1990.
\bibitem{Kress1997}
Kress R. Numerical analysis. Springer, New York, 1997.
\bibitem{LiZou2012}
Li J. Zou J. A direct sampling method for inverse scattering using far-field data. Inverse problems and imaging, \textbf{10}, 2012.
\bibitem{Murdock1991}
Murdock JA. Perturbations, theory and methods. Wiley, New York, 1991.
\bibitem{Nayfeh1981}
Nayfeh AH. Introduction to preturbation methods. Wiley, New York, 1981.
\bibitem{Nichelatti2002}
Nichelatti E. Complex refractive index of a slab from reflectance and transmittance: analytical solution. J. Opt. A: Pure Appl. Opt. \textbf{4}, 400-403, 2002.
\bibitem{Peskin1977}
Peskin CS. Numerical analysis of blood flow in the heart. J. Comput. Phys. \textbf{25}, 220-225, 1977.
\bibitem{Peskin2002}
The immersed boundary method. Acta Numerica \textbf{11}, 479-517, 2002
\bibitem{SecklerKeller1959}
Seckler BD. Keller JB. Asymptotic theory of diffraction in inhomogeneous media. J.A.S.A. \textbf{31}, 192-205, 1959.  
%\bibitem{Strikwerda2004}
%Strikwerda J. Finite difference schemes and partial differential equation, 2nd ed. SIAM,  Wadsworth and Brooks / Cole, Pacific Grove, CA, 2004
\bibitem{TornbergEngquist2003}
Tornberg AK. Engquist B. Regularization techniques for numerical approximation of PDEs with singularities. J. Sci. Comp. \textbf{19}, Nos. 1-3, 2003
\end{thebibliography}
\end{document}